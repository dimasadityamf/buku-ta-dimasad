% Atur variabel berikut sesuai namanya

% nama
\newcommand{\name}{Dimas Aditya Maulana Fajri}
\newcommand{\authorname}{Fajri, Dimas Aditya Maulana}
\newcommand{\nickname}{Dimas}
\newcommand{\advisor}{Arief Kurniawan, S.T., M.T.}
\newcommand{\coadvisor}{Dr. Eko Mulyanto Yuniarno, S.T., M.T.}
\newcommand{\examinerone}{Penguji 1}
\newcommand{\examinertwo}{Penguji 2}
\newcommand{\examinerthree}{Penguji 2}
\newcommand{\headofdepartment}{Dr. Supeno Mardi Susiki Nugroho, S.T., M.T}

% identitas
\newcommand{\nrp}{0721 19 4000 0012}
\newcommand{\advisornip}{19740907 200212 1 001}
\newcommand{\coadvisornip}{19680601 199512 1 009}
\newcommand{\examineronenip}{-}
\newcommand{\examinertwonip}{-}
\newcommand{\examinerthreenip}{-}
\newcommand{\headofdepartmentnip}{19700313 199512 1 001}

% judul
\newcommand{\tatitle}{PREDIKSI JUMLAH KALORI YANG TERBAKAR SAAT BEROLAHRAGA DENGAN TREADMILL BERBASIS KAMERA MENGGUNAKAN \emph{CONVOLUTIONAL NEURAL NETWORK}}
%\newcommand{\tatitle}{PREDIKSI JUMLAH KALORI YANG TERBAKAR SAAT BEROLAHRAGA DENGAN TREADMILL BERBASIS KAMERA MENGGUNAKAN CNN}
\newcommand{\engtatitle}{\emph{PREDICTION OF CALORIES BURNED WHEN EXERCISING ON A TREADMILL WITH CAMERA-BASED USING A CONVOLUTIONAL NEURAL NETWORK}}

% tempat
\newcommand{\place}{Surabaya}

% jurusan
\newcommand{\studyprogram}{Teknik Komputer}
\newcommand{\engstudyprogram}{Computer Engineering}

% fakultas
\newcommand{\faculty}{Teknologi Elektro dan Informatika Cerdas}
\newcommand{\engfaculty}{Intelligent Electrical and Informatics Technology}

% singkatan fakultas
\newcommand{\facultyshort}{FTEIC}
\newcommand{\engfacultyshort}{ELECTICS}

% departemen
\newcommand{\department}{Teknik Komputer}
\newcommand{\engdepartment}{Computer Engineering}

% kode mata kuliah
\newcommand{\coursecode}{EC224801}
