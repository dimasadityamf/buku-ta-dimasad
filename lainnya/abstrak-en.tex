\chapter*{ABSTRACT}
\begin{center}
  \large
  \textbf{PREDICTION OF CALORIES BURNED WHEN EXERCISING ON A \emph{TREADMILL} WITH CAMERA-BASED USING A \emph{CONVOLUTIONAL NEURAL NETWORK}}
\end{center}
% Menyembunyikan nomor halaman
\thispagestyle{empty}

\begin{flushleft}
  \setlength{\tabcolsep}{0pt}
  \bfseries
  \begin{tabular}{lc@{\hspace{6pt}}l}
  Student Name / NRP&: &Dimas Aditya Maulana Fajri / 07211940000012\\
  Department&: &Computer Engineering ELECTICS - ITS\\
  Advisor&: &1. Arief Kurniawan, S.T, M.T.\\
  & & 2. Dr. Eko Mulyanto Yuniarno, S.T, M.T.\\
  \end{tabular}
  \vspace{4ex}
\end{flushleft}
\textbf{Abstract}

% Isi Abstrak
Obesity is a condition where there is accumulation of fat in a person's body which causes the body weight to be above normal. Imbalance of calories consumed and used causes excess weight. One of the activities that can reduce excess weight is exercise that has good quality activities. Exercising on a treadmill is one of the activities that can be carried out and measures the calories burned. However, the calculation of calories on a treadmill still depends on each tool. This research creates a system that can predict calories burned while exercising on a treadmill using video images with a camera. The method used is by using video data which is then estimated for poses for body postures. The estimation result extract is continued for classification using a \emph{Convolutional Neural Network} (CNN). The results of the detection are in the form of many steps and time for calorie prediction. Predictions using linear regression and calculations with the \emph{Metabolic Equivalent of Task} (MET). The results of the tests carried out obtained step detection test results with an accuracy of 96.14\% and with a realtime accuracy of 81.35\%. The results of the calorie prediction test used regression with an accuracy of 80.93\% and an accuracy of 67.05\% in real time. While the calorie prediction test uses a formula calculation with an accuracy of 57.49\% and with a realtime accuracy of 46.64\%.

\vspace{2ex}
\noindent
\textbf{Keywords: \emph{Obesity, Calories, Detection, Prediction, Image}}