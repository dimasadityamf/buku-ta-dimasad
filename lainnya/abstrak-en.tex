\chapter*{ABSTRACT}
\begin{center}
  \large
  \textbf{PREDICTION OF CALORIES BURNED WHEN EXERCISING ON A \emph{TREADMILL} WITH CAMERA-BASED USING A \emph{CONVOLUTIONAL NEURAL NETWORK}}
\end{center}
% Menyembunyikan nomor halaman
\thispagestyle{empty}

\begin{flushleft}
  \setlength{\tabcolsep}{0pt}
  \bfseries
  \begin{tabular}{lc@{\hspace{6pt}}l}
  Student Name / NRP&: &Dimas Aditya Maulana Fajri / 07211940000012\\
  Department&: &Computer Engineering ELECTICS - ITS\\
  Advisor&: &1. Arief Kurniawan, S.T, M.T.\\
  & & 2. Dr. Eko Mulyanto Yuniarno, S.T, M.T.\\
  \end{tabular}
  \vspace{4ex}
\end{flushleft}
\textbf{Abstract}

% Isi Abstrak
Obesity is a condition where there is accumulation of fat in a person's body which causes the body weight to be above normal. Imbalance of calories consumed and used causes excess weight. One of the activities that can reduce excess weight is exercise that has good quality activities. Exercising on a treadmill is one of the activities that can be carried out and measures the calories burned. However, calculating calories on a treadmill is still not accurate and practical. This research creates a system that can predict calories burned while exercising on a treadmill using video images with a camera. The method used is by acquiring image data which is then detected and estimated for poses for body postures. Extract detection results are continued for classification using CNN. The results of the detection are in the form of many steps and time for calorie prediction. Prediction using linear regression with the dependent variable the number of calories burned. The final result is expected to be able to predict the number of calories burned from the video image.

\vspace{2ex}
\noindent
\textbf{Keywords: \emph{Obesity, Calories, Prediction, Image}}