\begin{center}
  \Large
  \textbf{KATA PENGANTAR}
\end{center}

\addcontentsline{toc}{chapter}{KATA PENGANTAR}

\vspace{2ex}

% Ubah paragraf-paragraf berikut dengan isi dari kata pengantar

Puji dan syukur kehadirat Allah SWT yang memberikan rahmat serta kemuliaan-Nya sehingga penulis dapat menyelesaikan laporan Tugas Akhir dengan judul "Prediksi Jumlah Kalori yang Terbakar Saat Berolahraga dengan Treadmill Berbasis Kamera Menggunakan \emph{Convolutional Neural Network}.

Penelitian ini disusun dalam rangka pemenuhan bidang riset di Departemen Teknik Komputer, serta digunakan sebagai persyaratan menyelesaikan pendidikan S1. Penelitian ini dapat terselesaikan tidak lepas dari bantuan berbagai pihak. Oleh karena itu, penulis mengucapkan terima kasih kepada:

\begin{enumerate}[nolistsep]

  \item Keluarga, Ibu, Bapak dan Saudara tercinta yang telah memberikan dorongan spiritual dan material dalam penyelesaian penelitian ini.

  \item Bapak Dr. Supeno Mardi Susiki Nugroho, S.T., M.T. selaku Kepala Departemen Teknik Komputer, Fakultas Teknologi Elektro dan Informatika Cerdas (FTEIC), Institut Teknologi Sepuluh Nopember.

  \item Bapak Arief Kurniawan, S.T., M.T. selaku dosen pembimbing I dan Bapak Dr. Eko Mulyanto Yuniarno, S.T., M.T. selaku dosen pembimbing II yang selalu memberikan arahan selama mengerjakan penelitian tugas akhir ini.

  \item Bapak-ibu dosen pengajar Departemen Teknik Komputer, atas pengajaran, bimbingan, serta perhatian yang diberikan kepada penulis selama proses perkuliahan.
  
  \item Mohammad Alfaris Fernanda dan Abimanyu Septian Triyananda yang telah membantu dalam melancarkan pengerjaan tugas akhir ini.

  \item Seluruh teman-teman dari angkatan e59 dan Teknik Komputer 2019.

  \item Rekan-rekan bimbingan seperjuangan Laboratorium Multimedia Internet of Things yang sudah berjuang bersama, menemani dan membantu mengerjakan tugas akhir ini.

\end{enumerate}

Akhir kata, bahwa kesempurnaan hanya milik Allah SWT dan untuk itu penulis memohon segenap kritik dan saran yang membangun. Semoga penelitian ini dapat memberikan manfaat bagi kita semua. Amin.

\begin{flushright}
  \begin{tabular}[b]{c}
    \place{}, \MONTH{} \the\year{} \\
    \\
    \\
    \\
    \\
    \name{}
  \end{tabular}
\end{flushright}
