\chapter*{ABSTRAK}
\begin{center}
  \large
  \textbf{PREDIKSI JUMLAH KALORI YANG TERBAKAR SAAT BEROLAHRAGA DENGAN \emph{TREADMILL} BERBASIS KAMERA MENGGUNAKAN \emph{CONVOLUTIONAL NEURAL NETWORK}}
\end{center}
\addcontentsline{toc}{chapter}{ABSTRAK}
% Menyembunyikan nomor halaman
\thispagestyle{empty}

\begin{flushleft}
  \setlength{\tabcolsep}{0pt}
  \bfseries
  \begin{tabular}{ll@{\hspace{6pt}}l}
    Nama Mahasiswa / NRP&:& Dimas Aditya Maulana Fajri / 07211940000012\\
    Departemen&:& Teknik Komputer FTEIC - ITS\\
    Dosen Pembimbing&:& 1. Arief Kurniawan, S.T, M.T.\\
    & & 2. Dr. Eko Mulyanto Yuniarno, S.T, M.T.\\
  \end{tabular}
  \vspace{4ex}
\end{flushleft}
\textbf{Abstrak}

% Isi Abstrak
Obesitas merupakan keadaan dimana terdapat penumpukan lemak pada tubuh seseorang yang menyebabkan berat badan berada pada nilai di atas normal. Ketidak seimbangan kalori yang dikonsumsi dan yang digunakan menyebabkan kelebihan berat badan. Salah satu aktivitas yang bisa mengurangi kelebihan berat badan adalah dengan olahraga yang memiliki kualitas aktivitas yang baik. Olahraga pada treadmill merupakan salah satu aktivitas yang dapat dilakukan dan melakukan pengukuran kalori yang terbakar. Namun perhitungan kalori pada treadmill masih bergantung pada masing-masing alat. Penelitian ini membuat sistem yang dapat memprediksi kalori yang terbakar saat olahraga pada treadmill menggunakan citra video dengan kamera. Metode yang digunakan dengan menggunakan data video yang kemudian diestimasi pose untuk postur tubuh. Ekstrak hasil estimasi dilanjutkan untuk klasifikasi menggunakan \emph{Convolutional Neural Network} (CNN). Hasil deteksi berupa banyak langkah dan waktu untuk dilakukan prediksi kalori. Prediksi menggunakan regresi linear dan perhitungan dengan \emph{Metabolic Equivalent of Task} (MET). Hasil pengujian yang dilakukan mendapatkan hasil pengujian deteksi langkah dengan akurasi 96,14\% dan dengan realtime akurasi 81,35\%. Hasil pengujian prediksi kalori menggunakan regresi dengan akurasi 80,93\% dan dengan realtime akurasi 67,05\%. Sedangkan pengujian prediksi kalori menggunakan perhitungan rumus dengan akurasi 57,49\% dan dengan realtime akurasi 46,64\%.

\vspace{2ex}
\noindent
\textbf{Kata Kunci: \emph{Obesitas, Kalori, Deteksi, Prediksi, Citra}}