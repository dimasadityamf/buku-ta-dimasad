\chapter{PENUTUP}
\label{chap:penutup}

% Ubah bagian-bagian berikut dengan isi dari penutup

\section{Kesimpulan}
\label{sec:kesimpulan}

Berdasarkan hasil pengujian yang dilakukan dengan perhitungan analisa performa dan hasil yang dilakukan mengenai sistem prediksi kalori yang terbakar, didapatkan beberapa kesimpulan sebagai berikut:

\begin{enumerate}[nolistsep]

  \item Berdasarkan hasil pengujian dataset yang digunakan sebagai klasifikasi langkah dengan melakukan pengambilan data didapatkan hasil akurasi \emph{training} sebesar 0.951 dan \emph{validation} sebesar 0.977, sedangkan hasil \emph{loss} pada \emph{training} sebesar 0.127 dan \emph{validation} sebesar 0.059. Pengujian \emph{testing} model yang digunakan didapatkan hasil akurasi sebesar 95\% dengan hasil deteksi kelas kanan dengan benar sebanyak 170 sampel (100\%) dan kelas kiri sebanyak 161 sampel (91\%).

  \item Pengujian deteksi langkah dengan model yang dibuat untuk melakukan klasifikasi langkah didapatkan hasil akurasi rata-rata deteksi berdasarkan model klasifikasi sebesar 96,14\% dengan hasil eror rata-rata sebesar 3,86\%.
  
  \item Pengujian prediksi kalori dengan sistem yang digunakan berdasarkan metode regresi didapat hasil akurasi sebesar 80,93\% dengan eror sebesar 19,07\% dan berdasarkan metode perhitungan rumus didapat hasil akurasi rata-rata sebesar 57,49\% dengan eror rata-rata sebesar 42,51\%.
  
  \item Pengujian performa berdasarkan skenario yang dilakukan mendapatkan hasil performa terbaik untuk hasil deteksi langkah pada kondisi intensitas cahaya tinggi dengan akurasi rata-rata sebesar 95,54\%. Kemudian hasil performa terbaik untuk prediksi kalori dengan regresi pada kondisi intensitas cahaya rendah dengan akurasi rata-rata sebesar 73,68\%. Hasil performa terbaik untuk prediksi kalori dengan perhitungan rumus pada kondisi intensitas cahaya tinggi dengan akurasi rata-rata sebesar 68,92\%.
  
  \item Pengujian sistem secara \emph{real time} didapatkan hasil akurasi rata-rata untuk deteksi langkah sebesar 81,35\%, hasil prediksi kalori dengan regresi sebesar 67,05\% dan prediksi kalori dengan perhitungan rumus sebesar 46,64\%.

\end{enumerate}

\section{Saran}
\label{chap:saran}

Untuk pengembangan lebih lanjut pada penelitian ini terdapat beberapa saran oleh penulis yang dapat dilakukan antara lain:

\begin{enumerate}[nolistsep]

  \item Menambah jumlah dataset untuk model panjang langkah dan regresi yang dapat diperbanyak dalam variasi pengambilan data untuk pembuatan dataset tersebut.
  
  \item Mengoptimasi metode dalam klasifikasi langkah agar dapat melakukan deteksi hasil klasifikasi yang lebih baik.
  
  \item Melanjutkan penelitian agar dapat dijalankan secara \emph{realtime} dan terintegrasi sebagai aplikasi.

\end{enumerate}
