\chapter{PENUTUP}
\label{chap:penutup}

% Ubah bagian-bagian berikut dengan isi dari penutup

\section{Kesimpulan}
\label{sec:kesimpulan}

Berdasarkan hasil pengujian yang dilakukan dengan perhitungan analisa performa dan hasil yang dilakukan mengenai sistem prediksi kalori yang terbakar, didapatkan beberapa kesimpulan sebagai berikut:

\begin{enumerate}[nolistsep]

  \item Penelitian sistem prediksi kalori yang terbakar saat berolahraga dengan treadmill berbasis kamera dengan menggunakan \emph{convolutional neural network} dapat bekerja dengan baik dengan melakukan hasil akhir nilai prediksi kalori yang terbakar menggunakan kamera.

  \item Berdasarkan hasil pengujian dataset yang digunakan sebagai klasifikasi langkah dengan melakukan pengambilan data didapatkan hasil akurasi \emph{training} sebesar 0.951 dan \emph{validation} sebesar 0.977, sedangkan hasil \emph{loss} pada \emph{training} sebesar 0.127 dan \emph{validation} sebesar 0.059.

  \item Berdasarkan hasil pengujian \emph{testing} model yang digunakan didapatkan hasil akurasi sebesar 95\% dengan hasil deteksi kelas kanan dengan benar sebanyak 170 sampel (100\%) dan kelas kiri sebanyak 161 sampel (91\%).

  \item Pengujian hasil deteksi dengan model yang dibuat untuk melakukan klasifikasi langkah didapatkan hasil akurasi deteksi berdasarkan model klasifikasi sebesar 99,36\% dengan hasil error sebesar 0,64\%.
  
  \item Pengujian hasil prediksi dengan sistem yang digunakan berdasarkan metode regresi didapat hasil akurasi sebesar 93,61\% dengan error sebesar 6,39\% dan berdasarkan metode perhitungan rumus didapat hasil akurasi sebesar 81,03\% dengan error sebesar 18,97\%.

\end{enumerate}

\section{Saran}
\label{chap:saran}

Untuk pengembangan lebih lanjut pada \lipsum[1][1-3] antara lain:

\begin{enumerate}[nolistsep]

  \item Memperbaiki \lipsum[2][1-3]

  \item \lipsum[2][4-6]

  \item \lipsum[2][7-10]

\end{enumerate}
