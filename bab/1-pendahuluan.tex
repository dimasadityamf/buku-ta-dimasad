\chapter{PENDAHULUAN}
\label{chap:pendahuluan}

% Ubah bagian-bagian berikut dengan isi dari pendahuluan

\section{Latar Belakang}
\label{sec:latarbelakang}

Obesitas merupakan keadaan dimana terdapat penumpukan lemak pada tubuh seseorang yang menyebabkan berat badan berada pada nilai di atas normal. Indikasi yang dapat digunakan untuk menilai jika seseorang menderita obesitas berdasarkan nilai \emph{body mass index} (BMI) yang lebih dari 30 kg/m2. Obesistas disebabkan oleh kalori yang dikonsumsi tidak seimbang dengan kalori yang digunakan oleh tubuh. Salah satu hal yang dapat digunakan untuk mencegah obesitas dan mengurangi kelebihan berat badan dengan melakukan olahraga.

Olahraga merupakan suatu bentuk aktivitas fisik dalam kegiatan jasmani yang dilakukan secara terstruktur dengan melibatkan pergerakan tubuh secara berulang-ulang. Aktvitas olahraga dilakukan dengan tujuan untuk memelihara kesehatan dan memperkuat otot-otot tubuh. Olahraga menjadi kegiatan yang sangat dekat dengan aktivitas manusia sebagai salah satu kebutuhan hidup dalam memberikan manfaat berupa kesehatan dan kebugaran tubuh. 

Aktivitas olahraga dinilai bermanfaat dan sesuai prosedur dengan melihat bagaimana kualitas aktivitas olahraga yang telah dilakukan. Kualitas aktivitas olahraga dapat diukur berdasarkan jumlah energi yang dikeluarkan selama melakukan aktivitas olahraga. Energi yang dikeluarkan akan membantu meningkatkan jumlah pembakaran kalori pada tubuh. Jumlah energi yang dikeluarkan selama melakukan aktivitas olahraga akan berbeda-beda tergantung dari jenis aktivitas, durasi dan beberapa faktor pada individu.


\section{Permasalahan}
\label{sec:permasalahan}

Aktivitas yang dilakukan pada treadmill dengan perhitungan pembakaran kalori hanya dapat dilakukan pada beberapa jenis treadmill yang memiliki sistem perhitungannya. Treadmill dengan sistem yang kompleks memungkinkan memiliki harga jual yang lebih tinggi dari treadmill yang sederhana. Sistem yang digunakan hanya bisa digunakan pada treadmill saja tanpa bisa terhubung satu sama lain antar alat. Hal ini membuat pengumpulan data dari setiap aktivitas yang dilakukan tidak tercatat dengan baik. Oleh karena itu, diperlukan sistem prediksi jumlah kalori yang terbakar yang lebih praktis dan mudah digunakan untuk berolahraga pada treadmill. 


\section{Batasan Masalah}
\label{sec:batasanmasalah}

Adapun batasan masalah dalam memfokuskan permasalahan yang dirumuskan pada penelitian ini adalah:

\begin{enumerate}[nolistsep]

  \item Metode yang digunakan dalam melakukan proses deteksi pose tubuh menggunakan Python dengan library OpenCV yaitu MediaPipe.

  \item Deteksi yang digunakan pada MediaPipe berfokus pada deteksi pose tubuh.

  \item Aktivitas fisik yang dideteksi berfokus hanya pada kegiatan olahraga menggunakan Treadmill.

  \item Akuisisi data citra diambil menggunakan perangkat kamera.

  \item Hasil deteksi berupa nilai prediksi perhitungan kalori yang terbakar selama aktivitas fisik yang dilakukan.

  \item Faktor kemiringan digunakan pada level 0 atau sama pada setiap percobaan.

\end{enumerate}

\section{Tujuan}
\label{sec:Tujuan}

Tujuan dari penelitian tugas akhir ini adalah membuat sistem prediksi jumlah kalori yang terbakar saat berolahraga pada treadmill dengan melakukan prediksi kalori menggunakan citra dari kamera.


\section{Manfaat}

Adapun manfaat yang didapat pada penelitian ini adalah dapat membuat sistem yang lebih praktis dalam menentukan prediksi pembakaran kalori yang bisa digunakan disegala jenis treadmill dan dapat menggunakan satu sistem untuk berbagai macam jenis treadill dalam melakukan prediksi pebakaran kalori dalam penurunan berat badan.
