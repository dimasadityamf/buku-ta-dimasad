\begin{center}
  \large\textbf{ABSTRACT}
\end{center}

\addcontentsline{toc}{chapter}{ABSTRACT}

\vspace{2ex}

\begingroup
% Menghilangkan padding
\setlength{\tabcolsep}{0pt}

\noindent
\begin{tabularx}{\textwidth}{l >{\centering}m{3em} X}
  \emph{Name}     & : & \name{}         \\

  \emph{Title}    & : & \engtatitle{}   \\

  \emph{Advisors} & : & 1. \advisor{}   \\
                  &   & 2. \coadvisor{} \\
\end{tabularx}
\endgroup

% Ubah paragraf berikut dengan abstrak dari tugas akhir dalam Bahasa Inggris
\emph{Obesity is a condition where there is accumulation of fat in a person's body which causes the body weight to be above normal. Imbalance of calories consumed and used causes excess weight. One of the activities that can reduce excess weight is exercise that has good quality activities. Exercising on a treadmill is one of the activities that can be carried out and measures the calories burned. However, calculating calories on a treadmill is still not accurate and practical. This research creates a system that can predict calories burned while exercising on a treadmill using video images with a camera. The method used is to collect data which is then detected by pose for body posture. The model was created using CNN. The results of the detection are in the form of many steps and time for calorie prediction. Predictions using linear regression and MET calculations. The final result is expected to be able to predict the number of calories burned from the video image.}

% Ubah kata-kata berikut dengan kata kunci dari tugas akhir dalam Bahasa Inggris
\emph{Keywords}: \emph{Obesity}, \emph{Calories}, \emph{Prediction}, \emph{Video}.
