\begin{center}
  \large\textbf{ABSTRAK}
\end{center}

\addcontentsline{toc}{chapter}{ABSTRAK}

\vspace{2ex}

\begingroup
% Menghilangkan padding
\setlength{\tabcolsep}{0pt}

\noindent
\begin{tabularx}{\textwidth}{l >{\centering}m{2em} X}
  Nama Mahasiswa    & : & \name{}         \\

  Judul Tugas Akhir & : & \tatitle{}      \\

  Pembimbing        & : & 1. \advisor{}   \\
                    &   & 2. \coadvisor{} \\
\end{tabularx}
\endgroup

% Ubah paragraf berikut dengan abstrak dari tugas akhir
Obesitas merupakan keadaan dimana terdapat penumpukan lemak pada tubuh seseorang yang menyebabkan berat badan berada pada nilai di atas normal. Ketidak seimbangan kalori yang dikonsumsi dan yang digunakan menyebabkan kelebihan berat badan. Salah satu aktivitas yang bisa mengurangi kelebihan berat badan adalah dengan olahraga yang memiliki kualitas aktivitas yang baik. Olahraga pada treadmill merupakan salah satu aktivitas yang dapat dilakukan dan melakukan pengukuran kalori yang terbakar. Namun perhitungan kalori pada treadmill masih belum akurat dan praktis. Penelitian ini membuat sistem yang dapat memprediksi kalori yang terbakar saat olahraga pada treadmill menggunakan citra video dengan kamera. Metode yang digunakan dengan melakukan pengambilan data yang kemudian dideteksi pose untuk postur tubuh. Model dibuat menggunakan CNN. Hasil deteksi berupa banyak langkah dan waktu untuk dilakukan prediksi kalori. Prediksi menggunakan regresi linear dan perhitungan MET. Hasil akhir yang diharapkan dapat memprediksi jumlah kalori yang terbakar dari citra video.

% Ubah kata-kata berikut dengan kata kunci dari tugas akhir
Kata Kunci: Obesitas, Kalori, Prediksi, Video.
