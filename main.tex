% Judul dokumen
\title{Buku Tugas Akhir ITS}
\author{Fajri, Dimas Aditya Maulana}

% Pengaturan ukuran teks dan bentuk halaman dua sisi
\documentclass[12pt,twoside]{report}

% Pengaturan ukuran halaman dan margin
\usepackage[a4paper,top=30mm,left=30mm,right=20mm,bottom=25mm]{geometry}

% Pengaturan ukuran spasi
\usepackage[singlespacing]{setspace}

% Pengaturan detail pada file PDF
\usepackage[pdfauthor={\@author},bookmarksnumbered,pdfborder={0 0 0}]{hyperref}

% Pengaturan jenis karakter
\usepackage[utf8]{inputenc}

% Pengaturan pewarnaan
\usepackage[table,xcdraw]{xcolor}

% Pengaturan kutipan artikel
\usepackage[style=apa, backend=biber]{biblatex}

% Package lainnya
\usepackage{changepage}
\usepackage{enumitem}
\usepackage{eso-pic}
\usepackage{txfonts} % Font times
\usepackage{etoolbox}
\usepackage{graphicx}
\usepackage{lipsum}
\usepackage{longtable}
\usepackage{tabularx}
\usepackage{wrapfig}
\usepackage{float}

% Definisi untuk "Hati ini sengaja dikosongkan"
\patchcmd{\cleardoublepage}{\hbox{}}{
  \thispagestyle{empty}
  \vspace*{\fill}
  \begin{center}\textit{[Halaman ini sengaja dikosongkan]}\end{center}
  \vfill}{}{}

% Pengaturan penomoran halaman
\usepackage{fancyhdr}
\fancyhf{}
\renewcommand{\headrulewidth}{0pt}
\pagestyle{fancy}
\fancyfoot[LE,RO]{\thepage}
\patchcmd{\chapter}{plain}{fancy}{}{}
\patchcmd{\chapter}{empty}{plain}{}{}

% Command untuk bulan
\newcommand{\MONTH}{%
  \ifcase\the\month
  \or Januari% 1
  \or Februari% 2
  \or Maret% 3
  \or April% 4
  \or Mei% 5
  \or Juni% 6
  \or Juli% 7
  \or Agustus% 8
  \or September% 9
  \or Oktober% 10
  \or November% 11
  \or Desember% 12
  \fi}
\newcommand{\ENGMONTH}{%
  \ifcase\the\month
  \or January% 1
  \or February% 2
  \or March% 3
  \or April% 4
  \or May% 5
  \or June% 6
  \or July% 7
  \or August% 8
  \or September% 9
  \or October% 10
  \or November% 11
  \or December% 12
  \fi}

% Pengaturan format judul bab
\usepackage{titlesec}
\titleformat{\chapter}[display]{\bfseries\Large}{BAB \centering\Roman{chapter}}{0ex}{\vspace{0ex}\centering}
\titleformat{\section}{\bfseries\large}{\MakeUppercase{\thesection}}{1ex}{\vspace{1ex}}
\titleformat{\subsection}{\bfseries\large}{\MakeUppercase{\thesubsection}}{1ex}{}
\titleformat{\subsubsection}{\bfseries\large}{\MakeUppercase{\thesubsubsection}}{1ex}{}
\titlespacing{\chapter}{0ex}{0ex}{4ex}
\titlespacing{\section}{0ex}{1ex}{0ex}
\titlespacing{\subsection}{0ex}{0.5ex}{0ex}
\titlespacing{\subsubsection}{0ex}{0.5ex}{0ex}

% Atur variabel berikut sesuai namanya

% nama
\newcommand{\name}{Dimas Aditya Maulana Fajri}
\newcommand{\authorname}{Fajri, Dimas Aditya Maulana}
\newcommand{\nickname}{Dimas}
\newcommand{\advisor}{Arief Kurniawan, S.T., M.T.}
\newcommand{\coadvisor}{Dr. Eko Mulyanto Yuniarno, S.T., M.T.}
\newcommand{\examinerone}{Penguji 1}
\newcommand{\examinertwo}{Penguji 2}
\newcommand{\examinerthree}{Penguji 2}
\newcommand{\headofdepartment}{Dr. Supeno Mardi Susiki Nugroho, S.T., M.T}

% identitas
\newcommand{\nrp}{0721 19 4000 0012}
\newcommand{\advisornip}{19740907 200212 1 001}
\newcommand{\coadvisornip}{19680601 199512 1 009}
\newcommand{\examineronenip}{18560710 194301 1 001}
\newcommand{\examinertwonip}{18560710 194301 1 001}
\newcommand{\examinerthreenip}{18560710 194301 1 001}
\newcommand{\headofdepartmentnip}{19700313 199512 1 001}

% judul
\newcommand{\tatitle}{PREDIKSI JUMLAH KALORI YANG TERBAKAR SAAT BEROLAHRAGA DENGAN TREADMILL BERBASIS KAMERA MENGGUNAKAN \emph{CONVOLUTIONAL NEURAL NETWORK}}
%\newcommand{\tatitle}{PREDIKSI JUMLAH KALORI YANG TERBAKAR SAAT BEROLAHRAGA DENGAN TREADMILL BERBASIS KAMERA MENGGUNAKAN CNN}
\newcommand{\engtatitle}{\emph{PREDICTION OF CALORIES BURNED WHEN EXERCISING ON A TREADMILL WITH CAMERA-BASED USING A CONVOLUTIONAL NEURAL NETWORK}}

% tempat
\newcommand{\place}{Surabaya}

% jurusan
\newcommand{\studyprogram}{Teknik Komputer}
\newcommand{\engstudyprogram}{Computer Engineering}

% fakultas
\newcommand{\faculty}{Teknologi Elektro dan Informatika Cerdas}
\newcommand{\engfaculty}{Intelligent Electrical and Informatics Technology}

% singkatan fakultas
\newcommand{\facultyshort}{FTEIC}
\newcommand{\engfacultyshort}{ELECTICS}

% departemen
\newcommand{\department}{Teknik Komputer}
\newcommand{\engdepartment}{Computer Engineering}

% kode mata kuliah
\newcommand{\coursecode}{EC224801}


% Tambahkan format tanda hubung yang benar di sini
\hyphenation{
  ro-ket
  me-ngem-bang-kan
  per-hi-tu-ngan
  tek-no-lo-gi
  me-la-ku-kan
  ber-so-si-al-i-sa-si
  ber-da-sar-kan
  an-ta-ra
}

% Menambahkan resource daftar pustaka
\addbibresource{pustaka/pustaka.bib}

% Pengaturan format potongan kode
\usepackage{listings}
\definecolor{comment}{RGB}{0,128,0}
\definecolor{string}{RGB}{255,0,0}
\definecolor{keyword}{RGB}{0,0,255}
\lstdefinestyle{codestyle}{
  commentstyle=\color{comment},
  stringstyle=\color{string},
  keywordstyle=\color{keyword},
  basicstyle=\footnotesize\ttfamily,
  numbers=left,
  numberstyle=\tiny,
  numbersep=5pt,
  frame=lines,
  breaklines=true,
  prebreak=\raisebox{0ex}[0ex][0ex]{\ensuremath{\hookleftarrow}},
  showstringspaces=false,
  upquote=true,
  tabsize=2,
}
\lstset{style=codestyle}

% Isi keseluruhan dokumen
\begin{document}

% Sampul luar Bahasa Indonesia
\newcommand\covercontents{sampul/konten-id.tex}
\input{sampul/sampul-luar.tex}

% Atur ulang penomoran halaman
\setcounter{page}{1}

% Sampul dalam Bahasa Indonesia
\renewcommand\covercontents{sampul/konten-id.tex}
\input{sampul/sampul-luar-tipis.tex}
\clearpage
\cleardoublepage

% Sampul dalam Bahasa Inggris
\renewcommand\covercontents{sampul/konten-en.tex}
\input{sampul/sampul-luar-tipis.tex}
\cleardoublepage

% Label tabel dan gambar dalam bahasa indonesia
\renewcommand{\figurename}{Gambar}
\renewcommand{\tablename}{Tabel}

% Pengaturan ukuran indentasi paragraf
\setlength{\parindent}{2em}

% Pengaturan ukuran spasi paragraf
\setlength{\parskip}{1ex}

% Lembar pengesahan
\chapter*{LEMBAR PENGESAHAN}

% Menyembunyikan nomor halaman
\thispagestyle{empty}

\begin{center}
  % Ubah kalimat berikut dengan judul tugas akhir
  \textbf{PREDIKSI JUMLAH KALORI YANG TERBAKAR SAAT BEROLAHRAGA DENGAN \emph{TREADMILL} BERBASIS KAMERA MENGGUNAKAN \emph{CONVOLUTIONAL NEURAL NETWORK}}
\end{center}

\begingroup
% Pemilihan font ukuran small
\small

\begin{center}
  % Ubah kalimat berikut dengan pernyataan untuk lembar pengesahan
  \textbf{PROPOSAL TUGAS AKHIR} \\
    Diajukan untuk memenuhi salah satu syarat memperoleh gelar
    Sarjana Teknik pada 
    Program Studi S-1 Teknik Kommputer \\
    Departemen Teknik Kommputer \\
    Fakultas Teknologi Elektro dan Informatika Cerdas \\
    Institut Teknologi Sepuluh Nopember
\end{center}

\begin{center}
  % Ubah kalimat berikut dengan nama dan NRP mahasiswa
  Oleh: \textbf{Dimas Aditya Maulana Fajri} \\
  NRP. 0721 19 4000 0012
\end{center}

\begin{center}
  Disetujui Oleh:
\end{center}

\vspace{10ex}

\begingroup
% Menghilangkan padding
\setlength{\tabcolsep}{0pt}

\noindent
\begin{tabularx}{\textwidth}{X c}
  % Ubah kalimat-kalimat berikut dengan nama dan NIP dosen pembimbing pertama
  Arief Kurniawan, S.T, M.T.      &                 \\
  NIP: 19740907 200212 1 001    & (Pembimbing)    \\
                                &                 \\
                                &                 \\
                                &                 \\
  % Ubah kalimat-kalimat berikut dengan nama dan NIP dosen pembimbing kedua
  Dr. Eko Mulyanto Yuniarno, S.T, M.T. &                 \\
  NIP: 19680601 199512 1 009    & (Ko-Pembimbing) \\
\end{tabularx}
\endgroup

\vspace{\fill}

\begin{center}
  % Ubah text dibawah menjadi tempat dan tanggal
  \textbf{SURABAYA} \\
  \textbf{Desember, 2022}
\end{center}
\endgroup

\cleardoublepage
\input{lainnya/lembar-pengesahan-en.tex}
\cleardoublepage

% Pernyataan keaslian
\input{lainnya/pernyataan-keaslian.tex}
\cleardoublepage
\input{lainnya/pernyataan-keaslian-en.tex}
\cleardoublepage

% Nomor halaman pembuka dimulai dari sini
\pagenumbering{roman}

% Abstrak Bahasa Indonesia
\begin{center}
  \large\textbf{ABSTRAK}
\end{center}

\addcontentsline{toc}{chapter}{ABSTRAK}

\vspace{2ex}

\begingroup
% Menghilangkan padding
\setlength{\tabcolsep}{0pt}

\noindent
\begin{tabularx}{\textwidth}{l >{\centering}m{2em} X}
  Nama Mahasiswa    & : & \name{}         \\

  Judul Tugas Akhir & : & \tatitle{}      \\

  Pembimbing        & : & 1. \advisor{}   \\
                    &   & 2. \coadvisor{} \\
\end{tabularx}
\endgroup

% Ubah paragraf berikut dengan abstrak dari tugas akhir
Obesitas merupakan keadaan dimana terdapat penumpukan lemak pada tubuh seseorang yang menyebabkan berat badan berada pada nilai di atas normal. Ketidak seimbangan kalori yang dikonsumsi dan yang digunakan menyebabkan kelebihan berat badan. Salah satu aktivitas yang bisa mengurangi kelebihan berat badan adalah dengan olahraga yang memiliki kualitas aktivitas yang baik. Olahraga pada treadmill merupakan salah satu aktivitas yang dapat dilakukan dan melakukan pengukuran kalori yang terbakar. Namun perhitungan kalori pada treadmill masih bergantung pada masing-masing alat. Penelitian ini membuat sistem yang dapat memprediksi kalori yang terbakar saat olahraga pada treadmill menggunakan citra video dengan kamera. Metode yang digunakan dengan menggunakan data video yang kemudian diestimasi pose untuk postur tubuh. Ekstrak hasil estimasi dilanjutkan untuk klasifikasi menggunakan \emph{Convolutional Neural Network} (CNN). Hasil deteksi berupa banyak langkah dan waktu untuk dilakukan prediksi kalori. Prediksi menggunakan regresi linear dan perhitungan dengan \emph{Metabolic Equivalent of Task} (MET). Hasil pengujian yang dilakukan mendapatkan hasil pengujian deteksi langkah dengan akurasi 96,14\% dan dengan realtime akurasi 81,35\%. Hasil pengujian prediksi kalori menggunakan regresi dengan akurasi 80,93\% dan dengan realtime akurasi 67,05\%. Sedangkan pengujian prediksi kalori menggunakan perhitungan rumus dengan akurasi 57,49\% dan dengan realtime akurasi 46,64\%.

% Ubah kata-kata berikut dengan kata kunci dari tugas akhir
Kata Kunci: Obesitas, Kalori, Deteksi, Prediksi, Video.

\cleardoublepage

% Abstrak Bahasa Inggris
\begin{center}
  \large\textbf{ABSTRACT}
\end{center}

\addcontentsline{toc}{chapter}{ABSTRACT}

\vspace{2ex}

\begingroup
% Menghilangkan padding
\setlength{\tabcolsep}{0pt}

\noindent
\begin{tabularx}{\textwidth}{l >{\centering}m{3em} X}
  \emph{Name}     & : & \name{}         \\

  \emph{Title}    & : & \engtatitle{}   \\

  \emph{Advisors} & : & 1. \advisor{}   \\
                  &   & 2. \coadvisor{} \\
\end{tabularx}
\endgroup

% Ubah paragraf berikut dengan abstrak dari tugas akhir dalam Bahasa Inggris
\emph{Obesity is a condition where there is accumulation of fat in a person's body which causes the body weight to be above normal. Imbalance of calories consumed and used causes excess weight. One of the activities that can reduce excess weight is exercise that has good quality activities. Exercising on a treadmill is one of the activities that can be carried out and measures the calories burned. However, calculating calories on a treadmill is still not accurate and practical. This research creates a system that can predict calories burned while exercising on a treadmill using video images with a camera. The method used is to collect data which is then detected by pose for body posture. The model was created using CNN. The results of the detection are in the form of many steps and time for calorie prediction. Predictions using linear regression and MET calculations. The final result is expected to be able to predict the number of calories burned from the video image.}

% Ubah kata-kata berikut dengan kata kunci dari tugas akhir dalam Bahasa Inggris
\emph{Keywords}: \emph{Obesity}, \emph{Calories}, \emph{Prediction}, \emph{Video}.

\cleardoublepage

% Kata pengantar
\begin{center}
  \Large
  \textbf{KATA PENGANTAR}
\end{center}

\addcontentsline{toc}{chapter}{KATA PENGANTAR}

\vspace{2ex}

% Ubah paragraf-paragraf berikut dengan isi dari kata pengantar

Puji dan syukur kehadirat Allah SWT yang memberikan rahmat serta kemuliaan-Nya sehingga penulis dapat menyelesaikan laporan Tugas Akhir dengan judul "Prediksi Jumlah Kalori yang Terbakar Saat Berolahraga dengan Treadmill Berbasis Kamera Menggunakan \emph{Convolutional Neural Network}.

Penelitian ini disusun dalam rangka pemenuhan bidang riset di Departemen Teknik Komputer, serta digunakan sebagai persyaratan menyelesaikan pendidikan S1. Penelitian ini dapat terselesaikan tidak lepas dari bantuan berbagai pihak. Oleh karena itu, penulis mengucapkan terima kasih kepada:

\begin{enumerate}[nolistsep]

  \item Keluarga, Ibu, Bapak dan Saudara tercinta yang telah memberikan dorongan spiritual dan material dalam penyelesaian penelitian ini.

  \item Bapak Dr. Supeno Mardi Susiki Nugroho, S.T., M.T. selaku Kepala Departemen Teknik Komputer, Fakultas Teknologi Elektro dan Informatika Cerdas (FTEIC), Institut Teknologi Sepuluh Nopember.

  \item Bapak Arief Kurniawan, S.T., M.T. selaku dosen pembimbing I dan Bapak Dr. Eko Mulyanto Yuniarno, S.T., M.T. selaku dosen pembimbing II yang selalu memberikan arahan selama mengerjakan penelitian tugas akhir ini.

  \item Bapak-ibu dosen pengajar Departemen Teknik Komputer, atas pengajaran, bimbingan, serta perhatian yang diberikan kepada penulis selama proses perkuliahan.

  \item Seluruh teman-teman dari angkatan e59 dan Teknik Komputer 2019.

\end{enumerate}

Akhir kata, bahwa kesempurnaan hanya milik Allah SWT dan untuk itu penulis memohon segenap kritik dan saran yang membangun. Semoga penelitian ini dapat memberikan manfaat bagi kita semua. Amin.

\begin{flushright}
  \begin{tabular}[b]{c}
    \place{}, \MONTH{} \the\year{} \\
    \\
    \\
    \\
    \\
    \name{}
  \end{tabular}
\end{flushright}

\cleardoublepage

% Daftar isi
\renewcommand*\contentsname{DAFTAR ISI}
\addcontentsline{toc}{chapter}{\contentsname}
\tableofcontents
\cleardoublepage

% Daftar gambar
\renewcommand*\listfigurename{DAFTAR GAMBAR}
\addcontentsline{toc}{chapter}{\listfigurename}
\listoffigures
\cleardoublepage

% Daftar tabel
\renewcommand*\listtablename{DAFTAR TABEL}
\addcontentsline{toc}{chapter}{\listtablename}
\listoftables
\cleardoublepage

% Nomor halaman isi dimulai dari sini
\pagenumbering{arabic}

% Bab 1 pendahuluan
\chapter{PENDAHULUAN}
\label{chap:pendahuluan}

% Ubah bagian-bagian berikut dengan isi dari pendahuluan

\section{Latar Belakang}
\label{sec:latarbelakang}

Obesitas merupakan keadaan dimana terdapat penumpukan lemak pada tubuh seseorang yang menyebabkan berat badan berada pada nilai di atas normal. Indikasi yang dapat digunakan untuk menilai jika seseorang menderita obesitas berdasarkan nilai \emph{body mass index} (BMI) yang lebih dari 30 kg/m2. Obesistas disebabkan oleh kalori yang dikonsumsi tidak seimbang dengan kalori yang digunakan oleh tubuh. Salah satu hal yang dapat digunakan untuk mencegah obesitas dan mengurangi kelebihan berat badan dengan melakukan olahraga.

Olahraga merupakan suatu bentuk aktivitas fisik dalam kegiatan jasmani yang dilakukan secara terstruktur dengan melibatkan pergerakan tubuh secara berulang-ulang. Aktvitas olahraga dilakukan dengan tujuan untuk memelihara kesehatan dan memperkuat otot-otot tubuh. Olahraga menjadi kegiatan yang sangat dekat dengan aktivitas manusia sebagai salah satu kebutuhan hidup dalam memberikan manfaat berupa kesehatan dan kebugaran tubuh. 

Aktivitas olahraga dinilai bermanfaat dan sesuai prosedur dengan melihat bagaimana kualitas aktivitas olahraga yang telah dilakukan. Kualitas aktivitas olahraga dapat diukur berdasarkan jumlah energi yang dikeluarkan selama melakukan aktivitas olahraga. Energi yang dikeluarkan akan membantu meningkatkan jumlah pembakaran kalori pada tubuh. Jumlah energi yang dikeluarkan selama melakukan aktivitas olahraga akan berbeda-beda tergantung dari jenis aktivitas, durasi dan beberapa faktor pada individu.


\section{Permasalahan}
\label{sec:permasalahan}

Aktivitas yang dilakukan pada treadmill dengan perhitungan pembakaran kalori hanya dapat dilakukan pada beberapa jenis treadmill yang memiliki sistem perhitungannya. Treadmill dengan sistem yang kompleks memungkinkan memiliki harga jual yang lebih tinggi dari treadmill yang sederhana. Sistem yang digunakan hanya bisa digunakan pada treadmill saja tanpa bisa terhubung satu sama lain antar alat. Hal ini membuat pengumpulan data dari setiap aktivitas yang dilakukan tidak tercatat dengan baik. Oleh karena itu, diperlukan sistem prediksi jumlah kalori yang terbakar yang lebih praktis dan mudah digunakan untuk berolahraga pada treadmill. 


\section{Batasan Masalah}
\label{sec:batasanmasalah}

Adapun batasan masalah dalam memfokuskan permasalahan yang dirumuskan pada penelitian ini adalah:

\begin{enumerate}[nolistsep]

  \item Metode yang digunakan dalam melakukan proses deteksi pose tubuh menggunakan Python dengan library OpenCV yaitu MediaPipe.

  \item Deteksi yang digunakan pada MediaPipe berfokus pada deteksi pose tubuh.

  \item Aktivitas fisik yang dideteksi berfokus hanya pada kegiatan olahraga menggunakan Treadmill.

  \item Akuisisi data citra diambil menggunakan perangkat kamera.

  \item Hasil deteksi berupa nilai prediksi perhitungan kalori yang terbakar selama aktivitas fisik yang dilakukan.

  \item Faktor kemiringan digunakan pada level 0 atau sama pada setiap percobaan.

\end{enumerate}

\section{Tujuan}
\label{sec:Tujuan}

Tujuan dari penelitian tugas akhir ini adalah membuat sistem prediksi jumlah kalori yang terbakar saat berolahraga pada treadmill dengan melakukan prediksi kalori menggunakan citra dari kamera.


\section{Manfaat}

Adapun manfaat yang didapat pada penelitian ini adalah dapat membuat sistem yang lebih praktis dalam menentukan prediksi pembakaran kalori yang bisa digunakan disegala jenis treadmill dan dapat menggunakan satu sistem untuk berbagai macam jenis treadill dalam melakukan prediksi pebakaran kalori dalam penurunan berat badan.

\cleardoublepage

% Bab 2 tinjauan pustaka
\chapter{TINJAUAN PUSTAKA}
\label{chap:tinjauanpustaka}

\section{Penelitian Terdahulu}
\label{sec:penelitianterdahulu}

Finanta Okmuyura, Noverta Effendi, Witri Ramadhani, dan Adlian Jefiza melakukan penelitian ini dengan membuat analisis dan desain untuk dapat memonitor pembakaran kalori saat jogging. Pada penelitian ini, dalam memonitor pembakaran kalori menggunakan sensor akselerometer yang dapat menghitung berdasarkan dari tekanan dari beban yang diterima untuk menghasilkan nilai \emph{threshold} untuk dikalkukasikan nantinya. Perhitungan kalori yang terbakar pada penelitian ini dengan menggunakan nilai jumlah langkah kaki, waktu dan berat pengguna untuk memberikan informasi pembakaran kalori dalam jogging (Okmuyura et al., 2019).

Dina Budhi Utami dan Muhammad Ichwan melakukan penelitian mengenai sistem prediksi kalori yang terbakar pada pesepeda menggunakan \emph{Feedforward Neural Network}. Penelitian ini melakukan prediksi berdasarkan detak jantung dan kecepatan kayuh saat bersepeda. Model prediksi kalori yang digunakan adalah \emph{Feedforward Neural Network} dengan arsitektur jaringan saraf tiruan terdiri dari 3 lapis. Hasil keluaran dari jaringan saraf tiruan adalah nilai prediksi kalori menggunakan pengujian 10000 data latih dengan memiliki tingkat kesalahan adalah 7\% (Utami \& Ichwan, 2017).

Pada tahun 2019, Philip Saponaro bersama Haoran Wei, Gregory Dominick dan Chandra Kambhamettu melakukan penelitian ini. Penelitian yang dilakukan mengenai perkirakan intensitas aktivitas fisik dan pengeluaran energi dengan menggunakan sistem visi komputer. Nilai perkiraan aktivitas fisik dan pengeluaran energi menggunakan faktor usia, jenis kelamin, kecepatan dan isyarat aktivitas. Data nilai usia dan jenis kelamin didapatkan dengan jaringan \emph{Deep Expectation} dan nilai aktivitas diperoleh dari perkiraan sudut sendi dan kecepatan gerak. Hasil yang didapat dengan akurasi nilai perkiraan aktivitas fisik sebesar 89,5\% dan perbedaan rata-rata pengeluaran energi sebesar 1,96 kCal/min (Saponaro et al.,  2019).


\section{Kalori}
\label{sec:kalori}

Kalori adalah unit pengukuran yang mengukur kandungan energi makanan. Saat kita mengonsumsi makanan dan minuman, kita menyuplai tubuh dengan kalori atau energi. Kalori ini digunakan oleh tubuh untuk menggerakkan berbagai aktivitasnya. Semakin besar tingkat aktivitas fisik, semakin tinggi jumlah kalori atau energi yang dibutuhkan. Item makanan biasanya diberi label dengan informasi kalori yang dinyatakan dalam kilokalori (kkal). Kalori memainkan peran penting dalam menopang tubuh kita dan memfasilitasi berbagai aktivitas (Inmawati N. D., 2016).

Perhitungan kebutuhan kalori memperhitungkan berbagai faktor seperti jenis kelamin, usia, tinggi badan, berat badan, komposisi tubuh, tingkat aktivitas, dan kondisi fisik. Pria dan wanita memiliki kebutuhan kalori yang berbeda, bahkan dalam rentang usia yang sama. Jika seseorang melakukan aktivitas fisik yang sangat berat, asupan kalori hariannya perlu ditingkatkan. Memahami kebutuhan energi harian seseorang sangat penting untuk menjaga kesehatan secara keseluruhan karena mempengaruhi keseimbangan energi sepanjang hari. Pasar menyaksikan peningkatan ketersediaan pilihan makanan dengan kandungan kalori rendah, yang sering disebut sebagai "rendah lemak". Namun, banyak orang berjuang untuk mempertimbangkan pertimbangan kalori ini secara memadai dalam pilihan diet mereka.

Menghitung kalori secara akurat memang menantang karena beberapa faktor, antara lain berat badan, intensitas aktivitas, kondisi tubuh, dan metabolisme. Aktivitas fisik menyebabkan pengeluaran energi dalam tubuh. Jika asupan kalori melebihi energi yang dibakar melalui aktivitas fisik yang seimbang, maka dapat mengakibatkan kenaikan berat badan. Seiring bertambahnya usia, mereka cenderung menjadi kurang aktif secara fisik, mengalami penurunan massa otot, dan mengalami penurunan laju pembakaran kalori. Hal ini dapat mempersulit tubuh untuk membakar kalori yang dikonsumsi, yang menyebabkan akumulasi energi dan berpotensi menyebabkan obesitas. Seiring bertambahnya usia dan asupan kalori yang konsisten, kemampuan tubuh untuk membakar kalori yang masuk semakin berkurang, sehingga memengaruhi pengelolaan berat badan (Widiantini et al., 2013).

\section{\emph{Metabolic Equivalent of Task}}
\label{sec:met}

Catatan \emph{physical activity} memberikan perkiraan \emph{energy expenditure} (EE) berdasarkan laporan terperinci dari semua hasil harian terhadap \emph{physical activity} (PA) yang dilakukan. Namun, catatan-catatan ini sering dianggap sebagai metode pelengkap karena subjektivitasnya. Data PA dikategorikan dan dikodekan berdasarkan jenis dan intensitas aktivitas, memungkinkan deskripsi pola aktivitas fisik dalam suatu populasi dan eksplorasi faktor yang mempengaruhi pola tersebut. Selain itu, catatan ini memungkinkan penyelidikan hubungan antara aktivitas fisik, kesehatan, dan penyakit. Catatan tersebut juga dapat digunakan untuk menilai kontribusi berbagai jenis aktivitas fisik terhadap pengeluaran energi total (TEE), yang menawarkan wawasan tambahan tentang jenis aktivitas yang biasanya dilakukan. Salah satu sistem pengkodean yang tersedia adalah The Compendium of Physical Activity, yang diterbitkan pada tahun 1993 (Ainsworth et al., 1993). Kompendium ini menggunakan kode lima digit untuk mewakili aktivitas tertentu yang dilakukan dalam berbagai situasi. Setiap aktivitas dikaitkan dengan tingkat intensitas yang sesuai, yang dinyatakan dalam unit \emph{Metabolic Equivalent of Task} (MET) (Pinheiro et al., 2011).

\emph{Metabolic Equivalent of Task} (MET) adalah pengukuran yang menghitung jumlah oksigen yang dibutuhkan selama istirahat per kilogram berat badan dan waktu. Konsumsi oksigen pada \emph{Basal Metabolic Rate} (BMR) umumnya diperkirakan sekitar 3,5 mL O\textsubscript{2}/kg/menit, tetapi faktor seperti usia, jenis kelamin, dan penyakit dapat mempengaruhi nilai ini. Istilah "konsumsi oksigen relatif" (VO\textsubscript{2} relatif) digunakan untuk menggambarkan nilai ini karena diberikan per kilogram massa tubuh. Mengalikannya dengan massa tubuh memberikan "konsumsi oksigen absolut" (VO\textsubscript{2} absolut). Misalnya, seseorang dengan berat 80 kg akan memiliki VO\textsubscript{2} absolut 280 mL/menit saat istirahat. Namun, nilai relatif lebih berharga karena memungkinkan perbandingan yang mudah antara individu dengan massa tubuh yang berbeda, dengan menghilangkan pengaruh massa tubuh. Konsep MET diperkenalkan untuk menghitung tingkat konsumsi oksigen yang berbeda dan menyederhanakan perhitungan dengan menggunakan faktor 3,5. Sebagai contoh peningkatan konsumsi oksigen (pengeluaran energi) sepuluh kali lipat dapat dinyatakan sebagai 35 mL O\textsubscript{2}/kg/menit atau hanya sebagai 10 MET (Steinach et al., 2015).

Pernyataan yang didefinisikan oleh Jetté pada tahun 1990 menyatakan sebagai informasi bahwa 1 MET setara dengan konsumsi oksigen sebesar 3,5 mL O\textsubscript{2}/kg/menit saat istirahat (Jetté et al., 1990). Pengeluaran energi memiliki korelasi kesetaraan satuan dengan konsumsi oksigen yang dilakukan oleh Scott bahwa konsumsi oksigen dengan nilai 1.000 mL O\textsubscript{2} setara dengan pengeluaran energi sebesar 5 kilokalori (kkal) (Scott, 2005). Dengan begitu dapat ditentukan dengan melakukan konversi konsumsi oksigen dalam mL O\textsubscript{2} menjadi pengeluaran energi dalam kkal dengan menggunakan persamaan satuan yang sudah disebutkan sebelumnya. Didapatkan persamaan dalam mencari kalori terbakar sebagai pengeluaran energi dalam kkal/menit dengan menggunakan Persamaan \ref{eq:RumusKalori1} berikut.

\begin{equation}
  \label{eq:RumusKalori1}
  Kalori \; Terbakar = \frac{MET \times 3.5 \times berat \; badan \; (kg)}{200} \; kkal/menit
\end{equation}

Konsumsi oksigen, dan nilai METs, bervariasi menurut usia. Misalnya, pada remaja usia 16-17 tahun, 1 MET setara dengan konsumsi oksigen 4,0 mL O\textsubscript{2}/kg/menit. Pada individu berusia 12-13 tahun, 1 MET setara dengan 4,58 mL O\textsubscript{2}/kg/menit, sedangkan pada anak di bawah usia 5 tahun, 1 MET setara dengan konsumsi oksigen 7,0 mL O\textsubscript{2}/kg/menit (Pinheiro et al., 2011).

\subsection{\emph{Compendium of Physical Activities}}
\label{subsec:compendium}

Dalam pemahaman yang komprehensif tentang skema pengkodean, organisasi, dan metode yang digunakan untuk menghitung biaya energi \emph{physical activity} (PA), telah dilakukan penelitian pada kompendium versi 1993 yang diterbitkan (Ainsworth et al., 1993). Kompendium disusun sedemikian rupa sehingga memaksimalkan fleksibilitas dalam pengkodean, entri data, dan interpretasi biaya energi untuk kelas dan jenis PA yang berbeda. Sistem pengkodean menggunakan kode lima digit yang mengkategorikan aktivitas berdasarkan tujuan atau kategori utamanya (dua digit pertama), aktivitas spesifik (tiga digit terakhir), dan intensitas (kolom dua atau tiga digit terpisah). Edisi revisi Kompendium Aktivitas Fisik memperkenalkan dua judul utama tambahan, menghasilkan total 21 kategori utama aktivitas fisik (Haskell et al., 2007). Berikut Tabel \ref{tb:metjenisaktivitas} yang memaparkan semua jenis aktivitas utama yang terdapat pada kompendium.

\begin{longtable}{|c|c|c|c|}
  \caption{Jenis aktivitas utama pada \emph{compendium of physical activities}}
  \label{tb:metjenisaktivitas}  \\
  \hline
  \rowcolor[HTML]{C0C0C0}
  \textbf{Digit Kode} & \textbf{Jenis Aktivitas} & \textbf{Digit Kode} & \textbf{Jenis Aktivitas} \\
  \hline
  01-     & Bicycling                 & 12-    & Running     \\
  \hline
  02-     & Conditioning Exercises    & 13-    & Self Care     \\
  \hline
  03-     & Dancing                   & 14-    & Sexual Activity     \\
  \hline
  04-     & Fishing and Hunting       & 15-    & Sports    \\
  \hline
  05-     & Home Activities           & 16-    & Transportation     \\
  \hline
  06-     & Home Repair               & 17-    & Walking    \\
  \hline
  07-     & Inactivity                & 18-    & Water Activities    \\
  \hline
  08-     & Lawn and Garden           & 19-    & Winter Activities    \\
  \hline
  09-     & Miscellaneous             & 20-    & Religious Activities    \\
  \hline
  10-     & Music Playing             & 21-    & Volunteer Activities    \\
  \hline
  11-     & Occupation                &        &     \\
  \hline
\end{longtable}

Versi terbaru Kompendium mencakup total 605 aktivitas spesifik, dengan 129 aktivitas baru yang ditambahkan dibandingkan dengan edisi tahun 1993. Perubahan juga dilakukan pada 94 aktivitas yang tercantum dalam Kompendium 1993, yang melibatkan penambahan atau penghapusan aktivitas spesifik yang terkait dengan setiap kode. Dalam beberapa kasus, aktivitas dihapus dari kode yang ada dan kode baru dibuat jika aktivitas yang dihapus memiliki tingkat MET yang berbeda atau secara kualitatif berbeda dari aktivitas lain. Semua aktivitas dalam Kompendium diberi tingkat intensitas yang dinyatakan sebagai MET, yang mewakili tingkat pengeluaran energi. Intensitas aktivitas diklasifikasikan sebagai kelipatan dari 1 MET atau sebagai rasio tingkat metabolisme terkait untuk aktivitas spesifik dibagi dengan Tingkat Metabolisme Istirahat (RMR) standar (Ainsworth et al., 2000). Pada Kompendium 1993, nilai MET ditetapkan untuk setiap aktivitas berdasarkan "representasi terbaik" dari tingkat intensitas yang bersumber dari daftar yang dipublikasikan dan data terpilih yang tidak dipublikasikan (Ainsworth et al., 1993).  Berdasarkan model klasifikasi yang diusulkan oleh Pate et al. untuk mengkategorikan intensitas aktivitas (ringan: kurang dari 3 METs, sedang: 3-6 METs, tinggi: lebih dari 6 METs) (Pate et al., 1995). Tabel \ref{tb:metintensitas} menunjukkan beberapa data aktivitas pada kompendium dengan dikategorikan berdasarkan intensitas beserta nilai MET dari aktivitas yang ditunjukkan.

\begin{longtable}{|c|c|c|c|}
  \caption{Nilai MET dari aktivitas fisik berdasarkan tingkat intensitas}
  \label{tb:metintensitas}  \\
  \hline
  \rowcolor[HTML]{C0C0C0}
  \textbf{Intensitas} & \textbf{Aktivitas} & \textbf{MET} \\
  \hline
  \multirow{4}{*}{Ringan}     & Berjalan perlahan di sekitar rumah      & 2.0     \\
  \cline{2-3} &
                                Duduk menggunakan kerja komputer        & 1.5     \\
  \cline{2-3} &
                                Berdiri melakukan pekerjaan ringan      & 2.0     \\
  \cline{2-3} &
                                Seni kerajinan, bermain kartu           & 1.5    \\
  \hline
  \multirow{4}{*}{Sedamg}     & Berjalan 3.0 mph                        & 3.3     \\
  \cline{2-3} &
                                Pertukangan umum                        & 3.6    \\
  \cline{2-3} &
                                Bulu tangkis dengan rekreasi            & 4.5    \\
  \cline{2-3} &
                                Tenis ganda                             & 5.0    \\
  \hline
  \multirow{4}{*}{tinggi}     & Berjalan dengan kecepatan 4.5 mph       & 6.3    \\
  \cline{2-3} &
                                Joging dengan kecepatan 5 mph           & 8.0    \\
  \cline{2-3} &
                                Menyekop, menggali parit                & 8.5    \\
  \cline{2-3} &
                                Berlari dengan kecepatan 7 mph          & 11.5    \\
  \hline
\end{longtable}

Kompendium berfungsi sebagai alat yang berguna untuk mencatat secara sistematis jenis, intensitas, dan durasi aktivitas fisik (PA) dalam catatan PA. Catatan-catatan ini, bersama dengan Kompendium, telah digunakan untuk memvalidasi survei PA yang umum digunakan dalam studi observasional dan klinis (Ainsworth et al., 2000). Pada tahun 1993, catatan sampel PA diperkenalkan untuk digunakan dengan kompendium (Ainsworth et al., 1993). Sejak itu, catatan PA yang diperbarui dan mudah digunakan telah dikembangkan, yang menyederhanakan pengkodean dan memberikan panduan komprehensif tentang penggunaan catatan PA dan kompendium dalam studi validasi PA (Ainsworth et al., 2000).


\section{Regresi}
\label{sec:regresi}

Analisis regresi melayani dua tujuan utama. Pertama, ini biasanya digunakan untuk peramalan dan prediksi, yang menyelaraskannya dengan bidang \emph{machine learning}. Kedua, analisis regresi dapat digunakan untuk mengidentifikasi hubungan sebab akibat antara variabel independen dan dependen. Penting dalam mengatahui bahwa analisis regresi saja mengungkapkan hubungan antara variabel dependen dan seperangkat variabel independen tertentu, tanpa menyiratkan sebab-akibat. Model regresi menyediakan sarana untuk memprediksi variabel dependen berdasarkan variabel independen. Dengan memeriksa kisaran nilai variabel independen 'x', analisis regresi memperkirakan nilai variabel dependen 'y'. Dalam penelitian ini melakukan eksplorasi regresi linier dan regresi polinomial sebagai model yang menawarkan peningkatan kesesuaian untuk tujuan prediktif. Analisis regresi dapat mencakup baik regresi linier sederhana maupun regresi linier berganda (Maulud et al., 2020).

\subsection{Regresi Linear Sederhana}
\label{subsec:regresilinearsederhana}

Regresi linier adalah teknik pemodelan prediktif yang memanfaatkan garis lurus untuk mewakili hubungan antara dua variabel atau lebih (Kurniawan, 2008). Ini adalah metode statistik yang biasa digunakan untuk menganalisis hubungan antara variabel dependen (juga dikenal sebagai variabel respons, dilambangkan sebagai Y) dan satu atau lebih variabel independen (juga disebut sebagai variabel prediktor, dilambangkan sebagai X). Ketika hanya ada satu variabel independen, itu disebut regresi linier sederhana, sedangkan kehadiran beberapa variabel independen disebut regresi linier berganda. Analisis regresi banyak digunakan dalam penelitian dan pertama kali diperkenalkan oleh Sir Francis Galton pada tahun 1886. Secara umum, analisis regresi adalah studi statistik yang mengeksplorasi hubungan antara variabel dependen (disebut sebagai variabel yang dijelaskan) dan satu atau lebih variabel independen (disebut sebagai variabel yang menjelaskannya) (Syilfi \& Ispriyanti, 2012). Dapat diketahui jika analisis regresi mengkaji hubungan antara satu variabel (disebut sebagai variabel dependen) dan satu atau lebih variabel yang menjelaskannya (dikenal sebagai variabel independen).

\begin{figure}[H]
  \centering
  \includegraphics[scale=0.43]{gambar/plotregresilinear.png}
  \caption{Contoh model regresi linear sederhana (Rivera, 2020).}
  \label{fig:plotregresilinear}
\end{figure}

Regresi linier memiliki tiga tujuan utama: menggambarkan fenomena data, kontrol, dan prediksi. Analisis regresi memungkinkan deskripsi fenomena data dengan membuat model hubungan numerik. Model ini membantu untuk memahami pola dan hubungan yang mendasari data yang sedang dipelajari. Kemudian regresi dapat digunakan untuk tujuan kontrol. Dengan menggunakan model regresi yang diperoleh, dimungkinkan untuk mengontrol atau memanipulasi variabel independen untuk melihat pengaruhnya terhadap variabel dependen. Hal ini memungkinkan untuk mempelajari hubungan sebab-akibat dan menentukan pengaruh variabel tertentu pada hasil yang diinginkan. Analisis regresi juga memfasilitasi prediksi. Model regresi dapat digunakan untuk membuat prediksi terhadap variabel dependen. Namun, penting untuk diketahui bahwa prediksi terbatas pada kisaran variabel independen yang digunakan dalam membangun model regresi (Sulardi et al., 2017).

Regresi linear sederhana melakukan pengujian hubungan linear antara variabel dependen Y dan variabel independen tunggal X. Model merepresentasikan hubungan ini menggunakan persamaan garis lurus yang ditunjukkan pada Persamaan \ref{eq:RumusRegresiLinearSederhana}.

\begin{equation}
  \label{eq:RumusRegresiLinearSederhana}
  Y = a + bx
\end{equation}

Pada persamaan yang digunakan dimana a adalah perpotongan Y, dan b adalah kemiringan garis. Metode statistik digunakan untuk mengestimasi nilai a dan b dari data Y dan X. Garis regresi memungkinkan kita untuk membuat prediksi variabel dependen Y berdasarkan variabel independen X. Kemiringan b, juga dikenal sebagai koefisien regresi, menunjukkan sejauh mana variabel independen X memengaruhi variabel dependen Y. Dalam kasus variabel independen kontinu, koefisien regresi mewakili perubahan variabel dependen untuk setiap unit perubahan variabel independen (Schneider et al., 2010).

\subsection{Regresi Linear Berganda}
\label{subsec:regresilinearberganda}

Regresi multivariabel atau berganda digunakan untuk menguji hubungan antara variabel dependen (hasil yang diinginkan) dan beberapa variabel independen. Model-model ini memiliki berbagai aplikasi, seperti mengidentifikasi karakteristik pasien yang terkait dengan hasil tertentu, menilai dampak teknik prosedural pada hasil, membandingkan strategi pengobatan yang berbeda sambil menyesuaikan perbedaan kelompok, mengukur ukuran efek, mengembangkan skor kecenderungan, dan membangun model prediksi risiko (Grant et al., 2018). Regresi linier berganda merupakan perluasan dari regresi linier sederhana yang melibatkan lebih dari satu variabel penjelas. Istilah "linear" digunakan karena diasumsi bahwa variabel respon berhubungan langsung dengan kombinasi linear dari variabel penjelas. Sementara analisis tidak membangun sebab-akibat, itu memungkinkan untuk mengeksplorasi hubungan antara satu set variabel penjelas dan variabel respon tertentu yang menarik (Tranmer et al., 2020).

Analisis regresi adalah alat statistik berharga yang digunakan dalam rekayasa untuk menentukan korelasi antara beberapa variabel. Ini melibatkan pemasangan model ke data, dan dalam kasus regresi linier, fungsi prediktor linier digunakan untuk memodelkan data, dengan parameter keluaran diperkirakan dari data. Ketika ada banyak variabel input yang terlibat, itu menjadi "regresi linier berganda", yang menilai korelasi antara variabel-variabel ini dengan menyesuaikan persamaan linier dengan data yang diamati. Tujuannya adalah untuk menemukan koefisien regresi parsial yang meminimalkan perbedaan antara keluaran model dan data aktual pada perangkat pelatihan. Optimalisasi ini bertujuan untuk meminimalkan jumlah kuadrat simpangan vertikal dari setiap titik data ke dalam persamaan regresi. Jika titik data terletak pada garis yang dipasang, deviasi vertikal adalah nol (Khademi et al., 2017). Persamaan untuk model regresi linier berganda ditunjukkan pada Persamaan \ref{eq:RumusRegresiLinearBerganda}.

\begin{equation}
  \label{eq:RumusRegresiLinearBerganda}
  Y = \beta_0 + \beta_1 X_1 + \beta_2 X_2 + \beta_3 X_3 + ... + \beta_n X_n + e
\end{equation}

Pada persamaan yang digunakan Y adalah keluaran model, X adalah variabel masukan independen untuk model, dan B adalah koefisien regresi yang akan menjadi nilai prediksi pada Y. Koefisien ini ditentukan melalui pelatihan model untuk mencapai kecocokan terbaik antara keluaran model dan keluaran aktual yang diamati dalam kumpulan data pelatihan (Grant et al., 2018).

\subsection{Regresi Polinomial}
\label{subsec:regresipolinomial}

Regresi polinomial adalah teknik statistik yang memungkinkan pemodelan hubungan antara beberapa variabel independen (X dan Y) dan variabel dependen (Z) melalui hubungan non-linear (Shanock et al., 2010). Prosesnya melibatkan analisis hierarki persamaan polinomial, dengan memasukkan suku-suku orde tinggi sampai varians yang dijelaskan oleh persamaan orde tinggi berikutnya tidak lagi signifikan secara statistik. Pada tahap pertama, skor komponen untuk X dan Y (direpresentasikan sebagai X1 dan Y1) digunakan untuk menguji hubungan liniernya dengan Z pada tahap pertama analisis. Pada tahap kedua, suku orde tinggi (X2 dan Y2) dimasukkan ke dalam persamaan bersama dengan suku hasil kali (XY) untuk menilai keberadaan hubungan lengkung (khususnya, kuadrat). Selain itu, regresi polinomial, dikombinasikan dengan metodologi permukaan respons, menawarkan kerangka kerja untuk menguji dan menginterpretasikan fitur permukaan yang sesuai dengan persamaan regresi kuadrat polinomial. Teknik-teknik ini umumnya digunakan dalam penelitian organisasi, baik di tingkat mikro maupun makro, untuk menguji kesesuaian dan/atau perbedaan antar variabel (Shanock et al., 2010).

Metodologi ini memungkinkan peneliti untuk mengeksplorasi bagaimana kombinasi dua variabel prediktor dikaitkan dengan variabel hasil. Mereka telah menemukan aplikasi yang luas dalam penelitian umpan balik multi-sumber (Shanock et al., 2010). Kombinasi dari teknik-teknik ini menawarkan alat statistik yang memungkinkan pemeriksaan mendalam tentang hubungan tripartit. Dengan menyelidiki variabel dalam ruang tiga dimensi, ini memberikan wawasan tentang hubungan antara kombinasi dua variabel prediktor dan variabel hasil (Shanock et al., 2010).

Regresi polinomial dapat digunakan sebagai alternatif regresi moderat untuk memeriksa hubungan antara kombinasi variabel, menawarkan kekuatan penjelas yang lebih besar dan pandangan bernuansa dalam ruang tiga dimensi. Kemudian, teknik ini cocok ketika asumsi teoretis yang mendasari menunjukkan hubungan non-linear antara variabel independen dan dependen. Setelah asumsi terpenuhi, regresi polinomial dilakukan untuk menyelesaikan persamaan polinomial dan mendapatkan output. Keluaran ini kemudian dapat diproyeksikan ke permukaan respons tiga dimensi. Panduan langkah demi langkah terperinci untuk melakukan regresi polinomial dan membuat permukaan respons menggunakan keluaran polinomial disediakan oleh Shanock et al. (Shanock et al., 2010). Bentuk umum model regresi polinomial dapat ditunjukkan pada Persamaan \ref{eq:RumusRegresiPoliSederhana}.

\begin{equation}
  \label{eq:RumusRegresiPoliSederhana}
  Y = \beta_0 + \beta_1 X + \beta_2 X^2 + \beta_3 X^3 + ... + \beta_n X^n + e
\end{equation}

Dalam persamaan ini, Y mewakili variabel respons yang ingin kita prediksi, dan b adalah koefisien regresi yang mengukur dampak dari setiap variabel prediktor pada urutan besarnya yang berbeda. Variabel e mewakili istilah residu atau kesalahan, menangkap varian yang tidak dapat dijelaskan yang tidak diperhitungkan oleh model regresi polinomial. Setiap variabel prediktor (X) dinaikkan ke urutan tertentu (n), dan efeknya digabungkan untuk membentuk model yang lebih fleksibel dan mudah beradaptasi (Malensang et al., 2013). Gambar \ref{fig:plotregresipoli} menunjukkan model regresi polinomial dengan tampilan data dan model regresi polinomial pada beberapa nilai orde yang digunakan.

\begin{figure}[H]
  \centering
  \includegraphics[scale=0.35]{gambar/plotpolinomialsederhana.png}
  \caption{Contoh model regresi polinomial (Ostertagova et al., 2016).}
  \label{fig:plotregresipoli}
\end{figure}



\section{\emph{Machine Learning}}
\label{sec:machinelearning}

\emph{Machine Learning} (ML) adalah cabang ilmu komputasi yang muncul dari studi klasifikasi data dan prinsip-prinsip \emph{Artificial Intelligence} (AI). Bidang ini melibatkan pelatihan komputer untuk belajar secara otomatis dari input data, tanpa perlu pemrograman eksplisit. Konsep pembelajaran dalam pembelajaran mesin menarik kesejajaran dari pembelajaran manusia dan hewan. Faktanya, banyak teknik pembelajaran mesin terinspirasi oleh model komputasi berdasarkan prinsip pembelajaran hewan dan manusia. Misalnya, pembiasaan adalah perilaku kognitif mendasar yang diamati pada hewan, di mana mereka secara bertahap berhenti merespons rangsangan berulang. Anjing, misalnya, berfungsi sebagai contoh utama pembelajaran hewan, karena mereka dapat menjalani pelatihan substansial untuk melakukan berbagai aktivitas seperti berguling, duduk, dan mengambil objek (Datta \& Davim, 2022).

\begin{figure}[H]
  \centering
  \includegraphics[scale=0.3]{gambar/machinelearning.png}
  \caption{Diagram Venn konsep dan kelas \emph{machine learning} (Janiesch et al., 2021).}
  \label{fig:machinelearningdiagram}
\end{figure}

Pembelajaran mesin menemukan aplikasi praktis dalam berbagai aspek kehidupan kita sehari-hari di era modern. Misalnya, asisten pribadi virtual, seperti Siri atau Alexa, menggunakan teknik pembelajaran mesin untuk memahami dan merespons perintah pengguna. Sistem navigasi GPS memanfaatkan pembelajaran mesin untuk memprediksi kondisi lalu lintas dan menyarankan rute yang optimal. Sistem pengawasan bertenaga AI dapat menganalisis umpan dari beberapa kamera untuk mendeteksi potensi kejahatan atau perilaku yang tidak biasa. Platform media sosial memanfaatkan pembelajaran mesin untuk tugas-tugas seperti pengenalan wajah dan kurasi umpan berita yang dipersonalisasi. Mesin pencari terus menyempurnakan hasil mereka menggunakan algoritma pembelajaran mesin. Filter spam email belajar dari email spam berlabel sebelumnya untuk mengidentifikasi dan memblokir pesan yang tidak diinginkan. Ini hanyalah beberapa contoh yang menunjukkan penggunaan luas pembelajaran mesin dalam aplikasi praktis. Memasukkan pengetahuan sebelumnya ke dalam proses pembelajaran dapat secara signifikan meningkatkan keefektifannya. Pembelajaran mesin juga terkait erat dengan statistik komputasi, memungkinkan pemodelan prediktif. Pentingnya pembelajaran mesin terletak pada pemahaman bagaimana hewan dan manusia belajar, serta dalam berbagai tujuan praktisnya (Datta \& Davim, 2022).

\subsection{\emph{Supervised Learning}}
\label{subsec:supervisedlearning}

Pembelajaran yang diawasi melibatkan penggunaan kumpulan data pelatihan yang berisi contoh dengan data input dan nilai target berlabel. Misalnya, memprediksi jumlah pengguna aktif pada platform pasar di bulan mendatang dapat dianggap sebagai tugas pembelajaran yang diawasi. Dalam hal ini, fitur input, seperti jumlah produk yang terjual atau review pengguna yang positif, digunakan untuk memprediksi variabel target atau output (sering dinotasikan sebagai variabel "y"). Dataset pelatihan digunakan untuk menyesuaikan parameter model pembelajaran mesin. Setelah model berhasil dilatih, model tersebut dapat diterapkan untuk memprediksi variabel target untuk titik data baru atau yang tidak terlihat berdasarkan fitur inputnya. Pembelajaran yang diawasi dapat dikategorikan lebih lanjut ke dalam masalah regresi, di mana nilai numerik diprediksi (mis., Jumlah pengguna), dan masalah klasifikasi, di mana hasil prediksi mewakili label kelas kategori, seperti "penonton" atau "pembeli" (Janiesch et al., 2021).

\subsection{\emph{Unsupervised Learning}}
\label{subsec:unsupervisedlearning}

Pembelajaran tanpa pengawasan terjadi ketika sistem pembelajaran ditugaskan untuk mengidentifikasi pola dalam data tanpa label atau spesifikasi yang telah ditentukan sebelumnya. Dalam pembelajaran tanpa pengawasan, data pelatihan hanya terdiri dari variabel (dilambangkan sebagai "x"), yang bertujuan untuk menemukan informasi struktural yang bermakna. Ini dapat melibatkan pendeteksian kelompok elemen yang memiliki sifat umum, yang dikenal sebagai pengelompokan, atau pengurangan dimensi data dengan memproyeksikannya dari ruang dimensi yang lebih tinggi ke ruang dimensi yang lebih rendah, yang dikenal sebagai pengurangan dimensi. Dalam konteks pasar elektronik, contoh menonjol dari pembelajaran tanpa pengawasan adalah penerapan teknik pengelompokan untuk mengelompokkan pelanggan atau pasar ke dalam segmen-segmen. Ini memungkinkan komunikasi yang lebih bertarget dan spesifik yang disesuaikan dengan kelompok pelanggan yang berbeda (Janiesch et al., 2021).

\subsection{\emph{Semi Supervised Learning}}
\label{subsec:semiisupervisedlearning}

Pembelajaran semi-diawasi adalah pendekatan hybrid yang menggabungkan metode pembelajaran yang diawasi dan tidak diawasi, dengan fokus pada penggunaan sampel berlabel dan tidak berlabel secara efektif selama proses pelatihan. Ini bertujuan untuk membangun hubungan antara sampel yang diprediksi dan tujuan pembelajaran berdasarkan hipotesis tertentu. Asumsi kehalusan menunjukkan bahwa sampel yang terletak berdekatan di wilayah kepadatan tinggi lebih cenderung memiliki label kelas yang sama. Asumsi cluster berpendapat bahwa sampel yang termasuk dalam cluster yang sama sangat mungkin untuk berbagi kelas yang sama. Terakhir, asumsi manifold mengusulkan bahwa sampel dalam lingkungan lokal kecil dalam manifold berdimensi rendah cenderung memiliki label kelas yang serupa (Wang et al., 2023).

\subsection{\emph{Reinforcement Learning}}
\label{subsec:reinforcementlearning}

Dalam sistem pembelajaran penguatan, pendekatannya berbeda dengan pemberian pasangan input-output. Sebaliknya, sistem dijelaskan oleh keadaan saat ini, tujuan yang ditentukan, serangkaian tindakan yang diizinkan, dan kendala lingkungan terkait untuk setiap hasil tindakan. Model pembelajaran mesin kemudian belajar dengan secara aktif terlibat dalam proses pencapaian tujuan melalui coba-coba, yang bertujuan untuk memaksimalkan hadiah. Pembelajaran penguatan telah menunjukkan pencapaian yang signifikan dalam lingkungan yang terkendali seperti permainan dan juga berlaku untuk sistem multi-agen seperti pasar elektronik (Janiesch et al., 2021).

\section{\emph{Deep Learning}}
\label{sec:deeplearning}

\emph{Deep learning} adalah cabang \emph{machine learning} yang melibatkan penggunaan beberapa lapisan pemrosesan informasi nonlinier untuk melakukan tugas-tugas seperti ekstraksi fitur, pengenalan pola, dan klasifikasi (Deng dan Yu, 2014). Menurut Goodfellow, dkk. (2016) dalam \emph{deep learning}, konsep kompleks dipelajari dengan menggabungkan konsep yang lebih sederhana secara hierarkis. Struktur hierarkis ini memungkinkan komputer mempelajari dan memahami pola dan hubungan yang rumit di dalam data. Istilah "dalam" mengacu pada beberapa lapisan dalam jaringan, membentuk grafik yang dalam saat divisualisasikan, oleh karena itu dinamakan "pembelajaran mendalam". \emph{Deep learning} telah terbukti sangat efektif di berbagai bidang, termasuk visi komputer, pemrosesan bahasa alami, dan pengenalan suara.

\emph{Deep learning} dicirikan oleh arsitekturnya yang terdiri dari beberapa lapisan, umumnya dikenal sebagai \emph{hidden layer}, yang ditumpuk bersama. Setiap lapisan dalam arsitektur ini berfungsi sebagai algoritma atau metode yang mengambil input dan menghasilkan output melalui serangkaian komputasi. Salah satu metode \emph{deep learning} yang menonjol adalah \emph{Convolutional Neural Network} (CNN). CNN dirancang khusus untuk memproses input gambar. Gambar input melewati lapisan konvolusional, di mana diproses menggunakan filter yang mengekstraksi pola dari berbagai bagian gambar. Ekstraksi pola hierarkis dari gambar ini membantu dalam proses klasifikasi. CNN telah menunjukkan keberhasilan luar biasa dalam berbagai tugas terkait gambar seperti pengenalan objek, klasifikasi gambar, dan segmentasi gambar (Danukusumo, 2017).

\begin{figure}[H]
  \centering
  \includegraphics[scale=1]{gambar/arsitekturdlnn.png}
  \caption{Arsitektur \emph{deep learning} (Janiesch et al., 2021)}
  \label{fig:arsidl}
\end{figure}


Di bidang pengenalan citra, \emph{Convolutional Neural Network} (CNN) saat ini merupakan metode \emph{deep learning} yang paling berdampak (Nugroho et al., 2020). CNN sangat berhasil dalam lingkup ini karena mencoba meniru sistem pengenalan gambar dari korteks visual manusia, memungkinkannya memproses informasi gambar secara efektif. Namun, seperti metode pembelajaran mendalam lainnya, CNN memiliki kekurangan, yaitu proses pelatihan yang lama. Untungnya, kemajuan teknologi perangkat keras, seperti penggunaan \emph{General Purpose Graphical Processing Unit} (GPGPU), telah membantu mengatasi tantangan ini.

CNN dirancang dengan fokus khusus pada tugas yang terkait dengan pengenalan dan klasifikasi gambar. Arsitektur dan strukturnya disesuaikan untuk memproses dan menganalisis data gambar secara efisien. Ini terdiri dari beberapa lapisan yang mengekstrak informasi yang relevan dari gambar dan membuat prediksi tentang klasifikasinya melalui skor klasifikasi. Lapisan di CNN, seperti lapisan konvolusional dan lapisan penyatuan, memainkan peran penting dalam menangkap fitur dan pola hierarki dalam gambar, memungkinkan pengenalan dan klasifikasi yang akurat. Secara keseluruhan, kemampuan CNN untuk mempelajari dan menginterpretasikan data visual yang kompleks menjadikannya metode terdepan dalam bidang pengenalan gambar.

\subsection{\emph{Neural Network}}
\label{subsec:cnn}

Artificial Neural Networks (ANNs) adalah sistem komputasi yang mengambil inspirasi dari Biological Neural Networks (BNNs). Mereka menawarkan solusi yang efektif untuk berbagai tugas, seperti klasifikasi, prediksi, pemfilteran, pengoptimalan, pengenalan pola, dan perkiraan fungsi. Sementara sistem saraf biologis sangat rumit, JST bertujuan untuk menyederhanakan dan abstrak kompleksitas ini, berfokus pada aspek-aspek penting yang relevan dengan pemrosesan informasi. Jaringan saraf awalnya diperkenalkan oleh McCulloch dan Pitts pada tahun 1990 ketika mereka berusaha memodelkan pemrosesan informasi secara matematis dalam sistem biologis. Jaringan ini terdiri dari simpul yang saling berhubungan, menyerupai neuron yang ditemukan di otak organisme hidup. Setiap node menghitung jumlah bobot inputnya, memprosesnya di dalam lapisan tersembunyi, dan menghasilkan output dengan menerapkan fungsi aktivasi ke nilai bobot (Thakur et al., 2021).

\begin{figure}[H]
  \centering
  \includegraphics[scale=1]{gambar/arsitekturann.png}
  \caption{Arsitektur \emph{neural network} (Janiesch et al., 2021)}
  \label{fig:arsiann}
\end{figure}

Neural network telah berevolusi dari arsitektur sederhana menjadi struktur yang semakin kompleks. Awalnya, jaringan saraf memiliki arsitektur yang sangat dasar yang hanya terdiri dari lapisan masukan dan keluaran, sering disebut sebagai jaringan lapisan tunggal. Namun, dengan memasukkan lapisan tersembunyi ke dalam jaringan saraf satu lapis, itu menjadi jaringan saraf multi-lapisan. Akibatnya, jaringan saraf multi-layer terdiri dari lapisan input, lapisan tersembunyi, dan lapisan output, seperti yang diilustrasikan pada gambar yang diberikan (Thakur et al., 2021).

\section{\emph{Convolutional Neural Network}}
\label{sec:cnn}

\emph{Convolutional Neural Network} (CNN) adalah jenis jaringan saraf khusus yang biasa digunakan dalam tugas pemrosesan gambar untuk mendeteksi dan mengenali objek di dalam gambar (Mehindra, 2020). CNN dirancang untuk memproses data yang diatur dalam struktur seperti kisi, seperti gambar. Mereka dapat dianggap sebagai kombinasi jaringan syaraf tiruan dan metode pembelajaran mendalam (Fonda, 2020). Arsitektur CNN tipikal terdiri dari satu atau lebih lapisan konvolusional, yang melakukan konvolusi pada data input, diikuti oleh lapisan downsampling, sering disebut sebagai lapisan penyatuan. Lapisan-lapisan ini membantu mengurangi dimensi spasial data sambil mempertahankan fitur-fitur penting. Terakhir, satu atau lebih lapisan yang terhubung sepenuhnya digabungkan, mirip dengan jaringan saraf tradisional, untuk memproses fitur yang diekstraksi dan membuat prediksi. Dengan menggunakan lapisan konvolusional, CNN efektif dalam menangkap ketergantungan spasial dan lokal dalam gambar, memungkinkan mereka untuk mempelajari representasi dan pola yang bermakna. Ini membuat CNN sangat cocok untuk tugas-tugas seperti klasifikasi gambar, deteksi objek, dan segmentasi gambar.

Arsitektur CNN dapat dibagi menjadi dua bagian utama, yaitu \emph{Fully-Connected Layer} dan \emph{Multi-Layer Perceptron} (MLP). CNN terdiri dari beberapa lapisan yang melakukan operasi tertentu. Mengikuti arsitektur LeNet5, terdapat empat layer utama dalam sebuah CNN berupa \emph{Convolution Layer, Pooling Layer, Subsampling Layer,} dan \emph{Fully Connected Layer} (Eka Putra, 2016). 

\begin{figure}[H]
  \centering
  \includegraphics[scale=0.8]{gambar/arsitekturcnn.png}
  \caption{Arsitektur \emph{convolutional neural network} oleh LeNet-5 (LeCun et al., 1998)}
  \label{fig:arsicnn}
\end{figure}

\emph{Convolution Layer} menerapkan operasi konvolusi untuk mengekstraksi fitur dari data masukan. Ini menggunakan filter untuk mendeteksi berbagai pola dan struktur dalam data. \emph{Pooling layer} mengurangi dimensi spasial dari fitur yang diekstraksi sambil mempertahankan karakteristik pentingnya. Ini membantu mengurangi kompleksitas komputasi dan membuat jaringan lebih kuat terhadap variasi dalam input data. \emph{Subsampling layer} selanjutnya mengurangi dimensi data melalui \emph{downsampling}, menangkap informasi yang paling signifikan. Terakhir, \emph{Fully-Connected Layer}, yang serupa dengan \emph{Multi-Layer Perceptron} (MLP), menerima fitur yang diproses dan melakukan tugas klasifikasi atau regresi. Ini menghubungkan semua neuron dari lapisan sebelumnya ke lapisan keluaran, memungkinkan jaringan membuat prediksi berdasarkan fitur yang dipelajari. Lapisan-lapisan ini bekerja sama dalam CNN untuk mengekstraksi dan mengubah data input menjadi representasi yang bermakna, yang pada akhirnya memungkinkan jaringan untuk melakukan tugas-tugas seperti pengenalan gambar dan klasifikasi secara efektif.

Pada CNN, data yang diproses oleh jaringan berupa data dua dimensi. Ini memerlukan adaptasi khusus dalam hal operasi linier dan bobot parameter di CNN dibandingkan dengan jaringan saraf tradisional. CNN menggunakan operasi konvolusi untuk operasi linier, dan bobot direpresentasikan dalam empat dimensi sebagai sekumpulan kernel konvolusi. Karena sifat yang melekat pada proses konvolusi, CNN cocok untuk menganalisis data dengan struktur dua dimensi, seperti gambar dan suara.

Sebagai contoh adalah data yang diproses CNN berupa gambar dua dimensi. Istilah "konvolusi" mengacu pada operasi aljabar linier di mana matriks filter dikalikan dengan matriks gambar yang sedang diproses. Operasi ini dikenal sebagai lapisan konvolusi, yang merupakan salah satu dari beberapa jenis lapisan yang dapat hadir dalam CNN. Lapisan konvolusi sangat penting dan banyak digunakan dalam arsitektur CNN. Lapisan lain yang umum digunakan adalah \emph{Pooling layer}, yang mengagregasi informasi dengan mengambil nilai maksimum atau rata-rata dari wilayah piksel di dalam gambar. Setiap lapisan masukan dalam CNN memiliki volume berbeda yang ditandai dengan kedalaman, tinggi, dan lebarnya. Nilai yang didapat pada setiap layer dipengaruhi oleh hasil proses penyaringan dari layer sebelumnya, serta jumlah filter yang digunakan. Model jaringan ini telah menunjukkan kemanjuran yang luar biasa dalam menangani tugas klasifikasi gambar.

\subsection{\emph{Convolutional Layer}}
\label{subsec:cnn}

Operasi konvolusi adalah komponen fundamental dari jaringan saraf convolutional (CNN). Lapisan convolutional dari CNN terdiri dari satu set filter yang dapat dipelajari, juga dikenal sebagai kernel. Setiap filter berukuran kecil, biasanya 3x3, 5x5, atau 7x7, dan menjangkau seluruh kedalaman volume input. Kedalaman filter sesuai dengan jumlah saluran dalam masukan, dengan gambar skala abu-abu memiliki kedalaman 1 dan gambar berwarna memiliki 3 saluran (RGB) (Bezdan et al., 2019). 

Selama proses propagasi maju dalam jaringan saraf convolutional, setiap filter melakukan konvolusi pada volume input dengan menghitung perkalian titik antara entri filter dan nilai input yang sesuai di setiap posisi. Operasi ini diikuti dengan menerapkan fungsi aktivasi nonlinier, seperti sigmoid, tanh, atau ReLU, ke hasilnya, yang menghasilkan peta fitur. Peta fitur mewakili respons filter pada posisi spasial yang berbeda. Peta aktivasi ini ditumpuk sepanjang dimensi kedalaman untuk membentuk volume keluaran. Karakteristik volume output ditentukan oleh tiga hyperparameter: depth, stride, dan padding (Bezdan et al., 2019).

Kedalaman volume keluaran sesuai dengan jumlah filter yang digunakan dalam operasi konvolusi. Setiap filter mempelajari aspek input yang berbeda, seperti tepi, blob, atau warna. Langkahnya menentukan jumlah langkah filter bergerak melintasi input. Langkah 1 menggerakkan filter satu piksel pada satu waktu, sedangkan langkah 2 membuat filter melompati 2 piksel sekaligus, menghasilkan volume output yang lebih kecil secara spasial. Padding digunakan untuk mengontrol ukuran output. Ini melibatkan penambahan nol di sekitar batas volume input untuk menyimpan informasi dan menghindari pengurangan ukuran output. Ada dua pilihan umum: valid convolution, yang berarti tidak ada padding, dan same convolution, di mana ukuran output tetap sama dengan ukuran input (Bezdan et al., 2019).

\subsection{\emph{Pooling Layer}}
\label{subsec:cnn}

Pooling layer di CNN sering digunakan setelah convolutional layer untuk mengurangi dimensi peta fitur. Operasi ini juga dikenal sebagai subsampling atau downsampling. Hyperparameters dari pooling layer mencakup ukuran dan langkah filter. Lapisan penyatuan yang paling umum digunakan memiliki ukuran filter 2 dan langkah 2. Ada dua jenis utama penyatuan lapisan: penyatuan maks dan penyatuan rata-rata. Max pooling memilih nilai maksimum dalam setiap wilayah, sedangkan average pooling menghitung nilai rata-rata. Max pooling lebih umum digunakan daripada pooling rata-rata. Penting untuk diperhatikan bahwa pooling layer tidak memiliki parameter yang dapat dipelajari. Tujuan max pooling adalah untuk menangkap fitur yang paling menonjol dengan memilih nilai terbesar (Bezdan et al., 2019).


\subsection{\emph{Fully Connected Layer}}
\label{subsec:cnn}

Setelah beberapa lapisan konvolusi dan penyatuan, CNN biasanya diakhiri dengan beberapa lapisan yang terhubung sepenuhnya. Output tensor dari lapisan sebelumnya diratakan menjadi vektor, dan kemudian ditambahkan lapisan jaringan saraf tambahan. Lapisan yang terhubung penuh ini biasanya diposisikan menjelang akhir arsitektur, seperti yang digambarkan pada Gambar 3. Untuk mencegah overfitting, teknik regularisasi dropout dapat digunakan pada lapisan yang terhubung penuh ini. Lapisan terhubung penuh terakhir dalam arsitektur terdiri dari jumlah neuron keluaran yang sama dengan jumlah kelas yang akan dikenali (Bezdan et al., 2019).

\section{Visi Komputer}
\label{sec:deteksigesturtubuh}

Visi komputer adalah bidang kecerdasan buatan yang bertujuan untuk meniru persepsi visual manusia dengan menggunakan algoritme dan sensor optik untuk mengekstrak informasi yang relevan dari objek. Tidak seperti metode tradisional yang melibatkan pemeriksaan laboratorium yang memakan waktu dan tenaga, visi komputer menawarkan pendekatan yang lebih efisien. Integrasi sistem pencahayaan dapat digunakan bersamaan dengan MediaPipe untuk meningkatkan proses akuisisi dan pemrosesan gambar. Proses analisis citra melibatkan beberapa langkah. Pertama, tahap pembentukan citra menangkap dan menyimpan citra suatu objek di dalam komputer. Selanjutnya, pra-pemrosesan gambar meningkatkan kualitas gambar untuk meningkatkan detail. Kemudian, segmentasi citra mengidentifikasi dan memisahkan citra objek dari latar belakang. Pada tahap pengukuran citra dilakukan pengukuran berbagai parameter penting. Akhirnya, interpretasi gambar melibatkan analisis dan pemahaman gambar yang diperoleh (Kotappa  et al., 2022).

\begin{figure}[H]
  \centering
  \includegraphics[scale=0.55]{gambar/computervision.png}
  \caption{Visi komputer pada deteksi objek (Shetty et al., 2022)}
  \label{fig:visikomdeteksi}
\end{figure}

Karena kemajuan terbaru dalam teknologi pemrosesan gambar, sekarang dimungkinkan untuk mengembangkan sistem yang dapat mengenali gambar digital. Bidang-bidang seperti matematika, aljabar linier, statistik, Soft Computing, dan ilmu saraf Komputasi telah memainkan peran penting dalam memajukan pemrosesan citra digital. Pengenalan pola, bagian dari visi komputer, berfokus pada identifikasi objek dengan meningkatkan kualitas gambar dan interpretasi melalui transformasi gambar. Ini melibatkan penggalian data dari gambar yang ditangkap sensor untuk membuat penilaian. Tujuan dari visi komputer adalah menciptakan mesin yang dapat "melihat". Kerangka umum dalam visi komputer mencakup pengambilan gambar, pemrosesan awal, ekstraksi fitur, deteksi/segmentasi, pemrosesan tingkat tinggi, dan pengambilan keputusan. Dalam visi komputer, ada kategori seperti analisis morfologi 3D dan pengoptimalan piksel. Optimalisasi piksel melibatkan karakterisasi morfologi piksel untuk pemrosesan gambar yang lebih baik dan pengenalan pola, sedangkan analisis morfologi 3D adalah teori standar dalam pemrosesan gambar komputer dan pengenalan pola (Kotappa  et al., 2022).

Kategori ini mencakup tugas-tugas yang terkait dengan pengambilan wilayah gambar tertentu, seperti kueri mesin telusur, penelusuran manusia, dan penelusuran gambar serupa. Metode segmentasi gambar, seperti pendekatan berbasis intensitas, berbasis warna, dan berbasis bentuk, biasanya digunakan untuk tujuan ini. Deteksi tepi dan segmentasi gambar sangat penting untuk pengenalan dan interpretasi objek dalam berbagai aplikasi visi komputer. Sementara gambar sampel kecil sering digunakan untuk mendemonstrasikan kinerja segmentasi dalam literatur analisis gambar, pengaturan parameter diperlukan untuk anotasi dalam database gambar berskala besar. Tekstur gradien dan pengelompokan tanpa pengawasan dalam ruang fitur digunakan untuk mencapai segmentasi. Segmentasi pelabelan yang akurat sangat penting untuk kinerja pelokalan dan pelokalan batas. Pendekatan pengelompokan dan segmentasi dapat digunakan untuk memperkirakan item dalam citra dengan menetapkan ambang batas pada metode pengelompokan fitur (Kotappa  et al., 2022).


\section{Treadmill}
\label{sec:deteksigesturtubuh}

Treadmill umumnya digunakan di bidang rehabilitasi untuk membantu pasien dengan gangguan gaya berjalan, seperti penyakit Parkinson, stroke, dan cedera tulang belakang, dalam pemulihan kemampuan berjalan mereka. Pelatihan treadmill menawarkan beberapa keuntungan dibandingkan dengan pelatihan di lapangan, antara lain bantuan yang dapat diakses, kebutuhan ruang yang lebih kecil, dan kemampuan untuk mengontrol kecepatan berjalan. Namun, penting untuk dicatat bahwa tujuan akhir rehabilitasi adalah agar pasien mendapatkan kembali kemampuannya untuk berjalan di tanah daripada hanya di atas treadmill. Oleh karena itu, sangat penting untuk memahami dampak berjalan treadmill pada tubuh manusia (Shi et al., 2019).

Bidang studi ini penting untuk tujuan pelatihan dan penelitian, dan ada temuan yang bertentangan dalam studi ilmiah mengenai topik ini. Beberapa penelitian telah menunjukkan bahwa berlari di atas treadmill dengan kecepatan sedang 3,3 hingga 4,8 m/s menghasilkan penurunan panjang langkah dan fase terbang, tetapi iramanya meningkat dibandingkan dengan berlari di tanah. Di sisi lain, penelitian lain telah menemukan kesamaan antara treadmill dan lari di atas tanah dalam hal faktor kinematik seperti sudut adduksi pinggul, rotasi pinggul internal / eksternal, eversi pergelangan kaki, dan rotasi panggul maksimum (Pakbaz et al., 2018).

Belakangan ini, penggunaan perangkat treadmill meningkat secara signifikan, menjadikannya pilihan populer di kalangan individu. Selain itu, treadmill telah mendapatkan popularitas sebagai alat penelitian di berbagai bidang investigasi karena keunggulan metodologinya, termasuk efisiensi ruang, reproduktifitas, dan kemampuan untuk mengontrol variabel seperti kondisi cuaca, kecepatan, dan kemiringan. Namun, penelitian yang dilakukan pada treadmill telah menghasilkan hasil yang bertentangan, yang mengarah ke pertanyaan tentang apakah kelelahan yang disebabkan oleh berlari di atas treadmill dibandingkan berlari di tanah memiliki efek yang berbeda pada distribusi tekanan plantar selama berlari (Pakbaz et al., 2018).

\begin{figure}[H]
  \centering
  \includegraphics[scale=0.28]{gambar/treadmill.png}
  \caption{Olahraga pada treadmill.}
  \label{fig:treadmillrun}
\end{figure}

\section{\emph{Human Pose Estimation}}
\label{sec:deteksigesturtubuh}

Estimasi pose manusia bertujuan untuk menentukan posisi sendi manusia menggunakan berbagai sumber input seperti gambar, urutan gambar, gambar kedalaman, atau data kerangka yang diperoleh dari perangkat penangkap gerak. Tugas ini menantang karena faktor-faktor seperti penampilan manusia yang beragam, variasi siluet, kondisi pencahayaan yang menantang, dan latar belakang yang berantakan. Di masa lalu, estimasi pose bergantung pada teknik seperti deteksi bagian tubuh menggunakan struktur bergambar. Namun, dengan munculnya pembelajaran mendalam, ada dua pendekatan utama yang digunakan: metode holistik dan berbasis bagian (Shetty et al., 2022).

\begin{figure}[H]
  \centering
  \includegraphics[scale=0.95]{gambar/humanpose.png}
  \caption{Contoh hasil \emph{human pose estimation} (Newell et al., 2016)}
  \label{fig:humanposeestimation}
\end{figure}

Metode holistik memproses gambar masukan secara global tanpa secara eksplisit memodelkan bagian tubuh individu dan hubungan spasialnya. Misalnya, DeepPose adalah model holistik yang memperlakukan estimasi pose manusia sebagai masalah regresi gabungan, tanpa secara eksplisit mendefinisikan model grafis atau pendeteksi bagian. Namun, pendekatan holistik mungkin berjuang dengan akurasi di wilayah presisi tinggi karena kesulitan untuk secara langsung meregresi vektor pose kompleks dari gambar (Shetty et al., 2022).

Di sisi lain, metode berbasis bagian secara eksplisit memodelkan bagian tubuh individu dan hubungan spasialnya. Metode ini menganalisis citra masukan dengan mempertimbangkan hubungan antara berbagai bagian tubuh manusia. Dengan memodelkan bagian-bagian secara eksplisit, pendekatan ini bertujuan untuk menangkap informasi yang lebih rinci tentang pose tersebut (Shetty et al., 2022).

Sebaliknya, metode berbasis bagian dalam estimasi pose manusia berfokus pada pendeteksian bagian tubuh individu secara terpisah dan kemudian menggunakan model grafis untuk memasukkan informasi spasial. Misalnya, dalam satu penelitian, alih-alih melatih jaringan pada seluruh gambar, penulis melatih jaringan saraf convolutional (CNN) menggunakan tambalan bagian lokal dan tambalan latar belakang untuk mempelajari probabilitas bersyarat dari kehadiran bagian dan hubungan spasial. Pendekatan lain melibatkan pelatihan beberapa CNN yang lebih kecil untuk klasifikasi bagian tubuh biner independen, diikuti oleh model spasial lemah tingkat tinggi untuk menghilangkan outlier dan memastikan konsistensi pose global. Selanjutnya,  multi-resolusi CNN dikembangkan untuk memperkirakan kemungkinan setiap bagian tubuh menggunakan peta panas, dan model grafis implisit diterapkan untuk menegakkan konsistensi sendi (Shetty et al., 2022).

\section{Mediapipe}
\label{sec:mediapipe}

MediaPipe adalah kerangka kerja yang dirancang untuk membuat jalur pipa untuk melakukan inferensi pada berbagai jenis data sensorik. Dengan menggunakan MediaPipe, dimungkinkan untuk membuat pipa yang terdiri dari komponen modular, termasuk inferensi model, algoritme pemrosesan media, dan transformasi data. Framework ini memungkinkan input data sensorik seperti audio dan video stream ke dalam pipeline, dan menghasilkan deskripsi yang dirasakan seperti lokalisasi objek dan face landmark stream sebagai outputnya (Lugaresi et al., 2019).

\begin{figure}[H]
  \centering
  \includegraphics[scale=1.3]{gambar/mediapipeb2.png}
  \caption{Mediapipe \emph{keypoints} untuk estimasi pose (Bazarevsky et al., 2020).}
  \label{fig:mediapipe}
\end{figure}

MediaPipe adalah platform yang melayani praktisi pembelajaran mesin, termasuk peneliti, mahasiswa, dan pengembang perangkat lunak. Tujuannya adalah untuk membantu pengembangan aplikasi ML yang siap produksi, mendukung publikasi kode penelitian, dan memungkinkan pembuatan prototipe teknologi. Kasus penggunaan utama untuk MediaPipe adalah pembuatan pipeline persepsi yang cepat dan efisien dengan memanfaatkan model inferensi dan komponen yang dapat digunakan kembali. Selain itu, MediaPipe menyederhanakan penerapan teknologi persepsi dalam demonstrasi dan aplikasi di berbagai platform perangkat keras. Platform ini juga memfasilitasi peningkatan iteratif pada alur persepsi melalui bahasa konfigurasi dan alat evaluasinya yang komprehensif (Lugaresi et al., 2019).

\section{Metode Pengujian}
\label{sec:deteksigesturtubuh}

Pada metode pengujian yang dilakukan pada penelitian ini menggunakan beberapa teori dalam menentukan hasil pengujian yang digunakan. Metode pengujian yang digunakan dalam penelitian ini adalah sebagai berikut:

\subsection{\emph{Confusion Matrix}}
\label{subsec:cnn}

Metode penilaian sangat penting untuk mengevaluasi kinerja klasifikasi dan memandu pemodelan pengklasifikasi. Proses klasifikasi melibatkan tiga fase utama: pelatihan, validasi, dan pengujian. Pada fase pelatihan, model dilatih menggunakan pola masukan atau data pelatihan, dan parameter model disesuaikan. Kesalahan pelatihan mengukur seberapa cocok model dengan data pelatihan, tetapi cenderung lebih kecil daripada kesalahan pengujian dan validasi karena cocok dengan data yang sama yang digunakan dalam pelatihan. Tahap pengujian bertujuan untuk memprediksi label kelas untuk data yang tidak terlihat, tetapi kesalahan pengujian tidak dapat diperkirakan karena label kelas yang sebenarnya tidak diketahui. Di sinilah fase validasi masuk, memberikan evaluasi yang tidak bias dari model yang dilatih sambil menyetel hyperparameternya (Tharwat, 2018).

\begin{figure}[H]
  \centering
  \includegraphics[scale=0.5]{gambar/confusionmatrix.png}
  \caption{Contoh \emph{Confusion Matrix} dengan dimensi 2 x 2 (Tharwat, 2018)}
  \label{fig:confusionmatrixex}
\end{figure}

Ada dua jenis masalah klasifikasi: klasifikasi biner dengan dua kelas dan klasifikasi multi-kelas dengan lebih dari dua kelas. Dalam klasifikasi biner, sampel diklasifikasikan sebagai positif (P) atau negatif (N). Model klasifikasi yang dilatih dalam fase pelatihan digunakan untuk memprediksi kelas sebenarnya dari sampel yang tidak diketahui, menghasilkan keluaran yang terpisah atau kontinu. Keluaran diskrit mewakili label kelas yang diprediksi, sedangkan keluaran kontinu menunjukkan probabilitas keanggotaan kelas yang diperkirakan. Matriks kontinjensi atau tabel kontinjensi dengan empat kemungkinan keluaran digunakan untuk mengevaluasi kinerja klasifikasi. Diagonal hijau mewakili prediksi yang benar, sedangkan diagonal merah muda mewakili prediksi yang salah. True positive (TP) adalah sampel positif yang diklasifikasikan dengan benar, false negative (FN) adalah sampel negatif yang salah diklasifikasikan sebagai positif, true negative (TN) adalah sampel negatif yang diklasifikasikan dengan benar, dan false positive (FP) adalah sampel positif yang salah diklasifikasikan sebagai negatif. Matriks konfusi digunakan untuk menghitung berbagai metrik klasifikasi (Tharwat, 2018).

\subsection{\emph{Recall}}
\label{subsec:cnn}

Penarikan kembali suatu pengklasifikasi menunjukkan rasio sampel positif yang diklasifikasikan dengan benar terhadap jumlah total sampel positif. Di sisi lain, spesifisitas, juga dikenal sebagai true negative rate (TNR) atau inverse recall, dihitung dengan membagi jumlah sampel negatif yang diklasifikasikan dengan benar dengan jumlah total sampel negatif. Oleh karena itu, spesifisitas mewakili proporsi sampel negatif yang diklasifikasikan secara akurat, sedangkan sensitivitas mengacu pada proporsi sampel positif yang diklasifikasikan dengan benar (Tharwat, 2018).

\begin{equation}
  \label{eq:KonversiPanjangLangkah}
  \emph{Precision} = \frac{TP}{TP + FN}
\end{equation}

\subsection{\emph{Precision}}
\label{subsec:cnn}

Presisi adalah ukuran yang menghitung rasio sampel positif yang diklasifikasikan dengan benar terhadap jumlah total sampel yang diprediksi positif. Sebaliknya, Nilai Prediktif Negatif (NPV), juga dikenal sebagai presisi terbalik atau True Negative Accuracy (TNA), mengkuantifikasi proporsi sampel negatif yang diklasifikasikan dengan benar ke jumlah total sampel prediksi negatif. Penting untuk dicatat bahwa kedua ukuran ini dipengaruhi oleh data yang tidak seimbang (Tharwat, 2018).

\begin{equation}
  \label{eq:KonversiPanjangLangkah}
  \emph{Precision} = \frac{TP}{TP + FP}
\end{equation}

\subsection{\emph{F-Measure}}
\label{subsec:cnn}

F-measure, juga dikenal sebagai F1-score, adalah metrik yang menghitung rata-rata harmonik presisi dan perolehan. Nilai ukuran-F berkisar dari nol hingga satu, di mana nilai yang lebih tinggi menunjukkan kinerja klasifikasi yang lebih baik. Variasi lain dari pengukuran ini adalah pengukuran F, yang mewakili rata-rata harmonik tertimbang dari presisi dan daya ingat. Penting untuk diperhatikan bahwa metrik ini sensitif terhadap perubahan distribusi data (Tharwat, 2018).

\begin{equation}
  \label{eq:KonversiPanjangLangkah}
  F-Measure = \frac{2 (\emph{precision}\times\emph{recall})}{\emph{precision} + \emph{recall}}
\end{equation}

\subsection{\emph{Error}}
\label{subsec:cnn}

Kesalahan mengacu pada perbedaan persentase antara nilai yang diharapkan atau diprediksi dan hasil aktual. Persentase kesalahan umumnya digunakan untuk menilai keakuratan model peramalan dalam perhitungan tertentu (Deepthi et al., 2017).

\begin{equation}
  \label{eq:KonversiPanjangLangkah}
  \emph{Error} = \frac{nilai \; sebenarnya - nilai \; terukur}{nilai \; sebenarnya} \times 100\%
\end{equation}

\subsection{\emph{Accuracy}}
\label{subsec:cnn}

Akurasi adalah metrik yang digunakan untuk mengevaluasi kinerja model dalam mengklasifikasikan atau memprediksi dataset tertentu. Ini ditentukan dengan membandingkan jumlah prediksi yang benar yang dibuat oleh model dengan jumlah total contoh data yang diuji (Deepthi et al., 2017).

\begin{equation}
  \label{eq:KonversiPanjangLangkah}
  \emph{Accuracy} =  100\% - \%Error
\end{equation}
\cleardoublepage

% Bab 3 desain dan implementasi
\chapter{DESAIN DAN IMPLEMENTASI}
\label{chap:desainimplementasi}

Penelitian ini dilaksanakan sesuai sistem berikut dengan implementasinya. Desain sistem merupakan konsep dari pembuatan dan perancangan infrastuktur yang kemudian diwujudkan dalam bentuk blok diagram alur yang harus dikerjakan. Pada bagian implementasi merupakan pelaksanaan teknis untuk setiap blok pada desain sistem. Pada Gambar 3.1 menunjukan bagan umum metodologi sistem.

\begin{figure}[H]
  \centering
  \includegraphics[scale=0.215]{gambar/blok diagram metodologi3.png}
  \caption{Blok Diagram Kerja Sistem}
  \label{fig:BlokDiagram}
\end{figure}

\section{Desain Sistem}
\label{sec:DesainSistem}

Penelitian ini dilaksanakan sesuai \lipsum[1][1-5]

\section{Pengambilan Data}
\label{sec:PengambilanData}

Pada tahap pertama yaitu pengambilan data, data diperoleh menggunakan kamera pada \emph{smartphone} yang akan direkam dan disimpan untuk kemudian akan digunakan pada proses yang dilakukan pada perangkat komputer/laptop atau dapat menggunakan kamera yang dimiliki oleh laptop atau kamera eksternal yang dihubungkan pada laptop ataupun komputer. Proses pengambilan data dilakukan dengan peraga melakukan aktivitas pada treadmill dengan ditampakkan secara jelas pada tampilan kamera. Setelah terdapat peraga dan tampak jelas pada tampilan maka data citra akan dilakukan pada tahap selanjutnya untuk dideteksi dan segmentasi pose seperti pada Gambar \ref{fig:PengambilanData}.

\begin{figure}[H]
  \centering
  \includegraphics[scale=0.8]{gambar/pengambilan data.png}
  \caption{Proses pengambilan data}
  \label{fig:PengambilanData}
\end{figure}


\section{Deteksi Pose}
\label{sec:DeteksiPose}

Deteksi dari hasil citra untuk dapat mengetahui bentuk postur tubuh manusia menggunakan \emph{Python} dengan \emph{library} OpenCV yaitu MediaPipe. Metode yang digunakan pada MediaPipe menggunakan deteksi pose untuk mendeteksi postur tubuh. Segementasi dilakukan dengan cara peraga melakukan aktivitas jogging pada treadmill dengan menentukan pose melangkah ataupun berlari sesuai dengan olahraga yang akan diteliti. Data yang diproses merupakan data yang telah dilakukan perekaman pada proses pengambilan data dengan menggunakan video yang telah direkam sebelumnya menggunakan alat perekam. Hasil video yang memperlihatkan seseorang dalam keadaan berolahraga pada treadmill melakukan aktivitas berjalan atau berlari akan diproses dalam tahap ini untuk dilakukan deteksi pose. Deteksi pose menggunakan model deteksi yang sudah ada yaitu Mediapipe. Hasil deteksi dari model Mediapipe berupa kerangka yang menyesuaikan bentuk tubuh yang dideteksi dari data video yang dimiliki. Dalam penelitian ini yang berfokuskan pada proses deteksi langkah pada tubuh bagian bawah seperti kaki menyebabkan perlunya pemodelan ulang dan melakukan proses modifikasi pada hasil deteksi model Mediapipe. Dengan begitu proses modifikasi yang dilakukan adalah dengan mengambil kerangka bagian kaki dengan memberikan warna yang berbeda untuk hasil deteksi kaki bagian kiri dan kanan. Proses deteksi pose pada kaki yang dilakukan akan digunakan dalam proses selanjutnya dalam penelitian ini yang digambarkan pada gambar yang terdapat pada Gambar \ref{fig:DeteksiEstimasi}.

\begin{figure}[H]
  \centering
  \includegraphics[scale=0.48]{gambar/deteksi pose.png}
  \caption{Deteksi pose dengan MediaPipe}
  \label{fig:DeteksiEstimasi}
\end{figure}

\subsection{\emph{Pre Processing} Data}
\label{subsec:PreProcessingData}

Hasil deteksi pose dengan bentuk kerangka pada kaki sesuai dengan data video dilakukan tahap \emph{Pre Processing}. \emph{Pre Processing} data dilakukan sebagai proses untuk mempersiapkan data dari input yang digunakan untuk kemudian dilakukannya proses pembuatan model dalam tahap \emph{training}. Terdapat beberapa tahapan yang dilakukan pada \emph{pre processing} data untuk kemudian akan didapatkan dataset yang akan dapat digunakan untuk melakukan pembuatan model dengan proses \emph{training}. Setiap data video yang akan digunakan dalam proses ini akan dilakukan ekstrak dari data video menjadi gambar. Data video yang memiliki resolusi sebesar 1920 x 1080 dengan 25 FPS akan diubah dan diekstrak menjadi satuan gambar untuk digunakan sebagai dataset melalui proses \emph{pre processing} data. Dengan resolusi data demikian akan menghasilkan 25 gambar atau \emph{frame} setiap satu detik dari video tersebut. Hasil ekstrak gambar dalam \emph{Pre Processing} data ini akan kemudian dilanjutkan dalam proses untuk membuat dataset untuk kemudian digunakan dalam proses \emph{training} menjadi model.

\subsection{Augmentasi Data}
\label{subsec:AugmentasiData}

Pada proses deteksi pose didapat hasil deteksi pada data video berupa kerangka pada kaki yang menyesuaikan bentuk tubuh yang sedang ditampilkan. Setelah dapat dideteksi dengan baik menggunakan model Mediapipe dan dimodifikasi untuk dapat terfokus pada bagian kaki, dilakukan proses augmentasi data. Augmentasi data dilakukan sebagai proses untuk mengubah atau memodifikasi gambar agar dapat dengan mudah diproses dalam tahap \emph{training}. Proses ini juga akan dapat membantu dalam meningkatkan akurasi dari model yang akan digunakan nanti karena data telah diproses dan dimodifikasi agar memiliki data-data tambahan yang akan berguna dalam tahap \emph{training}. 

Data dari proses deteksi pose akan dilakukan proses pemindahan hasil deteksi kerangka ke dalam gambar baru dengan penyesuaian yang akan digeneralisasikan terhadap semua data yang didapat. Proses dimulai dengan menentukan posisi kerangka dari data video asli untuk kemudian akan dipindahkan ke gambar baru. Dalam melakukan menentukan posisi kerangka dilakukan dengan deteksi menggunakan bingkai yang menyesuaikan posisi keseluruhan kerangka yang dapat dilihat pada Gambar \ref{fig:PreProcessing1}. Setelah bingkai menyesuaikan posisi kerangka maka akan dipindahkan pada gambar dengan latar belakang hitam yang berisikan hanya kerangka dari hasil deteksi sesuai dengan bingkai. Proses melakukan pemindahan hasil deteksi ke dalam gambar dengan latar belakang hitam dapat ditunjukan pada gambar Gambar \ref{fig:PreProcessing2}. Kemudian gambar akan dilakukan generalisasi terhadap ukuran yang berisikan hanya hasil deteksi sesuai bingkai untuk mempermudah dalam proses selanjutnya dengan membuat ukuran gambar seragam dalam bentuk persegi.

\begin{figure}[H]
  \centering
  \includegraphics[scale=0.48]{gambar/deteksi pose2.png}
  \caption{\emph{Pre Processing} hasil deteksi pose dengan bingkai}
  \label{fig:PreProcessing1}
\end{figure}

\begin{figure}[H]
  \centering
  \includegraphics[scale=0.8]{gambar/deteksi pose3.png}
  \caption{\emph{Pre Processing} hasil deteksi pose pada latar belakang hitam}
  \label{fig:PreProcessing2}
\end{figure}

\subsection{Pelabelan Objek}
\label{subsec:PelabelanObjek}

Hasil dari proses augmentasi data dengan melakukan pemindahan hasil deteksi ke dalam gambar baru yang telah disesuaikan akan diproses menjadi sebuah dataset. Pelabelan objek diperlukan untuk dapat memberikan informasi nama kelas dari objek yang akan dideteksi pada penelitian ini. Proses pada pelabelan objek ini dilakukan setelah didapatkan proses dalam \emph{pre processing} dengan mengekstrak data video menjadi gambar dan dilakukan proses augmentasi dari setiap data gambar yang didapat. Kemudian dari hasil augmentasi dengan memiliki hasil gambar kerangka berdasarkan hasil deteksi dikelompokkan pada kategori atau kelas yang berbeda. Pada penelitian ini proses deteksi yang diinginkan adalah dapat menentukan proses aktivitas melangkah atau berlari dengan mengetahui kaki kanan atau kiri yang sedang berada di depan. Dengan begitu kelas yang dimiliki pada dataset yang akan dibuat dengan terdapat dua label atau kelas yaitu kanan dan kiri. Hasil ekstraksi dan augmentasi yang dilakukan berdasarkan kelas yang diinginkan dapat dilihat untuk kelas kanan pada Gambar \ref{fig:KelasKanan} dan untuk kelas kiri pada Gambar \ref{fig:KelasKiri}.

\begin{figure}[H]
  \centering
  \includegraphics[scale=0.8]{gambar/dataset kanan.png}
  \caption{Hasil \emph{pre processing} untuk kelas kanan}
  \label{fig:KelasKanan}
\end{figure}

\begin{figure}[H]
  \centering
  \includegraphics[scale=0.8]{gambar/dataset kiri.png}
  \caption{Hasil \emph{pre processing} untuk kelas kiri}
  \label{fig:KelasKiri}
\end{figure}


\subsection{Dataset}
\label{subsec:Dataset}

Dataset yang digunakan pada penelitian ini berdasarkan hasil dari data video yang dimiliki dengan mengambil sampel pada proses pengambilan data. Kemudian dilakukan proses \emph{pre processing} dengan melakukan ekstraksi gambar, augmentasi data, dan pelabelan objek yang akan disimpan keseluruhan data yang dimiliki berdasarkan kelas untuk menjadi dataset yang akan digunakan pada penelitian ini. Dataset yang telah dimiliki pada penelitian ini berupa data gambar berisikan bentuk kerangka dari hasil deteksi dengan model Mediapipe yang telah dimodifikasi dan digeneralisasi untuk ukuran gambar. Hasil dari dataset yang akan digunakan untuk dataset kelas kanan ditunjukkan pada Gambar \ref{fig:DatasetKanan} dan untuk dataset kelas kiri ditunjukkan pada Gambar \ref{fig:DatasetKiri}.

\begin{figure}[H]
  \centering
  \includegraphics[scale=0.5]{gambar/folder dataset kanan.png}
  \caption{Dataset untuk kelas Kanan}
  \label{fig:DatasetKanan}
\end{figure}

\begin{figure}[H]
  \centering
  \includegraphics[scale=0.5]{gambar/folder dataset kiri.png}
  \caption{Dataset untuk kelas Kiri}
  \label{fig:DatasetKiri}
\end{figure}


\section{Model}
\label{sec:Klasifikasi}

Dataset yang telah dimiliki maka kemudian dilakukan \emph{training} untuk memperoleh model deteksi. Model deteksi dari dataset akan digunakan untuk melatih model dari sebuah algoritma pada \emph{Machine Learning}. Dalam melakukan klasifikasi menggunakan \emph{Convolutional Neural Networks} (CNN). Proses \emph{training} ini bertujuan agar nantinya komputasi yang dilakukan dalam proses deteksi akan dapat diolah berdasarkan akuisisi data citra yang ingin dideteksi menjadi bentuk atau pola pemahaman yang diinginkan. Hasil \emph{training} akan didapatkan model yang digunakan untuk melakukan klasifikasi atas dataset yang dimiliki yaitu terdapat dua kelas atau label untuk dapat diklasifikasikan menjadi kaki kanan dan kiri. Klasifikasi dalam menentukan aktivitas yang digunakan pada penelitian ini adalah dapat mengetahu langkah dari seseorang yang berjalan atau berlari. Hasil klasifikasi dari model yang telah dibuat untuk menentukan hasil deteksi untuk kelas kanan ditunjukkan pada Gambar \ref{fig:KlasifikasiKanan} dan hasil deteksi untuk kelas kiri ditunjukkan pada Gambar \ref{fig:KlasifikasiKiri}.

\begin{figure}[H]
  \centering
  \includegraphics[scale=0.8]{gambar/klasifikasi kanan.png}
  \caption{Klasifikasi untuk kelas kanan}
  \label{fig:KlasifikasiKanan}
\end{figure}

\begin{figure}[H]
  \centering
  \includegraphics[scale=0.8]{gambar/klasifikasi kiri.png}
  \caption{Klasifikasi untuk kelas kiri}
  \label{fig:KlasifikasiKiri}
\end{figure}


\section{Hasil Deteksi}
\label{sec:HasilDeteksi}

Bentuk hasil klasifikasi yang dibuat adalah mendeteksi pose aktivitas dengan dapat menghitung langkah dan waktu yang ditempuh. Nilai langkah dan waktu yang ditempuh akan digunakan dalam perhitungan selanjutnya. Banyaknya jumlah langkah yang didapat saat hasil deteksi digunakan sebagai nilai variable pertama yang akan digunakan dalam penentuan perhitungan kalori. Langkah dideteksi dan dihitung seberapa banyak langkah yang dilakukan saat proses deteksi. Waktu tempuh saat proses deteksi merupakan nilai variabel kedua yang akan digunakan dalam penentuan perhitungan kalori. Waktu tempuh dimulai saat dideteksi pertama kali nilai langkah yang ditemukan hingga saat akhir langkah tidak ada penambahan kembali yang menandakan proses deteksi telah selesai. Hasil deteksi akan ditampilkan seiring dengan proses deteksi yang dilakukan pada data citra seperti pada Gambar \ref{fig:HasilDeteksi} dan pada akhir proses deteksi akan menampilkan hasil akumulasi akhir dari hasil deteksi terhadap deteksi langkah dan waktu tempuh. Selain itu juga terdapat tampilan hasil perhitungan dan prediksi yang diharapkan dalam penelitian ini. Hasil tampilan untuk hasil deteksi di akhir sebagai akumulasi deteksi ditunjukkan pada Gambar \ref{fig:HasilDeteksi2}.

\begin{figure}[H]
  \centering
  \includegraphics[scale=0.48]{gambar/hasil deteksi.png}
  \caption{Tampilan hasil deteksi saat proses deteksi}
  \label{fig:HasilDeteksi}
\end{figure}

\begin{figure}[H]
  \centering
  \includegraphics[scale=0.8]{gambar/hasil deteksi2.png}
  \caption{Tampilan akumulasi hasil deteksi}
  \label{fig:HasilDeteksi2}
\end{figure}

\subsection{Hasil Deteksi Langkah}
\label{subsec:HasilLangkah}

Banyaknya jumlah langkah yang didapat saat hasil deteksi digunakan sebagai nilai variable pertama yang akan digunakan dalam penentuan perhitungan kalori. Langkah dideteksi dan dihitung seberapa banyak langkah yang dilakukan saat proses deteksi. Nilai banyaknya jumlah langkah akan disimpan dan akan digunakan pada saat proses perhitungan kalori setelah proses deteksi telah selesai dilakukan.

\subsection{Hasil Waktu Tempuh}
\label{subsec:HasilWaktu}

Waktu tempuh saat proses deteksi merupakan nilai variabel kedua yang akan digunakan dalam penentuan perhitungan kalori. Waktu tempuh dimulai saat dideteksi pertama kali nilai langkah yang ditemukan hingga saat akhir langkah tidak ada penambahan kembali yang menandakan proses deteksi telah selesai. Nilai waktu akan dibutuhkan dalam satuan waktu menit untuk proses perhitungan kalori.

\section{Prediksi Kalori}
\label{sec:PrediksiKalori}

Prediksi kalori dilakukn dengan dua metode, yaitu menggunakan metode regresi linear dan menggunakan perhitungan rumus EC (Exercise Calories) berdasarkan satuan ukuran MET (Metabolic Equivalent). Kedua metode prediksi ini digunakan sebagai pembanding dalam melakukan analisa terhadap hasil yang didapatkan dari metode prediksi menggunakan metode regresi linear dengan perhitungan rumus. Data yang digunakanan dalam proses prediksi kalori diambil berdasarkan hasil klasifikasi dan hasil deteksi yang telah dilakukan sebelumnya. Hasil deteksi berupa banyaknya langkah dan waktu tempuh digunakan untuk proses prediksi kalori baik dengan metode regresi linear maupun metode perhitungan rumus.

\subsection{Regresi Linear}
\label{subsec:PrediksiRegresi}

Regresi merupakan suatu teknik analisis untuk mengidentifikasi relasi antar dua variabel atau lebih. Model regresi linear dilakukan dengan cara membuat dataset terlebih dahulu yang diatur sesuai dengan variabel-variabel yang dibutuhkan pada penelitian ini. Terdapat variabel independen dan variabel dependen sebagai bentuk model dataset yang akan digunakan pada regresi linear yang digunakan. Variabel independen yang digunakan pada dataset ini adalah waktu tempuh dan jarak tempuh. Sedangkan untuk variabel dependen yang digunakan adalah jumlah kalori yang terbakar.

Dalam membuat model regresi linear, perlu dilakukan training dan pencarian dataset terlebih dahulu menggunakan alat bantu treadmill yang memiliki perhitungan jumlah kalori yang terbakar dengan menvariasikan sebanyak-banyaknya variabel yang akan digunakan pada model regresi linear. Setelah melakukan percobaan dan pencarian dataset maka akan didapatkan dataset untuk model regresi linear. Dengan didapatkannya model regresi linear, maka saat melakukan akuisisi data citra untuk diprediksi jumlah pembakaran kalori dengan menghasilkan hasil deteksi yang akan diregresikan dengan model regresi yang telah dimiliki dapat dilakukan prediksi jumlah pembakaaran kalori dari data citra tersebut.

\subsection{Perhitungan Rumus}
\label{subsec:PrediksiPerhitungan}

Perhitungan kalori dilakukan dengan mengacu pada nilai satuan ukuran MET (Metabolic Equivalent). Satuan MET akan mendapat pengukuran untuk konsumsi oksigen dan pembakaran kalori. Nilai dari satuan MET dapat didefinisikan pada Persamaan 1.

\begin{equation}
  \label{eq:SatuanMET}
  1 \mathbf{MET} = 3,5 ml O_2  / KG / min
\end{equation}

Berdasarkan nilai satuan ukuran MET, didapatkan suatu persamaan untuk menghitung pembakaran kalori yang didefiniskan pada Persamaan 2.

\begin{equation}
  \label{eq:RumusKalori}
  \mathbf{Cal} = \frac{MET x 3,5 x BB}{200} calories / min
\end{equation}

Pada persamaan pembakaran kalori yang akan digunakan untuk melakukan perhitungan pembakaran kalori dari aktivitas yang dilakukan dibutuhkan beberapa nilai variabel untuk mendapatkah hasil total pembakaran kalori, yaitu nilai MET, nilai berat badan (BB), dan nilai waktu tempuh dalam menit.

Setiap aktivitas memiliki nilai MET yang berbeda-beda dan telah ditentukan oleh peneliti yang telah merangkum banyak aktivitas untuk ditentukan berapa nilai MET yang dihasilkan. Pada aktivitas olahraga yang difokuskan saat ini adalah jogging pada treadmill juga memiliki perbedaan nilai MET yang dipengaruhi oleh kecepatan jogging. Kecepatan aktivitas jogging dapat diketahui dengan nilai hasil deteksi berupa banyak langkah dan waktu tempuh untuk menentukan kecepatan dengan menggunakan Persamaan 3.

\begin{equation}
  \label{eq:RumusKecepatan}
  \mathbf{Kecepatan} = \frac{Jarak}{Waktu}
\end{equation}

Untuk nilai jarak dilakukan dengan melakukan model regresi polinomial dari panjang langkah dari hasil kali kecepatan dan waktu yang dibagi dengan banyak langkah sesuai dataset yang diperoleh dari pengumpulan data pada treadmill. Model regresi digunakan untuk menentukan panjang jarak yang akan digunakan saat melakukan akuisisi data citra untuk dilakukan prediksi dengan menghasilkan data banyak langkah dan waktu tempuh. Kemudian banyak langkah akan dikalikan dengan hasil regresi untuk mentukan panjang langkah dan hasil pengalian tersebut dibagi dengan waktu tempuh untuk menemukan kecepatan. Dengan diperolehnya kecepatan maka dapat menentukan MET dengan membuat model regresi MET berdasarkan data yang telah dilakukan peneliti dan dapat dilanjutkan untuk mengetahui kalori dengan menggunakan rumus pada Persamaan 1.

\cleardoublepage

% Bab 4 pengujian dan analisis
\chapter{PENGUJIAN DAN ANALISIS}
\label{chap:pengujiananalisis}

% Ubah bagian-bagian berikut dengan isi dari pengujian dan analisis

Penelitian dilakukan dengan melakukan uji dan anaisa dari sistem yang telah dibuat berdasarkan langkah pada metodologi yang telah dilaksanakan. Pengujian ini dilakukan untuk menguji kemampuan sistem yang telah dibuat dalam menjawab permasalahan yang dijadikan acuan pada penelitian ini untuk mendapatkan hasil dari tujuan yang ingin didapat. Skenario pengujian dan pembahasan yang dilakukan pada penelitian ini, antara lain:

\begin{enumerate}[nolistsep]

  \item Pengujian Hasil \emph{Training}, \emph{Validation} dan \emph{Testing} Model

  \item Pengujian Deteksi Langkah 

  \item Pengujian Prediksi Kalori
  
  \item Pengujian Performa Berdasarkan Jarak Kamera
  
  \item Pengujian Performa Berdasarkan Posisi Kamera
  
  \item Pengujian Performa Berdasarkan Intensitas Cahaya
  
  \item Pengujian Sistem secara \emph{Real Time}

\end{enumerate}

\section{Pengujian Hasil \emph{Training}, \emph{Validation} dan \emph{Testing} Model}
\label{sec:PengujianTrainingValidation}

Pengujian pembuatan model berdasarkan hasil \emph{training} dan \emph{validation} dengan menggunakan dataset dengan jumlah keseluruhan yaitu 1731 sampel data. Dengan jumlah sample training sebanyak 1385 dan sampel validation sebanyak 346. Setelah dilakukan proses \emph{training} dan \emph{validation} terhadap dataset yang telah ditentukan oleh sampel data, didapatkan hasil pengujian akurasi dengan tingkat akurasi training sebesar 0,951 dan tingkat akurasi validation sebesar 0,977. Kemudian didapatkan hasil pengujian loss pada training sebesar 0,127 dan loss pada validation sebesar 0,059. Hasil pengujian ditunjukkan pada grafik nilai akurasi dan loss pada proses \emph{training} dan \emph{validation} seperti pada Gambar \ref{fig:HasilTrainingValidation}.

\begin{figure}[H]
  \centering
  \includegraphics[scale=0.6]{gambar/hasil training dan validation w.jpg}
  \caption{Grafik hasil \emph{training} dan \emph{validation}}
  \label{fig:HasilTrainingValidation}
\end{figure}

Pengujian dilanjutkan dengan melakukan \emph{testing} model dengan menggunakan dataset yang sudah dimiliki dengan jumlah keseluruhan yaitu 347 sampel data. Dataset yang digunakan merupakan dataset yang sudah dilakukan filtrasi yang hanya digunakan pada pengujian \emph{testing} saja. Hasil pengujian \emph{testing} model didapatkan akurasi sebesar 95\% dengan hasil deteksi benar untuk kelas kanan sebanyak 170 sampel 96\% dan kelas kiri sebanyak 161 sampel 95\%. Pengujian ditunjukkan dengan confusion matrix pada Gambar \ref{fig:HasilTesting} dan Tabel \ref{tb:ClassificationReport} merupakan classification report dari hasil pengujian yang telah dilakukan pada pengujian \emph{testing} model penelitian ini.

\begin{figure}[H]
  \centering
  \includegraphics[scale=0.9]{gambar/cm normalized.png}
  \caption{\emph{Confusion Matrix} hasil \emph{testing} model}
  \label{fig:HasilTesting}
\end{figure}

\begin{longtable}{|c|c|c|c|c|}
  \caption{\emph{Classification Report} hasil pengujian \emph{testing} model}
  \label{tb:ClassificationReport}                                   \\
  \hline
  \rowcolor[HTML]{C0C0C0}
   & \textbf{Precision} & \textbf{Recall} & \textbf{F1-Score} & \textbf{Support} \\
  \hline
  kanan     & 0,91    & 1,00    & 0,96    & 170         \\
  \hline
  kiri      & 1,00    & 0,91    & 0,95    & 177           \\
  \hline
  Accuracy  &         &         & 0,95    & 347            \\
  \hline
\end{longtable}

Model yang digunakan pada penelitiian ini terdapat dua macam yang akan digunakan pada proses klasifikasi langkah. Pada model yang telah dibuat dan telah dianalisa sebelumnya merupakan model yang digunakan pada sebagian besar pengujian pada penelitian ini. Posisi yang diambil untuk melakukan pembuatan model dengan posisi untuk melakukan klasifikasi dari arah samping objek yang akan dilakukan deteksi. Pada pengujian terdapat pengujian yang akan dilakukan dari posisi belakang objek. Dengan menggunakan posisi tersebut maka dibutuhkan model klasifikasi yang berbeda melihat dari proses estimasi pose dan data yang didapat memiliki karakteristik yang berbeda pada pengujian ini. 

Pembuatan model kedua dilakukan kembali proses estimasi pose untuk mendapatkan dataset yang kemudian dilakukan pembuatan model. Hasil estimasi dilakukan proses ekstrak fitur dan proses pelabelan objek. Dataset yang dibuat memiliki jumlah kelas yang sama yaitu terdapat dua kelas kanan dan kelas kiri. Hasil ekstrak fitur dan dilakukannya proses \emph{preprocessing} menghasilkan gambar dataset kelas kanan dan kelas kiri. Gambar \ref{fig:KelasBelakangKanan} menunjukkan gambar dataset kelas kanan dan Gambar \ref{fig:KelasBelakangKiri} menunjukkan gambar dataset kelas kiri.

\begin{figure}[H]
  \centering
  \includegraphics[scale=0.45]{gambar/dataset belakang kanan.jpg}
  \caption{Hasil \emph{preprocessing} untuk model kedua kelas kanan}
  \label{fig:KelasBelakangKanan}
\end{figure}

\begin{figure}[H]
  \centering
  \includegraphics[scale=0.45]{gambar/dataset belakang kiri.jpg}
  \caption{Hasil \emph{preprocessing} untuk model kedua kelas kiri}
  \label{fig:KelasBelakangKiri}
\end{figure}

Hasil gambar digunakan pada dataset yang akan dilakukan pelabelan objek untuk dapat memberikan informasi nama kelas dari objek yang akan dideteksi pada model ini. Label yang diberikan pada dataset sesuai dengan kelas yaitu label kanan dan label kiri. Dari setiap gambar yang telah dilakukan ekstraksi fitur dilakukan pemberian label sesuai kelas yang digunakan Hasil pelabelan objek gambar tersebut ditunjukkan pada Tabel 
\noindent
\begin{longtable}{|c|c|}
  \caption{Hasil anotasi dari pelabelan objek model kedua}
  \label{tb:HasilAnotasi}  \\
  \hline
  \rowcolor[HTML]{C0C0C0}
  \textbf{Kelas} & \textbf{Jumlah Anotasi}  \\
  \hline
  Kanan           & 896    \\
  \hline
  Kiri            & 860    \\
  \hline
  \textbf{Total}  & \textbf{1.756}  \\
  \hline
\end{longtable}

Dataset yang telah dilakukan anotasi untuk pelabelan objek dilakukan proses pembuatan model klasifikasi. Dengan menggunakan arsitetkur yang sama dilakukan proses \emph{training} untuk mendapatkan hasil \emph{training} dan \emph{validation} yang kemudian dilakukan \emph{testing} model yang telah dibuat terhadap dataset. Pengujian pembuatan model berdasarkan hasil \emph{training} dan \emph{validation} dengan menggunakan dataset dengan jumlah keseluruhan yaitu 1756 sampel data. Dengan jumlah sample training sebanyak 1405 dan sampel validation sebanyak 317. Setelah dilakukan proses \emph{training} dan \emph{validation} terhadap dataset yang telah ditentukan oleh sampel data, didapatkan hasil pengujian akurasi dengan tingkat akurasi training sebesar 0,959 dan tingkat akurasi validation sebesar 0,972. Kemudian didapatkan hasil pengujian loss pada training sebesar 0,107 dan loss pada validation sebesar 0,053. Hasil pengujian ditunjukkan pada grafik nilai akurasi dan loss pada proses \emph{training} dan \emph{validation} seperti pada Gambar \ref{fig:HasilTrainingValidationModel2}.

\begin{figure}[H]
  \centering
  \includegraphics[scale=0.75]{gambar/belakang w.jpg}
  \caption{Grafik hasil \emph{training} dan \emph{validation} model kedua}
  \label{fig:HasilTrainingValidationModel2}
\end{figure}

Pengujian dilanjutkan dengan melakukan \emph{testing} model dengan menggunakan dataset yang sudah dimiliki dengan jumlah keseluruhan yaitu 317 sampel data. Dataset yang digunakan merupakan dataset yang sudah dilakukan filtrasi yang hanya digunakan pada pengujian \emph{testing} saja. Hasil pengujian \emph{testing} model didapatkan akurasi sebesar 90\% dengan hasil deteksi benar untuk kelas kanan sebanyak 158 sampel 90\% dan kelas kiri sebanyak 182 sampel 91\%. Pengujian ditunjukkan dengan confusion matrix pada Gambar \ref{fig:HasilTestingModel2} dan Tabel \ref{tb:ClassificationReportModel2} merupakan classification report dari hasil pengujian yang telah dilakukan pada pengujian \emph{testing} model penelitian ini.

\begin{figure}[H]
  \centering
  \includegraphics[scale=0.98]{gambar/cm model belakang.png}
  \caption{\emph{Confusion Matrix} hasil \emph{testing} model kedua}
  \label{fig:HasilTestingModel2}
\end{figure}

\begin{longtable}{|c|c|c|c|c|}
  \caption{\emph{Classification Report} hasil pengujian \emph{testing} model kedua}
  \label{tb:ClassificationReportModel2}                                   \\
  \hline
  \rowcolor[HTML]{C0C0C0}
   & \textbf{Precision} & \textbf{Recall} & \textbf{F1-Score} & \textbf{Support} \\
  \hline
  kanan     & 0,94    & 0,87    & 0,90    & 179         \\
  \hline
  kiri      & 0,87    & 0,94    & 0,91    & 172           \\
  \hline
  Accuracy  &         &         & 0,90    & 351            \\
  \hline
\end{longtable}

\section{Pengujian Hasil Deteksi}
\label{sec:PengujianDeteksi}

Proses deteksi yang dilakukan dengan menggunakan model yang telah dibuat dilakukan pengujian dari hasil yang didapat dari hasil deteksi dengan perhitungan yang sebenarnya. Pengujian dilakukan dengan membuat data sebenarnya dengan melakukan perhitungan langkah dari akuisisi maupun data video yang digunakan dalam percobaan dan dibandingkan dengan hasil perhitungan langkah dari hasil deteksi. Tabel \ref{tb:PengujianDeteksi} menunjukkan hasil perhitungan data sebenarnya dengan data hasil deteksi terhadap langkah pada data video. 

\begin{longtable}{|c|c|c|}
  \caption{Pengujian Hasil Deteksi}
  \label{tb:PengujianDeteksi}                                   \\
  \hline
  \rowcolor[HTML]{C0C0C0}
  \textbf{Percobaan} & \textbf{Langkah} & \textbf{Deteksi Langkah} \\
  \hline
  1   & 241   & 328    \\
  \hline
  2   & 169   & 170    \\
  \hline
  3   & 302   & 302    \\
  \hline
  4   & 246   & 248    \\
  \hline
  5   & 321   & 322    \\
  \hline
  6   & 220   & 218    \\
  \hline
  7   & 274   & 274    \\
  \hline
  8   & 260   & 256    \\
  \hline
  9   & 276   & 260    \\
  \hline
  10   & 321   & 334    \\
  \hline
  11   & 309   & 314    \\
  \hline
  12   & 317   & 276    \\
  \hline
  13   & 276   & 244    \\
  \hline
  14   & 259   & 260    \\
  \hline
  15   & 292   & 231    \\
  \hline
  16   & 117   & 116    \\
  \hline
  17   & 141   & 144    \\
  \hline
\end{longtable}

Pengujian dilakukan setelah didapatkan data hasil deteksi yang telah didapat dan juga data sebenarnya berdasarkan hasil perhitungan yang dilakukan terhadap data video. Setiap percobaan dilakukan analisa terkait perbandingan hasil yang didapat antara data hasil deteksi dan data perhitungan sebenarnya. Perbandingan dilakukan dengan mencari nilai eror dari hasil perbedaan yang didapat dari setiap data percobaan untuk kemudian ditentukan dan dikalkulasikan dalam hasil akurasi dan error yang didapat dari analisa tersebut. Hasil analisa yang didapat dengan melakukan analisa pengujian hasil deteksi didapatkan hasil akurasi rata-rata sebesar 96,14\% dengan hasil eror rata-rata sebesar 3,86\%. Tabel \ref{tb:AnalisaDeteksi} menunjukkan hasil analisa dari setiap percobaan yang dilakukan analisa pengujian.

\begin{longtable}{|c|c|c|c|c|c|}
  \caption{Analisa Pengujian Hasil Deteksi}
  \label{tb:AnalisaDeteksi}                                   \\
  \hline
  \rowcolor[HTML]{C0C0C0}
  \textbf{Percobaan} & \textbf{Langkah} & \textbf{Deteksi Langkah} & \textbf{Eror} & \textbf{Eror\%} & \textbf{Akurasi\%} \\
  \hline
  1   & 241   & 238   & 3    & 1,24\%    & 98,76\%   \\
  \hline
  2   & 169   & 170   & 1    & 0,59\%    & 99,41\%   \\
  \hline
  3   & 302   & 302   & 0    & 0\%       & 100\%     \\
  \hline
  4   & 246   & 248   & 2    & 0,81\%    & 99,19\%   \\
  \hline
  5   & 321   & 322   & 1    & 0,31\%    & 99,69\%   \\
  \hline
  6   & 220   & 218   & 2    & 0,91\%    & 99,09\%   \\
  \hline
  7   & 274   & 274   & 0    & 0\%       & 100\%   \\
  \hline
  8   & 260   & 256   & 4    & 1,54\%    & 98,46\%   \\
  \hline
  9   & 276   & 260   & 16   & 5,80\%    & 94,20\%   \\
  \hline
  10   & 321   & 334  & 13   & 4,05\%    & 95,95\%   \\
  \hline
  11   & 309   & 314  & 5    & 1,62\%    & 98,38\%   \\
  \hline
  12   & 317   & 276  & 41   & 12,93\%   & 87,07\%   \\
  \hline
  13   & 276   & 244  & 32   & 11,59\%   & 88,41\%   \\
  \hline
  14   & 259   & 260  & 1    & 0,39\%    & 99,61\%   \\
  \hline
  15   & 292   & 231  & 61   & 20,89\%   & 79,11\%   \\
  \hline
  16   & 117   & 116  & 1    & 0,85\%    & 99,15\%   \\
  \hline
  17   & 141   & 144  & 3    & 2,13\%    & 97,87\%   \\
  \hline

  \multicolumn{4}{|c|}{\textbf{Rata-rata}} & 3,86\% & 96,14\% \\
  \hline
\end{longtable}


\section{Pengujian Prediksi Kalori}
\label{sec:PengujianPrediksi}

Pengujian pada prediksi jumlah kalori yang terbakar dilakukan setelah melakukan pengujian pada model deteksi yang telah dilakukan. Prediksi dilakukan dengan dua metode, yaitu regresi linear dan perhitungan rumus. Dengan menggunakan dataset berupa citra video yang akan digunakan untuk melakukan prediksi kalori sebanyak 17 sampel video sehingga terdapat 17 percobaan yang dilakukan. Video pengujian menggunakan variasi kecepatan dan hasil kalori yang didapatkan. Variasi kecepatan yang digunakan adalah 3 km/jam, 6 km/jam, 8 km/jam, 9 km/jam dan 12 km/jam. Kemudian variasi hasil kalori yang didapatkan adalah 5, 10 dan 20 kcal. 

\subsection{Prediksi dengan Regresi Linear}
\label{subsec:PengujianPrediksiRegresi}

Pada pengujian dengan regresi linear dengan melakukan proses deteksi dan menggunakan model regresi linear dalam melakukan proses prediksi kalori didapatkan beberapa data pendukung dalam hasil deteksi dan data dari hasil regresi prediksi kalori. Data video yang digunakan sebagai pengujian dilakukan proses deteksi sehingga mendapatkan beberapa data yang akan digunakan dalam proses prediksi kalori. Data deteksi langkah merupakan data hasil deteksi langkah yang didapat dari sistem terhadap data video. Data waktu didapat dari hasil waktu tempuh yang dilakukan selama melakukan proses deteksi pada data video. Dengan data yang didapat dari hasil deteksi berupa langkah dan waktu akan dilakukan akumulasi untuk mendapatkan data jarak dan prediksi kalori dengan regresi linear berdasarkan model yang sudah dibuat sehingga mendapatkan data prediksi kalori. Tabel \ref{tb:PengujianPrediksiRegresi} menunjukkan hasil deteksi dan hasil prediksi kalori dengan menggunakan regresi linear.

\begin{longtable}{|c|c|c|c|c|}
  \caption{Pengujian Prediksi dengan Regresi Linear}
  \label{tb:PengujianPrediksiRegresi}                                   \\
  \hline
  \rowcolor[HTML]{C0C0C0}
  \textbf{Percobaan} & \textbf{Deteksi Langkah} & \textbf{Jarak} & \textbf{Waktu} & \textbf{Prediksi Kalori} \\
  \hline
  1   & 238   & 167,181    & 2:50    & 11,652  \\
  \hline
  2   & 170   & 118,437    & 1:35    & 8,245   \\
  \hline
  3   & 302   & 293,143    & 2:02    & 20,534  \\
  \hline
  4   & 248   & 284,417    & 1:41    & 19,927  \\
  \hline
  5   & 322   & 287,577    & 2:03    & 20,14   \\
  \hline
  6   & 218   & 286,18     & 1:21    & 20,056  \\
  \hline
  7   & 274   & 175,936    & 2:49    & 12,267  \\
  \hline
  8   & 256   & 177,101    & 2:49    & 12,349  \\
  \hline
  9   & 260   & 177,780    & 2:49    & 12,397  \\
  \hline
  10   & 334   & 505,568   & 2:01    & 35,488  \\
  \hline
  11   & 314   & 312,691   & 2:02    & 21,909  \\
  \hline
  12   & 276   & 155,812   & 2:02    & 10,865  \\
  \hline
  13   & 244   & 216,140   & 1:39    & 15,119  \\
  \hline
  14   & 260   & 280,461   & 1:40    & 19,647  \\
  \hline
  15   & 231   & 150,241   & 1:45    & 10,478  \\
  \hline
  16   & 116   & 78,449    & 1:08    & 5,435  \\
  \hline
  17   & 144   & 142,144   & 0:59    & 9,922  \\
  \hline
\end{longtable}

Pada pengujian prediksi kalori dilakukan berdasarkan hasil deteksi yang didapat. Analisa pengujian hasil prediksi kalori dilakukan dengan melakukan perbandingan data prediksi kalori dari sistem yang digunakan dengan data kalori treadmill sebagai data sebenarnya atau data \emph{actual score}. Perbandingan hasil prediksi kalori dengan kalori treadmill didapatkan data eror yang kemudian dilakukan perhitungan persentase berdasarkan data sebenarnya pada data kalori treadmill. Proses analisa pengujian dilakukan terhadap seluruh data video pada data percobaan. Hasil analisa yang didapat dengan melakukan analisa pengujian prediksi kalori didapatkan hasil akurasi rata-rata sebesar 80,93\% dengan hasil eror rata-rata sebesar 19,07\%. Tabel \ref{tb:AnalisaPrediksiRegresi} menunjukkan hasil analisa pengujian dari setiap percobaan yang dilakukan analisa prediksi kalori dengan regresi linear.

\begin{longtable}{|c|c|c|c|c|c|}
  \caption{Analisa Pengujian Prediksi dengan Regresi Linear}
  \label{tb:AnalisaPrediksiRegresi}                                   \\
  \hline
  \rowcolor[HTML]{C0C0C0}
  \textbf{Percobaan} & \textbf{Kalori Treadmill} & \textbf{Prediksi Kalori} & \textbf{Eror} & \textbf{Eror\%} & \textbf{Akurasi\%} \\
  \hline
  1   & 10   & 11,652   & 1,652    & 16,52\%     & 83,48\%   \\
  \hline
  2   & 10   & 8,245    & 1,755    & 17,55\%     & 82,45\%   \\
  \hline
  3   & 20   & 20,534   & 0,534    & 2,67\%      & 97,33\%   \\
  \hline
  4   & 20   & 19,927   & 0,073    & 0,365\%     & 99,635\%  \\
  \hline
  5   & 20   & 20,14    & 0,14     & 0,7\%       & 99,3\%    \\
  \hline
  6   & 20   & 20,056   & 0,056    & 0,28\%      & 99,72\%   \\
  \hline
  7   & 10   & 12,267   & 2,267    & 22,67\%     & 77,33\%   \\
  \hline
  8   & 10   & 12,349   & 2,349    & 23,49\%     & 76,51\%   \\
  \hline
  9   & 10   & 12,397   & 2,397    & 23,97\%     & 76,03\%   \\
  \hline
  10   & 20   & 35,488   & 15,488   & 77,44\%     & 22,56\%   \\
  \hline
  11   & 20   & 21,909   & 1,909    & 9,55\%      & 90,45\%   \\
  \hline
  12   & 20   & 10,865   & 9,135    & 45,67\%     & 54,33\%   \\
  \hline
  13   & 20   & 15,119   & 4,881    & 24,41\%     & 75,59\%   \\
  \hline
  14   & 20   & 19,647   & 0,353    & 1,77\%      & 98,23\%   \\
  \hline
  15   & 20   & 10,478   & 9,522    & 47,61\%     & 52,39\%   \\
  \hline
  16   & 5   & 5,435    & 0,435     & 8,70\%      & 91,03\%   \\
  \hline
  17   & 10   & 9,922    & 0,078    & 0,78\%      & 99,22\%   \\
  \hline

  \multicolumn{4}{|c|}{\textbf{Rata-rata}} & 19,07\% & 80,93\% \\
  \hline
\end{longtable}

\subsection{Prediksi dengan Perhitungan Rumus}
\label{subsec:PengujianPrediksiPerhitungan}

Pada pengujian dengan perhitungan rumus berdasarkan MET dilakukan dengan proses deteksi dan menggunakan perhitungan rumus untuk menghitung prediksi kalori yang terbakar didapatkan data pendukung dalam hasil deteksi dan data dari hasil perhitungan prediksi kalori. Data video yang digunakan sebagai pengujian dilakukan proses deteksi sehingga mendapatkan beberapa data yang akan digunakan dalam proses prediksi kalori. Data deteksi kecepatan merupakan data hasil deteksi yang sudah diproses untuk bisa menentukan kecepatan yang dilakukan dalam proses deteksi. Data MET merupakan data prediksi MET berdasarkan kecepatan yang ditempuh dengan menggunakan model regresi MET. Data waktu didapat dari hasil waktu tempuh yang dilakukan selama proses deteksi pada data video. Dengan data yang didapat dari hasil deteksi dan proses yang dilakukan dapat dilakukan perhitungan rumus dengan menggunakan data MET dan waktu untuk menghitung dan mendapatkan data prediksi kalori dengan rumus yang digunakan. Tabel \ref{tb:PengujianPrediksiPerhitungan} menunjukkan hasil deteksi dan hasil prediksi kalori dengan menggunakan perhitungan rumus.

\begin{longtable}{|c|c|c|c|c|}
  \caption{Pengujian Prediksi dengan Perhitungan Rumus}
  \label{tb:PengujianPrediksiPerhitungan}                                   \\
  \hline
  \rowcolor[HTML]{C0C0C0}
  \textbf{Percobaan} & \textbf{Deteksi Kecepatan} & \textbf{MET} & \textbf{Waktu} & \textbf{Kalori} \\
  \hline
  1   & 3,626   & 2,003    & 2:50    & 6,788   \\
  \hline
  2   & 5,016   & 2,733    & 1:35    & 4.744   \\
  \hline
  3   & 8,943   & 9,323    & 2:02    & 22.462  \\
  \hline
  4   & 10,893  & 10,721   & 1:41    & 20.575  \\
  \hline
  5   & 8,151   & 8,533    & 2:03    & 22.126  \\
  \hline
  6   & 13,041  & 11,985   & 1:21    & 19.331  \\
  \hline
  7   & 3,572   & 2,178    & 2:49    & 6,979  \\
  \hline
  8   & 3,862   & 1,029    & 2:49    & 3,296  \\
  \hline
  9   & 3,820   & 1,192    & 2:49    & 3,819  \\
  \hline
  10   & 15,114   & 13,453   & 2:01    & 30,862  \\
  \hline
  11   & 9,311   & 9,609   & 2:02    & 22,224  \\
  \hline
  12   & 4,557   & 1,396   & 2:02    & 3,228  \\
  \hline
  13   & 8,105   & 8,471   & 1:39    & 15,900  \\
  \hline
  14   & 10,234   & 10,270   & 1:40    & 19,470  \\
  \hline
  15   & 6,326   & 5,832   & 1:45    & 11,057  \\
  \hline
  16   & 4,284   & 0,499    & 1:08    & 0,643  \\
  \hline
  17   & 8,966   & 9,322   & 0:59    & 10,427  \\
  \hline
\end{longtable}

Pada pengujian prediksi kalori dilakukan berdasarkan hasil deteksi yang didapat. Analisa pengujian hasil prediksi kalori dilakukan dengan melakukan perbandingan data prediksi kalori dari sistem yang digunakan dengan data kalori treadmill sebagai data sebenarnya atau data \emph{actual score}. Perbandingan hasil prediksi kalori dengan kalori treadmill didapatkan data eror yang kemudian dilakukan perhitungan persentase berdasarkan data sebenarnya pada data kalori treadmill. Proses analisa pengujian dilakukan terhadap seluruh data video pada data percobaan. Hasil analisa yang didapat dengan melakukan analisa pengujian prediksi kalori didapatkan hasil akurasi rata-rata sebesar 57,49\% dengan hasil eror rata-rata sebesar 42,51\%. Tabel \ref{tb:AnalisaPrediksiPerhitungan} menunjukkan hasil analisa pengujian dari prediksi kalori dengan perhitungan rumus.

\begin{longtable}{|c|c|c|c|c|c|}
  \caption{Analisa Pengujian Prediksi dengan Perhitungan Rumus}
  \label{tb:AnalisaPrediksiPerhitungan}                                   \\
  \hline
  \rowcolor[HTML]{C0C0C0}
  \textbf{Percobaan} & \textbf{Kalori Treadmill} & \textbf{Prediksi Kalori} & \textbf{Error} & \textbf{Error\%} & \textbf{Akurasi\%} \\
  \hline
  1   & 10   & 6,788   & 3,212    & 32,12\%     & 67,88\%   \\
  \hline
  2   & 10   & 4,744   & 5,251    & 52,56\%     & 47,44\%   \\
  \hline
  3   & 20   & 22,462  & 2,462    & 12,31\%     & 87,69\%   \\
  \hline
  4   & 20   & 20,575  & 0,575    & 2,86\%      & 91,14\%   \\
  \hline
  5   & 20   & 22,126  & 2,126    & 10,63\%     & 89,37\%   \\
  \hline
  6   & 20   & 19,311  & 0,669    & 3,35\%      & 96,65\%   \\
  \hline
  7   & 10   & 6,979   & 3,021    & 30,21\%     & 69,79\%   \\
  \hline
  8   & 10   & 3,296   & 6,704    & 67,04\%     & 32,96\%   \\
  \hline
  9   & 10   & 3,819   & 6,181    & 61,81\%     & 38,19\%   \\
  \hline
  10   & 20   & 30,862  & 10,862   & 54,31\%     & 45,69\%   \\
  \hline
  11   & 20   & 22,224  & 2,224    & 11,12\%     & 88,88\%   \\
  \hline
  12   & 20   & 3,228   & 16,772   & 83,86\%     & 16,14\%   \\
  \hline
  13   & 20   & 15,900  & 4,100    & 20,50\%     & 79,50\%   \\
  \hline
  14   & 20   & 19,470  & 0,53     & 2,65\%      & 97,35\%   \\
  \hline
  15   & 20   & 11,057  & 8,943    & 44,72\%     & 55,28\%   \\
  \hline
  16   & 5    & 0,643   & 4,357    & 87,14\%     & 12,86\%   \\
  \hline
  17   & 10   & 10,427  & 0,427    & 4,27\%      & 95,73\%   \\
  \hline

  \multicolumn{4}{|c|}{\textbf{Rata-rata}} & 42,51\% & 57,49\% \\
  \hline
\end{longtable}

Pengujian yang telah dilakukan berdasarkan metode yang digunakan yaitu regresi linear dan perhitungan rumus mendapatkan hasil performa dengan nilai akurasi rata-rata dan eror rata-rata. Hasil yang diperoleh melalui model yang telah dibuat untuk deteksi dan melakukan prediksi sesuai metode yang dilakukan didapatkan hasil akumulasi kalori dengan prediksi regresi sebesar 80,93\% dengan akumulasi error sebesar 19,07\%. Kemudian prediksi dengan perhitungan rumus didapatkan hasil akumulasi akurasi kalori sebesar 57,49\% dengan akumulasi error sebesar 42,51\%. Tabel \ref{tb:PengujianPrediksi} merupakan hasil perbandingan antara dataset percobaan dengan nilai kalori pembanding dataset dengan proses prediksi.

\begin{longtable}{|c|c|c|c|c|}
  \caption{Hasil Perbandingan Pengujian Prediksi Kalori}
  \label{tb:PengujianPrediksi}                                   \\
  \hline
  \rowcolor[HTML]{C0C0C0}
  \textbf{Percobaan} & \textbf{Kalori Treadmill} & \textbf{Prediksi Regresi} & \textbf{Perhitungan MET} \\
  \hline
  1   & 10    & 11,652    & 6,788   \\
  \hline
  2   & 10    & 8,245     & 4,744   \\
  \hline
  3   & 20    & 20,534    & 22,462   \\
  \hline
  4   & 20    & 19,927    & 20,575   \\
  \hline
  5   & 20    & 20,14     & 22,126   \\
  \hline
  6   & 20    & 20,056    & 19,331   \\
  \hline
  7   & 10    & 12,267    & 6,979  \\
  \hline
  8   & 10    & 12,349    & 3,296  \\
  \hline
  9   & 10    & 12,397    & 3,819  \\
  \hline
  10   & 20   & 35,488    & 30,862  \\
  \hline
  11   & 20   & 21,909    & 22,224  \\
  \hline
  12   & 20   & 10,865    & 3,228  \\
  \hline
  13   & 20   & 15,119    & 15,900  \\
  \hline
  14   & 20   & 19,647    & 19,470  \\
  \hline
  15   & 20   & 10,478    & 11,057  \\
  \hline
  16   & 5    & 5,435     & 0,643  \\
  \hline
  17   & 20   & 9,922     & 10,427  \\
  \hline

  \multicolumn{2}{|c|}{\textbf{Akurasi rata-rata}} & 80,93\% & 57,49\% \\
  \hline
\end{longtable}


\section{Pengujian Performa Berdasarkan Jarak Kamera}
\label{sec:PengujianJarak}

Pada pengujian ini dilakukan deteksi dan prediksi kalori dengan menggunakan skenario berdasarkan jarak kamera terhdap objek deteksi yang berbeda-beda. Skenario pertama yang dilakukan adalah pengujian pada jarak kamera dekat, jarak kamera yang digunakan terhadap objek yang akan dilakukan pengujian dengan jarak 120 cm yang hanya dapat mencakup untuk melihat bagian kaki yang akan dilakukan deteksi. Skenario kedua yang dilakukan adalah pengujian pada jarak kamera jauh, jarak kamera yang digunakan terhadp objek yang akan dilakukan pengujian dengan jarak 320 cm yang terlihat objek cukup jauh dan kecil dari pandangan kamera yang akan diujikan untuk dilakukan deteksi. Video pengujian yang digunakan dengan banyak percobaan sebanyak 9 kali dengan menggunakan variasi kecepatan dan hasil kalori yang didapatkan. Variasi kecepatan yang digunakan adalah 3 km/jam, 9 km/jam dan 12 km/jam. Kemudian variasi hasil kalori yang didapatkan adalah 10 dan 20 kcal. 

\subsection{Pengujian Pada Jarak Kamera Dekat}
\label{subsec:PengujianJarakDekat}

Pengujian dilakukan dengan melakukan pengujian performa sistem pada jarak kamera dekat. Jarak kamera yang digunakan pada pengujian ini dengan jarak 120 cm terhadap objek yang dideteksi. Contoh cuplikan gambar yang memperlihatkan kondisi jarak kamera dekat dengan jaarak 120 cm dapat dilihat pada Gambar \ref{fig:PengujianJarakDekat}. Data percobaan yang digunakan dalam pengujian ini dilakukan proses deteksi dengan sistem yang dibuat untuk dapat melakukan proses estimasi dan deteksi langkah maupun waktu. Proses deteksi yang dilakukan pada data percobaan dapat dilihaat pada Gambar \ref{fig:PengujianJarakDekat2}. Pengujian dilanjutkan dengan melakukan proses prediksi dari hasil deteksi.

\begin{figure}[H]
  \centering
  \includegraphics[scale=0.42]{gambar/jarak_dekat.png}
  \caption{Pengujian pada jarak 120 cm}
  \label{fig:PengujianJarakDekat}
\end{figure}

\begin{figure}[H]
  \centering
  \includegraphics[scale=0.42]{gambar/jarak_dekat2.png}
  \caption{Hasil deteksi pengujian pada jarak 120 cm}
  \label{fig:PengujianJarakDekat2}
\end{figure}

Pengujian dilakukan setelah didapatkan data hasil deteksi yang telah didapat dan juga data sebenarnya berdasarkan hasil perhitungan yang dilakukan terhadap data video. Dengan data video sebagai banyaknya percobaan dilakukan analisa terkait perbandingan hasil yang didapat antara data hasil deteksi dan data perhitungan sebenarnya. Data langkah merupakan data sebagai data sebenarnya atau \emph{actual score} dengan melakukan perhitungan pada data video dan data deteksi langkah sebagai data hasil deteksi langkah yang didapatkan berdasarkan proses deteksi pada sistem. Perbandingan hasil deteksi dengan data sebenarnya dicantumkan pada data eror dan dilakukan persentase berdasarkan data sebenarnya. Hasil persentase eror mendapatkan hasil persentase akurasi yang juga didapatkan berdasarkan persentase eror. Hasil analisa yang didapat dengan melakukan analisa pengujian hasil deteksi didapatkan hasil akurasi rata-rata sebesar 72,37\% dengan hasil eror rata-rata sebesar 27,63\%. Tabel \ref{tb:PengujianJarakDekatAnalisaDeteksi} menunjukkan hasil pengujian dari setiap percobaan yang dilakukan analisa pengujian terhadap hasil deteksi.

\begin{longtable}{|c|c|c|c|c|c|}
  \caption{Hasil Deteksi Pengujian Pada Jarak Dekat}
  \label{tb:PengujianJarakDekatAnalisaDeteksi}                                   \\
  \hline
  \rowcolor[HTML]{C0C0C0}
  \textbf{Percobaan} & \textbf{Langkah} & \textbf{Deteksi Langkah} & \textbf{Eror} & \textbf{Eror\%} & \textbf{Akurasi\%} \\
  \hline
  1   & 255   & 290  & 35  & 13,73\%    & 86,27\%   \\
  \hline
  2   & 254   & 290  & 36  & 14,17\%    & 85,83\%   \\
  \hline
  3   & 265   & 264  & 1   & 0,38\%     & 99,62\%   \\
  \hline
  4   & 327   & 338  & 11  & 3,36\%     & 96,64\%   \\
  \hline
  5   & 311   & 168  & 143 & 45,98\%    & 54,02\%   \\
  \hline
  6   & 322   & 174  & 148 & 45,96\%    & 54,04\%   \\
  \hline
  7   & 271   & 231  & 40  & 14,76\%    & 85,24\%   \\
  \hline
  8   & 254   & 100  & 154 & 60,63\%    & 39,37\%   \\
  \hline
  9   & 292   & 147  & 145 & 49,66\%    & 50,34\%   \\
  \hline

  \multicolumn{4}{|c|}{\textbf{Rata-rata}} & 27,63\% & 72,37\% \\
  \hline
\end{longtable}

Pada pengujian prediksi kalori dilakukan berdasarkan hasil deteksi yang didapat. Hasil deteksi berupa banyaknya langkah, jarak dan waktu tempuh dari data video yang digunakan. Data hasil deteksi akan diilakukan untuk proses prediksi kalori dengan menggunakan model regresi linear yang sudah didapatkan dan menghasilkan data pada prediksi kalori. Analisa pengujian hasil prediksi kalori dilakukan dengan melakukan perbandingan data prediksi kalori dari sistem yang digunakan dengan data kalori treadmill sebagai data sebenarnya atau data \emph{actual score}. Perbandingan hasil prediksi kalori dengan kalori treadmill didapatkan data eror yang kemudian dilakukan perhitungan persentase berdasarkan data sebenarnya pada data kalori treadmill. Proses analisa pengujian dilakukan terhadap seluruh data video pada data percobaan. Hasil analisa yang didapat dengan melakukan analisa pengujian prediksi kalori didapatkan hasil akurasi rata-rata sebesar 49,53\% dengan hasil eror rata-rata sebesar 50,47\%. Tabel \ref{tb:PengujianJarakDekatAnalisaPrediksiRegresi} menunjukkan hasil analisa pengujian dari setiap percobaan yang dilakukan analisa prediksi kalori dengan regresi linear.

\begin{longtable}{|c|c|c|c|c|c|c|c|}
  \caption{Hasil Prediksi Pengujian Pada Jarak Dekat dengan Regresi Linear}
  \label{tb:PengujianJarakDekatAnalisaPrediksiRegresi}                                   \\
  \hline
  \rowcolor[HTML]{C0C0C0}
  & & & & \textbf{Kalori} & \textbf{Prediksi} & & \\
  \rowcolor[HTML]{C0C0C0}
  \multirow{-2}{*}{\textbf{Percobaan}} & \multirow{-2}{*}{\textbf{Langkah}} & \multirow{-2}{*}{\textbf{Jarak}} & \multirow{-2}{*}{\textbf{Waktu}} & \textbf{Treadmill} & \textbf{Kalori} & \multirow{-2}{*}{\textbf{Eror\%}} & \multirow{-2}{*}{\textbf{Akurasi\%}} \\
  
  \hline
  1   & 290   & 163,100    & 2:48    & 10    & 11,364   & 13,64\%      & 86,36\%   \\
  \hline  
  2   & 290   & 163,100    & 1:48    & 10    & 11,364   & 13,64\%      & 86,36\%  \\
  \hline
  3   & 264   & 179,273    & 2:50    & 10    & 12,502   & 25,02\%      & 74,98\%   \\
  \hline
  4   & 338   & 528,603    & 2:03    & 20    & 37,109   & 85,54\%      & 14,46\%  \\
  \hline
  5   & 168   & 106,133    & 2:01    & 20    & 7,368    & 63,16\%      & 36,84\%    \\
  \hline
  6   & 174   & 110,560    & 2:01    & 20    & 7,680    & 61,60\%      & 38,40\%   \\
  \hline
  7   & 231   & 176,116    & 1:39    & 20    & 12,301   & 38,49\%      & 61,51\%   \\
  \hline
  8   & 100   & 46,154     & 1:37    & 20    & 3,315    & 84,23\%      & 15,77\%   \\
  \hline
  9   & 147   & 89,715     & 1:45    & 20    & 6,217    & 68,91\%      & 31,09\%   \\
  \hline

  \multicolumn{6}{|c|}{\textbf{Rata-rata}} & 50,47\% & 49,53\%  \\
  \hline
\end{longtable}

Pada pengujian prediksi kalori dilakukan berdasarkan hasil deteksi yang didapat. Hasil deteksi berupa kecepatan, MET dan waktu tempuh dari data video yang digunakan. Data hasil deteksi akan diilakukan untuk proses prediksi kalori dengan menggunakan perhitungan rumus berdasarkan MET yang sudah didapatkan dan menghasilkan data pada prediksi kalori. Analisa pengujian hasil prediksi kalori dilakukan dengan melakukan perbandingan data prediksi kalori dari sistem yang digunakan dengan data kalori treadmill sebagai data sebenarnya atau data \emph{actual score}. Perbandingan hasil prediksi kalori dengan kalori treadmill didapatkan data eror yang kemudian dilakukan perhitungan persentase berdasarkan data sebenarnya pada data kalori treadmill. Proses analisa pengujian dilakukan terhadap seluruh data video pada data percobaan. Hasil analisa yang didapat dengan melakukan analisa pengujian prediksi kalori didapatkan hasil akurasi rata-rata sebesar 51,87\% dengan hasil eror rata-rata sebesar 48,13\%. Tabel \ref{tb:PengujianJarakDekatAnalisaPrediksiPerhitungan} menunjukkan hasil analisa pengujian dari prediksi kalori dengan perhitungan rumus.

\begin{longtable}{|c|c|c|c|c|c|c|c|}
  \caption{Hasil Prediksi Pengujian Pada Jarak Dekat dengan Perhitungan Rumus}
  \label{tb:PengujianJarakDekatAnalisaPrediksiPerhitungan}                                   \\
  \hline
  \rowcolor[HTML]{C0C0C0}
  & & & & \textbf{Kalori} & \textbf{Prediksi} & & \\
  \rowcolor[HTML]{C0C0C0}
  \multirow{-2}{*}{\textbf{Percobaan}} & \multirow{-2}{*}{\textbf{Kecepatan}} & \multirow{-2}{*}{\textbf{MET}} & \multirow{-2}{*}{\textbf{Waktu}} & \textbf{Treadmill} & \textbf{Kalori} & \multirow{-2}{*}{\textbf{Eror\%}} & \multirow{-2}{*}{\textbf{Akurasi\%}} \\
  \hline
  1   & 3,065   & 4,400    & 2:48    & 10   & 14,014   & 40,14\%     & 67,86\%   \\
  \hline
  2   & 3,065   & 4,400    & 2:48    & 10   & 14,014   & 40,14\%     & 59,86\%   \\
  \hline
  3   & 3,798   & 1,275    & 2:50    & 10   & 4,109    & 58,91\%     & 41,09\%   \\
  \hline
  4   & 14,777  & 13,172   & 2:03    & 20   & 30,715   & 53,58\%      & 46,42\%   \\
  \hline
  5   & 3,115   & 4,170    & 2:01    & 20   & 9,566    & 52,17\%     & 47,83\%   \\
  \hline
  6   & 3,208   & 3,744    & 2:01    & 20   & 8,589    & 57,05\%      & 42,95\%   \\
  \hline
  7   & 6,600   & 6,337    & 1:39    & 20    & 11,893  & 40,53\%      & 59,47\%   \\
  \hline
  8   & 1,195   & 15,325   & 1:37    & 20    & 28,182  & 40,91\%      & 59,09\%   \\
  \hline
  9   & 2,928   & 5,051    & 1:45    & 20    & 10,053  & 49,73\%      & 50,27\%   \\
  \hline

  \multicolumn{6}{|c|}{\textbf{Rata-rata}} & 48,13\% & 51,87\%  \\
  \hline
\end{longtable}


\subsection{Pengujian Pada Jarak Jauh}
\label{subsec:PengujianJarakJauh}

Pengujian dilakukan dengan melakukan pengujian performa sistem pada jarak kamera dekat. Jarak kamera yang digunakan pada pengujian ini dengan jarak 320 cm terhadap objek yang dideteksi. Contoh cuplikan gambar yang memperlihatkan kondisi jarak kamera dekat dengan jaarak 320 cm dapat dilihat pada Gambar \ref{fig:PengujianJarakJauh}. Data percobaan yang digunakan dalam pengujian ini dilakukan proses deteksi dengan sistem yang dibuat untuk dapat melakukan proses estimasi dan deteksi langkah maupun waktu. Proses deteksi yang dilakukan pada data percobaan dapat dilihaat pada Gambar \ref{fig:PengujianJarakJauh2}. Pengujian dilanjutkan dengan melakukan proses prediksi dari hasil deteksi.

\begin{figure}[H]
  \centering
  \includegraphics[scale=0.5]{gambar/jarak_jauh.png}
  \caption{Pengujian pada jarak 320 cm}
  \label{fig:PengujianJarakJauh}
\end{figure}

\begin{figure}[H]
  \centering
  \includegraphics[scale=0.5]{gambar/jarak_jauh2.png}
  \caption{Hasil deteksi pengujian pada jarak 320 cm}
  \label{fig:PengujianJarakJauh2}
\end{figure}

Pengujian dilakukan setelah didapatkan data hasil deteksi yang telah didapat dan juga data sebenarnya berdasarkan hasil perhitungan yang dilakukan terhadap data video. Dengan data video sebagai banyaknya percobaan dilakukan analisa terkait perbandingan hasil yang didapat antara data hasil deteksi dan data perhitungan sebenarnya. Data langkah merupakan data sebagai data sebenarnya atau \emph{actual score} dengan melakukan perhitungan pada data video dan data deteksi langkah sebagai data hasil deteksi langkah yang didapatkan berdasarkan proses deteksi pada sistem. Perbandingan hasil deteksi dengan data sebenarnya dicantumkan pada data eror dan dilakukan persentase berdasarkan data sebenarnya. Hasil persentase eror mendapatkan hasil persentase akurasi yang juga didapatkan berdasarkan persentase eror. Hasil analisa yang didapat dengan melakukan analisa pengujian hasil deteksi didapatkan hasil akurasi rata-rata sebesar 82,59\% dengan hasil eror rata-rata sebesar 17,41\%. Tabel \ref{tb:PengujianJarakJauhAnalisaDeteksi} menunjukkan hasil pengujian dari setiap percobaan yang dilakukan analisa pengujian terhadap hasil deteksi.

\begin{longtable}{|c|c|c|c|c|c|}
  \caption{Hasil Deteksi Pengujian Pada Jarak Jauh}
  \label{tb:PengujianJarakJauhAnalisaDeteksi}                                   \\
  \hline
  \rowcolor[HTML]{C0C0C0}
  \textbf{Percobaan} & \textbf{Langkah} & \textbf{Deteksi Langkah} & \textbf{Eror} & \textbf{Eror\%} & \textbf{Akurasi\%} \\
  \hline
  1   & 255   & 336 & 81   & 31,76\%    & 68,24\%   \\
  \hline
  2   & 254   & 268 & 14   & 5,51\%     & 94,49\%   \\
  \hline
  3   & 265   & 269 & 4    & 1,51\%     & 98,49\%     \\
  \hline
  4   & 327   & 284 & 43   & 13,15\%    & 86,85\%   \\
  \hline
  5   & 311   & 302 & 9    & 2,89\%     & 97,11\%   \\
  \hline
  6   & 322   & 204 & 118  & 36,65\%    & 63,35\%   \\
  \hline
  7   & 271   & 224 & 47   & 17,34\%    & 82,66\%   \\
  \hline
  8   & 254   & 236 & 18   & 7,09\%     & 92,91\%   \\
  \hline
  9   & 292   & 173 & 119  & 40,75\%   & 59,25\%   \\
  \hline

  \multicolumn{4}{|c|}{\textbf{Rata-rata}} & 17,41\% & 82,59\% \\
  \hline
\end{longtable}

Pada pengujian prediksi kalori dilakukan berdasarkan hasil deteksi yang didapat. Hasil deteksi berupa banyaknya langkah, jarak dan waktu tempuh dari data video yang digunakan. Data hasil deteksi akan diilakukan untuk proses prediksi kalori dengan menggunakan model regresi linear yang sudah didapatkan dan menghasilkan data pada prediksi kalori. Analisa pengujian hasil prediksi kalori dilakukan dengan melakukan perbandingan data prediksi kalori dari sistem yang digunakan dengan data kalori treadmill sebagai data sebenarnya atau data \emph{actual score}. Perbandingan hasil prediksi kalori dengan kalori treadmill didapatkan data eror yang kemudian dilakukan perhitungan persentase berdasarkan data sebenarnya pada data kalori treadmill. Proses analisa pengujian dilakukan terhadap seluruh data video pada data percobaan. Hasil analisa yang didapat dengan melakukan analisa pengujian prediksi kalori didapatkan hasil akurasi rata-rata sebesar 63,80\% dengan hasil eror rata-rata sebesar 36,20\%. Tabel \ref{tb:PengujianJarakJauhAnalisaPrediksiRegresi} menunjukkan hasil analisa pengujian dari setiap percobaan yang dilakukan analisa prediksi kalori dengan regresi linear.

\begin{longtable}{|c|c|c|c|c|c|c|c|}
  \caption{Hasil Prediksi Pengujian Pada Jarak Jauh dengan Regresi Linear}
  \label{tb:PengujianJarakJauhAnalisaPrediksiRegresi}                                   \\
  \hline
  \rowcolor[HTML]{C0C0C0}
  & & & & \textbf{Kalori} & \textbf{Prediksi} & & \\
  \rowcolor[HTML]{C0C0C0}
  \multirow{-2}{*}{\textbf{Percobaan}} & \multirow{-2}{*}{\textbf{Langkah}} & \multirow{-2}{*}{\textbf{Jarak}} & \multirow{-2}{*}{\textbf{Waktu}} & \textbf{Treadmill} & \textbf{Kalori} & \multirow{-2}{*}{\textbf{Eror\%}} & \multirow{-2}{*}{\textbf{Akurasi\%}} \\
  
  \hline
  1   & 336   & 97,956     & 2:48    & 10    & 6,778    & 32,22\%      & 67,78\%   \\
  \hline  
  2   & 268   & 175,943    & 2:48    & 10    & 12,268   & 22,68\%      & 77,32\%  \\
  \hline
  3   & 269   & 178,949    & 2:50    & 10    & 12,479   & 24,79\%      & 75,21\%   \\
  \hline
  4   & 284   & 166,973    & 2:03    & 20    & 11,650   & 41,75\%      & 58,25\%  \\
  \hline
  5   & 302   & 252,115    & 2:01    & 20    & 17,645   & 11,77\%      & 88,23\%    \\
  \hline
  6   & 204   & 123,654    & 2:01    & 20    & 8,601    & 56,99\%      & 43,01\%   \\
  \hline
  7   & 224   & 160,596    & 1:39    & 20    & 11,209   & 43,95\%      & 56,05\%   \\
  \hline
  8   & 236   & 203,020    & 1:37    & 20    & 14,196   & 29,02\%      & 70,98\%   \\
  \hline
  9   & 173   & 107,558    & 1:45    & 20    & 7,473    & 62,63\%      & 37,37\%   \\
  \hline

  \multicolumn{6}{|c|}{\textbf{Rata-rata}} & 36,20\% & 63,80\% \\
  \hline
\end{longtable}

Pada pengujian prediksi kalori dilakukan berdasarkan hasil deteksi yang didapat. Hasil deteksi berupa kecepatan, MET dan waktu tempuh dari data video yang digunakan. Data hasil deteksi akan diilakukan untuk proses prediksi kalori dengan menggunakan perhitungan rumus berdasarkan MET yang sudah didapatkan dan menghasilkan data pada prediksi kalori. Analisa pengujian hasil prediksi kalori dilakukan dengan melakukan perbandingan data prediksi kalori dari sistem yang digunakan dengan data kalori treadmill sebagai data sebenarnya atau data \emph{actual score}. Perbandingan hasil prediksi kalori dengan kalori treadmill didapatkan data eror yang kemudian dilakukan perhitungan persentase berdasarkan data sebenarnya pada data kalori treadmill. Proses analisa pengujian dilakukan terhadap seluruh data video pada data percobaan. Hasil analisa yang didapat dengan melakukan analisa pengujian prediksi kalori didapatkan hasil akurasi rata-rata sebesar 47,78\% dengan hasil eror rata-rata sebesar 52,22\%. Tabel \ref{tb:PengujianJarakJauhAnalisaPrediksiPerhitungan} menunjukkan hasil analisa pengujian dari prediksi kalori dengan perhitungan rumus.

\begin{longtable}{|c|c|c|c|c|c|c|c|}
  \caption{Hasil Prediksi Pengujian Pada Jarak Jauh dengan Perhitungan Rumus}
  \label{tb:PengujianJarakJauhAnalisaPrediksiPerhitungan}                                   \\
  \hline
  \rowcolor[HTML]{C0C0C0}
  & & & & \textbf{Kalori} & \textbf{Prediksi} & & \\
  \rowcolor[HTML]{C0C0C0}
  \multirow{-2}{*}{\textbf{Percobaan}} & \multirow{-2}{*}{\textbf{Kecepatan}} & \multirow{-2}{*}{\textbf{MET}} & \multirow{-2}{*}{\textbf{Waktu}} & \textbf{Treadmill} & \textbf{Kalori} & \multirow{-2}{*}{\textbf{Eror\%}} & \multirow{-2}{*}{\textbf{Akurasi\%}} \\
  \hline
  1   & 1,387   & 5,740    & 2:48    & 10   & 18,281   & 82,81\%     & 17,19\%   \\
  \hline
  2   & 3,657   & 1,835    & 2:48    & 10   & 5,844    & 41,56\%     & 58,44\%   \\
  \hline
  3   & 3,717   & 1,593    & 2:50    & 10   & 5,135    & 48,65\%     & 51,35\%   \\
  \hline
  4   & 4,906   & 2,454    & 2:03    & 20   & 5,722    & 71,39\%     & 28,61\%   \\
  \hline
  5   & 7,666   & 7,946    & 2:01    & 20   & 18,227   & 8,86\%      & 91,14\%   \\
  \hline
  6   & 3,371   & 3,027    & 2:01    & 20   & 6,944    & 65,28\%     & 34,72\%   \\
  \hline
  7   & 5,978   & 5,128    & 1:39    & 20   & 9,625    & 51,88\%     & 48,12\%   \\
  \hline
  8   & 7,786   & 8,097    & 1:37    & 20   & 14,890   & 25,55\%     & 74,45\%   \\
  \hline
  9   & 3,469   & 2,611    & 1:45    & 20   & 5,197    & 74,02\%     & 25,98\%   \\
  \hline

  \multicolumn{6}{|c|}{\textbf{Rata-rata}} & 52,22\% & 47,78\%  \\
  \hline
\end{longtable}

Berdasarkan pengujian yang telah dilakukan terhadap hasil performa deteksi langkah, prediksi kalori dengan regresi dan prediksi kalori dengan perhitungan rumus didapatkan hasil performa dalam nilai akurasi rata-rata dan eror rata-rata dari setiap pengujian. Pada pengujian deteksi langkah didapatkan akurasi rata-rata lebih baik pada jarak kamera 320 cm dengan nilai akurasi rata-rata sebesar 82,59\% sedangkan pada jarak kamera 120 cm didapat nilai akurasi rata-rata sebesar 72,37\%. Pada pengujian prediksi kalori dengan regresi didapatkan akurasi rata-rata lebih baik pada jarak kamera 320 cm dengan nilai akurasi rata-rata sebesar 63,80\% sedangkan pada jarak kamera 120 cm didapat nilai akurasi rata-rata sebesar 49,53\%. Pada pengujian prediksi kalori dengan perhitungan rumus didapatkan akurasi rata-rata lebih baik pada jarak kamera 120 cm dengan nilai akurasi rata-rata sebesar 51,87\% sedangkan pada jarak kamera 320 cm didapat nilai akurasi rata-rata sebesar 47,78\%. Gambar \ref{fig:DiagramJarak} menunjukkan diagram perbandingan performa pengujian dengan nilai akurasi rata-rata berdasarkan jarak kamera.

\begin{figure}[H]
  \centering
  \includegraphics[scale=0.7]{gambar/diagram_jarak.png}
  \caption{Diagram perbandingan performa berdasarkan jarak kamera}
  \label{fig:DiagramJarak}
\end{figure}

\section{Pengujian Performa Berdasarkan Posisi Kamera}
\label{sec:PengujianPosisi}

Pada pengujian ini dilakukan deteksi dan prediksi kalori dengan menggunakan skenario berdasarkan posisi kamera terhdap objek deteksi yang berbeda-beda. Skenario pertama yang dilakukan adalah pengujian pada posisi kamera serong, posisi kamera berada di arah samping yang sedikit di arah depan dari objek yang terlihat dengan sudut 45 derajat dari samping objek sehingga terlihat sebagian wajah dari objek yang akan dilakukan deteksi. Skenario kedua yang dilakukan adalah pengujian pada posisi kamera belakang, posisi kamera berada di arah belakang dari objek sehingga terlihat bagian belakang objek yang akan dilakukan deteksi. Video pengujian yang digunakan dengan banyak percobaan sebanyak 9 kali dengan menggunakan variasi kecepatan dan hasil kalori yang didapatkan. Variasi kecepatan yang digunakan adalah 3 km/jam, 9 km/jam dan 12 km/jam. Kemudian variasi hasil kalori yang didapatkan adalah 10 dan 20 kcal. 


\subsection{Pengujian Pada Posisi Serong}
\label{subsec:PengujianPosisiSerong}

Pengujian dilakukan dengan melakukan pengujian performa sistem pada posisi kamera serong. Posisi kamera yang digunakan pada pengujian ini dengan posisi serong dengan sudut 45 derajat dari samping objek yang dilakukan deteksi. Contoh cuplikan gambar yang memperlihatkan kondisi posisi kamera serong dapat dilihat pada Gambar \ref{fig:PengujianPosisiSerong}. Data percobaan yang digunakan dalam pengujian ini dilakukan proses deteksi dengan sistem yang dibuat untuk dapat melakukan proses estimasi dan deteksi langkah maupun waktu. Proses deteksi yang dilakukan pada data percobaan dapat dilihaat pada Gambar \ref{fig:PengujianPosisiSerong2}. Pengujian dilanjutkan dengan melakukan proses prediksi dari hasil deteksi.

\begin{figure}[H]
  \centering
  \includegraphics[scale=0.5]{gambar/posisi_serong.png}
  \caption{Pengujian pada posisi serong}
  \label{fig:PengujianPosisiSerong}
\end{figure}

\begin{figure}[H]
  \centering
  \includegraphics[scale=0.5]{gambar/posisi_serong2.png}
  \caption{Hasil deteksi pengujian pada posisi serong}
  \label{fig:PengujianPosisiSerong2}
\end{figure}

Pengujian dilakukan setelah didapatkan data hasil deteksi yang telah didapat dan juga data sebenarnya berdasarkan hasil perhitungan yang dilakukan terhadap data video. Dengan data video sebagai banyaknya percobaan dilakukan analisa terkait perbandingan hasil yang didapat antara data hasil deteksi dan data perhitungan sebenarnya. Data langkah merupakan data sebagai data sebenarnya atau \emph{actual score} dengan melakukan perhitungan pada data video dan data deteksi langkah sebagai data hasil deteksi langkah yang didapatkan berdasarkan proses deteksi pada sistem. Perbandingan hasil deteksi dengan data sebenarnya dicantumkan pada data eror dan dilakukan persentase berdasarkan data sebenarnya. Hasil persentase eror mendapatkan hasil persentase akurasi yang juga didapatkan berdasarkan persentase eror. Hasil analisa yang didapat dengan melakukan analisa pengujian hasil deteksi didapatkan hasil akurasi rata-rata sebesar 86,80\% dengan hasil eror rata-rata sebesar 13,20\%. Tabel \ref{tb:PengujianPosisiSerongAnalisaDeteksi} menunjukkan hasil pengujian dari setiap percobaan yang dilakukan analisa pengujian terhadap hasil deteksi.

\begin{longtable}{|c|c|c|c|c|c|}
  \caption{Hasil Deteksi Pengujian Pada Posisi Serong}
  \label{tb:PengujianPosisiSerongAnalisaDeteksi}                                   \\
  \hline
  \rowcolor[HTML]{C0C0C0}
  \textbf{Percobaan} & \textbf{Langkah} & \textbf{Deteksi Langkah} & \textbf{Eror} & \textbf{Eror\%} & \textbf{Akurasi\%} \\
  \hline
  1   & 226   & 221  & 5   & 2,21\%    & 97,79\%   \\
  \hline
  2   & 236   & 207  & 29  & 12,29\%   & 87,71\%   \\
  \hline
  3   & 241   & 213  & 28  & 11,62\%   & 88,38\%   \\
  \hline
  4   & 371   & 357  & 14  & 3,77\%    & 96,23\%   \\
  \hline
  5   & 350   & 334  & 16  & 4,57\%    & 95,43\%   \\
  \hline
  6   & 359   & 291  & 68  & 18,94\%   & 81,06\%   \\
  \hline
  7   & 281   & 267  & 14  & 4,98\%    & 95,02\%   \\
  \hline
  8   & 267   & 192  & 75  & 28,09\%   & 71,91\%   \\
  \hline
  9   & 294   & 199  & 95  & 32,31\%   & 67,69\%   \\
  \hline

  \multicolumn{4}{|c|}{\textbf{Rata-rata}} & 13,20\% & 86,80\% \\
  \hline
\end{longtable}

Pada pengujian prediksi kalori dilakukan berdasarkan hasil deteksi yang didapat. Hasil deteksi berupa banyaknya langkah, jarak dan waktu tempuh dari data video yang digunakan. Data hasil deteksi akan diilakukan untuk proses prediksi kalori dengan menggunakan model regresi linear yang sudah didapatkan dan menghasilkan data pada prediksi kalori. Analisa pengujian hasil prediksi kalori dilakukan dengan melakukan perbandingan data prediksi kalori dari sistem yang digunakan dengan data kalori treadmill sebagai data sebenarnya atau data \emph{actual score}. Perbandingan hasil prediksi kalori dengan kalori treadmill didapatkan data eror yang kemudian dilakukan perhitungan persentase berdasarkan data sebenarnya pada data kalori treadmill. Proses analisa pengujian dilakukan terhadap seluruh data video pada data percobaan. Hasil analisa yang didapat dengan melakukan analisa pengujian prediksi kalori didapatkan hasil akurasi rata-rata sebesar 70,18\% dengan hasil eror rata-rata sebesar 29,82\%. Tabel \ref{tb:PengujianPosisiSerongAnalisaPrediksiRegresi} menunjukkan hasil analisa pengujian dari setiap percobaan yang dilakukan analisa prediksi kalori dengan regresi linear.

\begin{longtable}{|c|c|c|c|c|c|c|c|}
  \caption{Hasil Prediksi Pengujian Pada Posisi Serong dengan Regresi Linear}
  \label{tb:PengujianPosisiSerongAnalisaPrediksiRegresi}                                   \\
  \hline
  \rowcolor[HTML]{C0C0C0}
  & & & & \textbf{Kalori} & \textbf{Prediksi} & & \\
  \rowcolor[HTML]{C0C0C0}
  \multirow{-2}{*}{\textbf{Percobaan}} & \multirow{-2}{*}{\textbf{Langkah}} & \multirow{-2}{*}{\textbf{Jarak}} & \multirow{-2}{*}{\textbf{Waktu}} & \textbf{Treadmill} & \textbf{Kalori} & \multirow{-2}{*}{\textbf{Eror\%}} & \multirow{-2}{*}{\textbf{Akurasi\%}} \\
  
  \hline
  1   & 221   & 136,365    & 2:15    & 10    & 9,492    & 5,08\%      & 94,92\%   \\
  \hline  
  2   & 207   & 132,305    & 2:13    & 10    & 9,207    & 7,93\%      & 92,07\%  \\
  \hline
  3   & 213   & 133,706    & 2:16    & 10    & 9,305    & 6,95\%      & 93,05\%   \\
  \hline
  4   & 357   & 579,448    & 2:10    & 20    & 17,249   & 13,75\%     & 86,25\%  \\
  \hline
  5   & 334   & 324,206    & 2:10    & 20    & 10,165   & 49,17\%     & 50,83\%    \\
  \hline
  6   & 291   & 143,608    & 2:11    & 20    & 8,154    & 59,23\%     & 40,77\%   \\
  \hline
  7   & 267   & 259,668    & 1:45    & 20    & 18,182   & 9,09\%      & 90,91\%   \\
  \hline
  8   & 192   & 117,049    & 1:46    & 20    & 11,859   & 59,29\%     & 40,71\%   \\
  \hline
  9   & 199   & 120,996    & 1:45    & 20    & 11,581   & 57,90\%     & 42,10\%   \\
  \hline

  \multicolumn{6}{|c|}{\textbf{Rata-rata}} & 29,82\% & 70,18\%  \\
  \hline
\end{longtable}

Pada pengujian prediksi kalori dilakukan berdasarkan hasil deteksi yang didapat. Hasil deteksi berupa kecepatan, MET dan waktu tempuh dari data video yang digunakan. Data hasil deteksi akan diilakukan untuk proses prediksi kalori dengan menggunakan perhitungan rumus berdasarkan MET yang sudah didapatkan dan menghasilkan data pada prediksi kalori. Analisa pengujian hasil prediksi kalori dilakukan dengan melakukan perbandingan data prediksi kalori dari sistem yang digunakan dengan data kalori treadmill sebagai data sebenarnya atau data \emph{actual score}. Perbandingan hasil prediksi kalori dengan kalori treadmill didapatkan data eror yang kemudian dilakukan perhitungan persentase berdasarkan data sebenarnya pada data kalori treadmill. Proses analisa pengujian dilakukan terhadap seluruh data video pada data percobaan. Hasil analisa yang didapat dengan melakukan analisa pengujian prediksi kalori didapatkan hasil akurasi rata-rata sebesar 53,20\% dengan hasil eror rata-rata sebesar 46,80\%. Tabel \ref{tb:PengujianPosisiSerongAnalisaPrediksiPerhitungan} menunjukkan hasil analisa pengujian dari prediksi kalori dengan perhitungan rumus.

\begin{longtable}{|c|c|c|c|c|c|c|c|}
  \caption{Hasil Prediksi Pengujian Pada Posisi Serong dengan Perhitungan Rumus}
  \label{tb:PengujianPosisiSerongAnalisaPrediksiPerhitungan}                                   \\
  \hline
  \rowcolor[HTML]{C0C0C0}
  & & & & \textbf{Kalori} & \textbf{Prediksi} & & \\
  \rowcolor[HTML]{C0C0C0}
  \multirow{-2}{*}{\textbf{Percobaan}} & \multirow{-2}{*}{\textbf{Kecepatan}} & \multirow{-2}{*}{\textbf{MET}} & \multirow{-2}{*}{\textbf{Waktu}} & \textbf{Treadmill} & \textbf{Kalori} & \multirow{-2}{*}{\textbf{Eror\%}} & \multirow{-2}{*}{\textbf{Akurasi\%}} \\
  \hline
  1   & 3,357   & 3,088    & 2:15    & 10   & 7,903     & 20,97\%     & 79,03\%   \\
  \hline
  2   & 3,416   & 2,834    & 2:13    & 10   & 7,145     & 28,55\%     & 71,45\%   \\
  \hline
  3   & 3,434   & 2,758    & 2:16    & 10   & 7,110     & 28,90\%     & 71,10\%   \\
  \hline
  4   & 15,216  & 13,542   & 2:10    & 20   & 33,376    & 66,88\%     & 33,12\%   \\
  \hline
  5   & 9,034   & 9,380    & 2:10    & 20   & 23,118    & 15,59\%     & 84,41\%   \\
  \hline
  6   & 3,660   & 1,821    & 2:10    & 20   & 4,521     & 77,39\%     & 22,61\%   \\
  \hline
  7   & 9,095   & 9,432    & 1:45    & 20    & 18,775   & 6,12\%      & 93,88\%   \\
  \hline
  8   & 3,729   & 1,547    & 1:46    & 20    & 3,109    & 84,45\%     & 15,55\%   \\
  \hline
  9   & 3,931   & 0,770    & 1:45    & 20    & 1,532    & 92,34\%     & 7,66\%   \\
  \hline

  \multicolumn{6}{|c|}{\textbf{Rata-rata}} & 46,80\% & 53,20\%  \\
  \hline
\end{longtable}

\subsection{Pengujian Pada Posisi Belakang}
\label{subsec:PengujianPosisiBelakang}

Pengujian dilakukan dengan melakukan pengujian performa sistem pada posisi kamera serong. Posisi kamera yang digunakan pada pengujian ini dengan posisi belakang dengan terlihat bagian belakang objek yang dilakukan deteksi. Contoh cuplikan gambar yang memperlihatkan kondisi posisi kamera belakang dapat dilihat pada Gambar \ref{fig:PengujianPosisiBelakang}. Data percobaan yang digunakan dalam pengujian ini dilakukan proses deteksi dengan sistem yang dibuat untuk dapat melakukan proses estimasi dan deteksi langkah maupun waktu. Proses deteksi yang dilakukan pada data percobaan dapat dilihaat pada Gambar \ref{fig:PengujianPosisiBelakang2}. Pengujian dilanjutkan dengan melakukan proses prediksi dari hasil deteksi.

\begin{figure}[H]
  \centering
  \includegraphics[scale=0.4]{gambar/posisi_belakang.png}
  \caption{Pengujian pada posisi belakang}
  \label{fig:PengujianPosisiBelakang}
\end{figure}

\begin{figure}[H]
  \centering
  \includegraphics[scale=0.4]{gambar/posisi_belakang2.png}
  \caption{Hasil deteksi pengujian pada posisi belakang}
  \label{fig:PengujianPosisiBelakang2}
\end{figure}

Pengujian dilakukan setelah didapatkan data hasil deteksi yang telah didapat dan juga data sebenarnya berdasarkan hasil perhitungan yang dilakukan terhadap data video. Dengan data video sebagai banyaknya percobaan dilakukan analisa terkait perbandingan hasil yang didapat antara data hasil deteksi dan data perhitungan sebenarnya. Data langkah merupakan data sebagai data sebenarnya atau \emph{actual score} dengan melakukan perhitungan pada data video dan data deteksi langkah sebagai data hasil deteksi langkah yang didapatkan berdasarkan proses deteksi pada sistem. Perbandingan hasil deteksi dengan data sebenarnya dicantumkan pada data eror dan dilakukan persentase berdasarkan data sebenarnya. Hasil persentase eror mendapatkan hasil persentase akurasi yang juga didapatkan berdasarkan persentase eror. Hasil analisa yang didapat dengan melakukan analisa pengujian hasil deteksi didapatkan hasil akurasi rata-rata sebesar 67,23\% dengan hasil eror rata-rata sebesar 32,77\%. Tabel \ref{tb:PengujianPosisiBelakangAnalisaDeteksi} menunjukkan hasil pengujian dari setiap percobaan yang dilakukan analisa pengujian terhadap hasil deteksi.

\begin{longtable}{|c|c|c|c|c|c|}
  \caption{Hasil Deteksi Pengujian Pada Posisi Belakang}
  \label{tb:PengujianPosisiBelakangAnalisaDeteksi}                                   \\
  \hline
  \rowcolor[HTML]{C0C0C0}
  \textbf{Percobaan} & \textbf{Langkah} & \textbf{Deteksi Langkah} & \textbf{Eror} & \textbf{Eror\%} & \textbf{Akurasi\%} \\
  \hline
  1   & 234   & 194  & 40   & 17,09\%    & 82,91\%   \\
  \hline
  2   & 235   & 150  & 85   & 36,17\%    & 63,83\%   \\
  \hline
  3   & 249   & 236  & 13   & 5,22\%     & 94,78\%   \\
  \hline
  4   & 361   & 92   & 269  & 74,52\%    & 25,48\%   \\
  \hline
  5   & 345   & 234  & 111  & 32,17\%    & 67,83\%   \\
  \hline
  6   & 343   & 272  & 71   & 20,70\%    & 79,30\%   \\
  \hline
  7   & 276   & 138  & 138  & 50\%       & 50\%   \\
  \hline
  8   & 267   & 192  & 75   & 28,09\%    & 71,91\%   \\
  \hline
  9   & 284   & 196  & 88   & 30,99\%    & 69,07\%   \\
  \hline

  \multicolumn{4}{|c|}{\textbf{Rata-rata}} & 32,77\% & 67,23\% \\
  \hline
\end{longtable}

Pada pengujian prediksi kalori dilakukan berdasarkan hasil deteksi yang didapat. Hasil deteksi berupa banyaknya langkah, jarak dan waktu tempuh dari data video yang digunakan. Data hasil deteksi akan diilakukan untuk proses prediksi kalori dengan menggunakan model regresi linear yang sudah didapatkan dan menghasilkan data pada prediksi kalori. Analisa pengujian hasil prediksi kalori dilakukan dengan melakukan perbandingan data prediksi kalori dari sistem yang digunakan dengan data kalori treadmill sebagai data sebenarnya atau data \emph{actual score}. Perbandingan hasil prediksi kalori dengan kalori treadmill didapatkan data eror yang kemudian dilakukan perhitungan persentase berdasarkan data sebenarnya pada data kalori treadmill. Proses analisa pengujian dilakukan terhadap seluruh data video pada data percobaan. Hasil analisa yang didapat dengan melakukan analisa pengujian prediksi kalori didapatkan hasil akurasi rata-rata sebesar 49,86\% dengan hasil eror rata-rata sebesar 50,14\%. Tabel \ref{tb:PengujianPosisiBelakangAnalisaPrediksiRegresi} menunjukkan hasil analisa pengujian dari setiap percobaan yang dilakukan analisa prediksi kalori dengan regresi linear.

\begin{longtable}{|c|c|c|c|c|c|c|c|}
  \caption{Hasil Prediksi Pengujian Pada Posisi Belakang dengan Regresi Linear}
  \label{tb:PengujianPosisiBelakangAnalisaPrediksiRegresi}                                   \\
  \hline
  \rowcolor[HTML]{C0C0C0}
  & & & & \textbf{Kalori} & \textbf{Prediksi} & & \\
  \rowcolor[HTML]{C0C0C0}
  \multirow{-2}{*}{\textbf{Percobaan}} & \multirow{-2}{*}{\textbf{Langkah}} & \multirow{-2}{*}{\textbf{Jarak}} & \multirow{-2}{*}{\textbf{Waktu}} & \textbf{Treadmill} & \textbf{Kalori} & \multirow{-2}{*}{\textbf{Eror\%}} & \multirow{-2}{*}{\textbf{Akurasi\%}} \\
  
  \hline
  1   & 194   & 128,024    & 2:17    & 10    & 8,096     & 10,96\%      & 89,04\%   \\
  \hline  
  2   & 150   & 83,096     & 2:16    & 10    & 4,258     & 42,58\%      & 57,42\%  \\
  \hline
  3   & 236   & 137,155    & 2:17    & 10    & 0,453     & 4,53\%       & 95,47\%   \\
  \hline
  4   & 92    & 19,994     & 2:12    & 20    & 18,700    & 93,50\%      & 6,50\%  \\
  \hline
  5   & 234   & 132,183    & 2:12    & 20    & 10,801    & 54\%         & 46\%    \\
  \hline
  6   & 272   & 126,359    & 2:11    & 20    & 11,211    & 56,05\%      & 43,95\%   \\
  \hline
  7   & 138   & 81,793     & 1:45    & 20    & 14,341    & 71,70\%      & 28,30\%   \\
  \hline
  8   & 192   & 117,049    & 1:46    & 20    & 11,859    & 59,30\%      & 40,70\%   \\
  \hline
  9   & 196   & 118,975    & 1:46    & 20    & 11,723    & 58,62\%      & 41,38\%   \\
  \hline

  \multicolumn{6}{|c|}{\textbf{Rata-rata}} & 50,14\% & 49,86\%  \\
  \hline
\end{longtable}

Pada pengujian prediksi kalori dilakukan berdasarkan hasil deteksi yang didapat. Hasil deteksi berupa kecepatan, MET dan waktu tempuh dari data video yang digunakan. Data hasil deteksi akan diilakukan untuk proses prediksi kalori dengan menggunakan perhitungan rumus berdasarkan MET yang sudah didapatkan dan menghasilkan data pada prediksi kalori. Analisa pengujian hasil prediksi kalori dilakukan dengan melakukan perbandingan data prediksi kalori dari sistem yang digunakan dengan data kalori treadmill sebagai data sebenarnya atau data \emph{actual score}. Perbandingan hasil prediksi kalori dengan kalori treadmill didapatkan data eror yang kemudian dilakukan perhitungan persentase berdasarkan data sebenarnya pada data kalori treadmill. Proses analisa pengujian dilakukan terhadap seluruh data video pada data percobaan. Hasil analisa yang didapat dengan melakukan analisa pengujian prediksi kalori didapatkan hasil akurasi rata-rata sebesar 51,04\% dengan hasil eror rata-rata sebesar 48,96\%. Tabel \ref{tb:PengujianPosisiBelakangAnalisaPrediksiPerhitungan} menunjukkan hasil analisa pengujian dari prediksi kalori dengan perhitungan rumus.

\begin{longtable}{|c|c|c|c|c|c|c|c|}
  \caption{Hasil Prediksi Pengujian Pada Posisi Belakang dengan Perhitungan Rumus}
  \label{tb:PengujianPosisiBelakangAnalisaPrediksiPerhitungan}                                   \\
  \hline
  \rowcolor[HTML]{C0C0C0}
  & & & & \textbf{Kalori} & \textbf{Prediksi} & & \\
  \rowcolor[HTML]{C0C0C0}
  \multirow{-2}{*}{\textbf{Percobaan}} & \multirow{-2}{*}{\textbf{Kecepatan}} & \multirow{-2}{*}{\textbf{MET}} & \multirow{-2}{*}{\textbf{Waktu}} & \textbf{Treadmill} & \textbf{Kalori} & \multirow{-2}{*}{\textbf{Eror\%}} & \multirow{-2}{*}{\textbf{Akurasi\%}} \\
  \hline
  1   & 3,423   & 2,804    & 2:18    & 10   & 7,283    & 27,17\%      & 72,83\%   \\
  \hline
  2   & 2,346   & 2,529    & 2:16    & 10   & 6,521    & 34,79\%      & 65,21\%   \\
  \hline
  3   & 3,191   & 3,822    & 2:17    & 10   & 9,856    & 1,44\%       & 98,56\%   \\
  \hline
  4   & 0,434   & 2,332    & 2:12    & 20   & 5,837    & 70,82\%      & 29,18\%   \\
  \hline
  5   & 3,178   & 3,884    & 2:12    & 20   & 9,720    & 51,40\%      & 48,60\%   \\
  \hline
  6   & 3,095   & 4,263    & 2:11    & 20   & 10,586   & 47,07\%      & 52,93\%   \\
  \hline
  7   & 2,657   & 6,399    & 1:45    & 20   & 12,738   & 36,31\%      & 63,69\%   \\
  \hline
  8   & 3,729   & 1,547    & 1:46    & 20   & 3,109    & 84,45\%      & 15,55\%   \\
  \hline
  9   & 3,799   & 1,271    & 1:46    & 20   & 2,554    & 87,23\%      & 12,77\%   \\
  \hline

  \multicolumn{6}{|c|}{\textbf{Rata-rata}} & 48,96\% & 51,04\%  \\
  \hline
\end{longtable}

Berdasarkan pengujian yang telah dilakukan terhadap hasil performa deteksi langkah, prediksi kalori dengan regresi dan prediksi kalori dengan perhitungan rumus didapatkan hasil performa dalam nilai akurasi rata-rata dan eror rata-rata dari setiap pengujian. Pada pengujian deteksi langkah didapatkan akurasi rata-rata lebih baik pada posisi kamera serong dengan nilai akurasi rata-rata sebesar 86,80\% sedangkan pada posisi kamera belakang didapat nilai akurasi rata-rata sebesar 67,23\%. Pada pengujian prediksi kalori dengan regresi didapatkan akurasi rata-rata lebih baik pada posisi kamera serong dengan nilai akurasi rata-rata sebesar 70,18\% sedangkan pada posisi kamera belakang didapat nilai akurasi rata-rata sebesar 49,86\%. Pada pengujian prediksi kalori dengan perhitungan rumus didapatkan akurasi rata-rata lebih baik pada posisi kamera serong cm dengan nilai akurasi rata-rata sebesar 53,20\% sedangkan pada posisi kamera belakang didapat nilai akurasi rata-rata sebesar 51,04\%. Gambar \ref{fig:DiagramPosisi} menunjukkan diagram perbandingan performa pengujian dengan nilai akurasi rata-rata berdasarkan posisi kamera.

\begin{figure}[H]
  \centering
  \includegraphics[scale=0.7]{gambar/diagram_posisi.png}
  \caption{Diagram perbandingan performa berdasarkan posisi kamera}
  \label{fig:DiagramPosisi}
\end{figure}

\section{Pengujian Performa Berdasarkan Intensitas Cahaya}
\label{sec:PengujianIntensitas}

Pada pengujian ini dilakukan deteksi dan prediksi kalori dengan menggunakan skenario berdasarkan intensitas cahaya yang berbeda-beda. Skenario pertama yang dilakukan adalah pengujian pada intensitas cahaya rendah, cahaya rendah yang digunakan berdasarkan pada nilai rata-rata kecerahan yang rendah pada setiap frame yang digunakan pada video percobaan. Skenario kedua yang dilakukan adalah pengujian pada intensitas cahaya tinggi, cahaya tinggi yang digunakan berdasarkan pada nilai rata-rata kecerahan yang tinggi pada setiap frame yang digunakan pada video percobaan. Video pengujian yang digunakan dengan banyak percobaan sebanyak 9 kali dengan menggunakan variasi kecepatan dan hasil kalori yang didapatkan. Variasi kecepatan yang digunakan adalah 3 km/jam, 9 km/jam dan 12 km/jam. Kemudian variasi hasil kalori yang didapatkan adalah 10 dan 20 kcal. 

\subsection{Pengujian Pada Intensitas Cahaya Rendah}
\label{subsec:PengujianIntensitasRendah}

Pengujian dilakukan dengan melakukan pengujian performa sistem pada intensitas cahaya rendah. Contoh cuplikan gambar yang memperlihatkan kondisi intensitas cahaya rendah dapat dilihat pada Gambar \ref{fig:PengujianIntensitasRendah}. Data percobaan yang digunakan dalam pengujian ini dilakukan proses deteksi dengan sistem yang dibuat untuk dapat melakukan proses estimasi dan deteksi langkah maupun waktu. Proses deteksi yang dilakukan pada data percobaan dapat dilihaat pada Gambar \ref{fig:PengujianIntensitasRendah2}. Pengujian dilanjutkan dengan melakukan proses prediksi dari hasil deteksi.

\begin{figure}[H]
  \centering
  \includegraphics[scale=0.5]{gambar/cahaya_rendah.png}
  \caption{Pengujian pada intensitas cahaya rendah}
  \label{fig:PengujianIntensitasRendah}
\end{figure}

\begin{figure}[H]
  \centering
  \includegraphics[scale=0.5]{gambar/cahaya_rendah2.png}
  \caption{Hasil deteksi pengujian pada intensitas cahaya rendah}
  \label{fig:PengujianIntensitasRendah2}
\end{figure}

Pengujian dilakukan setelah didapatkan data hasil deteksi yang telah didapat dan juga data sebenarnya berdasarkan hasil perhitungan yang dilakukan terhadap data video. Dengan data video sebagai banyaknya percobaan dilakukan analisa terkait perbandingan hasil yang didapat antara data hasil deteksi dan data perhitungan sebenarnya. Data langkah merupakan data sebagai data sebenarnya atau \emph{actual score} dengan melakukan perhitungan pada data video dan data deteksi langkah sebagai data hasil deteksi langkah yang didapatkan berdasarkan proses deteksi pada sistem. Perbandingan hasil deteksi dengan data sebenarnya dicantumkan pada data eror dan dilakukan persentase berdasarkan data sebenarnya. Hasil persentase eror mendapatkan hasil persentase akurasi yang juga didapatkan berdasarkan persentase eror. Hasil analisa yang didapat dengan melakukan analisa pengujian hasil deteksi didapatkan hasil akurasi rata-rata sebesar 90,69\% dengan hasil eror rata-rata sebesar 9,31\%. Tabel \ref{tb:PengujianIntensitasRendahAnalisaDeteksi} menunjukkan hasil pengujian dari setiap percobaan yang dilakukan analisa pengujian terhadap hasil deteksi.

\begin{longtable}{|c|c|c|c|c|c|}
  \caption{Hasil Deteksi Pengujian Pada Intensitas Cahaya Rendah}
  \label{tb:PengujianIntensitasRendahAnalisaDeteksi}                                   \\
  \hline
  \rowcolor[HTML]{C0C0C0}
  \textbf{Percobaan} & \textbf{Langkah} & \textbf{Deteksi Langkah} & \textbf{Error} & \textbf{Eror\%} & \textbf{Akurasi\%} \\
  \hline
  1   & 274   & 274 & 0    & 0\%       & 100\%   \\
  \hline
  2   & 260   & 260 & 0    & 0\%       & 100\%   \\
  \hline
  3   & 276   & 254 & 22   & 7,97\%    & 92,03\%     \\
  \hline
  4   & 321   & 304 & 17   & 5,30\%    & 94,70\%   \\
  \hline
  5   & 309   & 311 & 2    & 0,65\%    & 99,35\%   \\
  \hline
  6   & 317   & 268 & 49   & 15,46\%   & 84,54\%   \\
  \hline
  7   & 276   & 228 & 48   & 17,39\%   & 82,61\%   \\
  \hline
  8   & 259   & 258 & 1    & 0,39\%    & 99,61\%   \\
  \hline
  9   & 292   & 185 & 107  & 36,64\%   & 63,36\%   \\
  \hline

  \multicolumn{4}{|c|}{\textbf{Rata-rata}} & 9,31\% & 90,69\% \\
  \hline
\end{longtable}

Pada pengujian prediksi kalori dilakukan berdasarkan hasil deteksi yang didapat. Hasil deteksi berupa banyaknya langkah, jarak dan waktu tempuh dari data video yang digunakan. Data hasil deteksi akan diilakukan untuk proses prediksi kalori dengan menggunakan model regresi linear yang sudah didapatkan dan menghasilkan data pada prediksi kalori. Analisa pengujian hasil prediksi kalori dilakukan dengan melakukan perbandingan data prediksi kalori dari sistem yang digunakan dengan data kalori treadmill sebagai data sebenarnya atau data \emph{actual score}. Perbandingan hasil prediksi kalori dengan kalori treadmill didapatkan data eror yang kemudian dilakukan perhitungan persentase berdasarkan data sebenarnya pada data kalori treadmill. Proses analisa pengujian dilakukan terhadap seluruh data video pada data percobaan. Hasil analisa yang didapat dengan melakukan analisa pengujian prediksi kalori didapatkan hasil akurasi rata-rata sebesar 73,68\% dengan hasil eror rata-rata sebesar 26,32\%. Tabel \ref{tb:PengujianIntensitasRendahAnalisaPrediksiRegresi} menunjukkan hasil analisa pengujian dari setiap percobaan yang dilakukan analisa prediksi kalori dengan regresi linear.

\begin{longtable}{|c|c|c|c|c|c|c|c|}
  \caption{Hasil Prediksi Pengujian Pada Intensitas Cahaya Rendah dengan Regresi Linear}
  \label{tb:PengujianIntensitasRendahAnalisaPrediksiRegresi}                                   \\
  \hline
  \rowcolor[HTML]{C0C0C0}
  & & & & \textbf{Kalori} & \textbf{Prediksi} & & \\
  \rowcolor[HTML]{C0C0C0}
  \multirow{-2}{*}{\textbf{Percobaan}} & \multirow{-2}{*}{\textbf{Langkah}} & \multirow{-2}{*}{\textbf{Jarak}} & \multirow{-2}{*}{\textbf{Waktu}} & \textbf{Treadmill} & \textbf{Kalori} & \multirow{-2}{*}{\textbf{Eror\%}} & \multirow{-2}{*}{\textbf{Akurasi\%}} \\
  
  \hline
  1   & 274   & 175,936    & 2:49    & 10    & 12,267   & 23,67\%    & 77,33\%   \\
  \hline  
  2   & 260   & 177,780    & 2:49    & 10    & 12,397   & 23,97\%    & 76,03\%  \\
  \hline
  3   & 254   & 176,568    & 2:49    & 10    & 12,312   & 23,12\%    & 76,88\%   \\
  \hline
  4   & 304   & 263,156    & 2:01    & 20    & 18,422   & 7,89\%     & 92,11\%  \\
  \hline
  5   & 311   & 291,916    & 2:02    & 20    & 20,447   & 2,23\%     & 97,77\%    \\
  \hline
  6   & 268   & 143,707    & 2:02    & 20    & 10,013   & 49,93\%    & 50,07\%   \\
  \hline
  7   & 228   & 169,023    & 1:39    & 20    & 11,802   & 40,99\%    & 59,01\%   \\
  \hline
  8   & 258   & 269,480    & 1:40    & 20    & 18,874   & 5,63\%     & 94,37\%   \\
  \hline
  9   & 185   & 113,831    & 1:45    & 20    & 7,915    & 60,42\%    & 39,58\%   \\
  \hline

  \multicolumn{6}{|c|}{\textbf{Rata-rata}} & 26,32\% & 73,68\% \\
  \hline
\end{longtable}

Pada pengujian prediksi kalori dilakukan berdasarkan hasil deteksi yang didapat. Hasil deteksi berupa kecepatan, MET dan waktu tempuh dari data video yang digunakan. Data hasil deteksi akan diilakukan untuk proses prediksi kalori dengan menggunakan perhitungan rumus berdasarkan MET yang sudah didapatkan dan menghasilkan data pada prediksi kalori. Analisa pengujian hasil prediksi kalori dilakukan dengan melakukan perbandingan data prediksi kalori dari sistem yang digunakan dengan data kalori treadmill sebagai data sebenarnya atau data \emph{actual score}. Perbandingan hasil prediksi kalori dengan kalori treadmill didapatkan data eror yang kemudian dilakukan perhitungan persentase berdasarkan data sebenarnya pada data kalori treadmill. Proses analisa pengujian dilakukan terhadap seluruh data video pada data percobaan. Hasil analisa yang didapat dengan melakukan analisa pengujian prediksi kalori didapatkan hasil akurasi rata-rata sebesar 58,08\% dengan hasil eror rata-rata sebesar 41,92\%. Tabel \ref{tb:PengujianIntensitasRendahAnalisaPrediksiPerhitungan} menunjukkan hasil analisa pengujian dari prediksi kalori dengan perhitungan rumus.

\begin{longtable}{|c|c|c|c|c|c|c|c|}
  \caption{Hasil Prediksi Pengujian Pada Intensitas Cahaya Rendah dengan Perhitungan Rumus}
  \label{tb:PengujianIntensitasRendahAnalisaPrediksiPerhitungan}                                   \\
  \hline
  \rowcolor[HTML]{C0C0C0}
  & & & & \textbf{Kalori} & \textbf{Prediksi} & & \\
  \rowcolor[HTML]{C0C0C0}
  \multirow{-2}{*}{\textbf{Percobaan}} & \multirow{-2}{*}{\textbf{Kecepatan}} & \multirow{-2}{*}{\textbf{MET}} & \multirow{-2}{*}{\textbf{Waktu}} & \textbf{Treadmill} & \textbf{Kalori} & \multirow{-2}{*}{\textbf{Eror\%}} & \multirow{-2}{*}{\textbf{Akurasi\%}} \\
  \hline
  1   & 3,572   & 2,178    & 2:50    & 10   & 6,979    & 30,21\%     & 69,79\%   \\
  \hline
  2   & 3,820   & 1,192    & 1:35    & 10   & 3,819    & 61,81\%     & 38,19\%   \\
  \hline
  3   & 3,880   & 0,966    & 2:02    & 10   & 3,095    & 69,05\%     & 30,95\%   \\
  \hline
  4   & 7,987   & 8,337    & 1:41    & 20   & 19,125   & 4,38\%      & 95,62\%   \\
  \hline
  5   & 8,735   & 9,114    & 2:03    & 20   & 21,079   & 5,40\%      & 94,60\%   \\
  \hline
  6   & 3,653   & 1,852    & 1:21    & 20   & 4,389    & 78,05\%     & 21,95\%   \\
  \hline
  7   & 6,319   & 5,818    & 1:39    & 20   & 10,920   & 45,40\%     & 54,60\%   \\
  \hline
  8   & 9,869   & 10,024   & 1:37    & 20   & 19,004   & 4,98\%      & 95,05\%   \\
  \hline
  9   & 3,574   & 2,171    & 1:45    & 20   & 4,403    & 77,98\%     & 22,02\%   \\
  \hline

  \multicolumn{6}{|c|}{\textbf{Rata-rata}} & 41,92\% & 58,08\%  \\
  \hline
\end{longtable}


\subsection{Pengujian Pada Intensitas Cahaya Tinggi}
\label{subsec:PengujianIntensitasTinggi}

Pengujian dilakukan dengan melakukan pengujian performa sistem pada intensitas cahaya tinggi. Contoh cuplikan gambar yang memperlihatkan kondisi intensitas cahaya tinggi dapat dilihat pada Gambar \ref{fig:PengujianIntensitasTinggi}. Data percobaan yang digunakan dalam pengujian ini dilakukan proses deteksi dengan sistem yang dibuat untuk dapat melakukan proses estimasi dan deteksi langkah maupun waktu. Proses deteksi yang dilakukan pada data percobaan dapat dilihaat pada Gambar \ref{fig:PengujianIntensitasTinggi2}. Pengujian dilanjutkan dengan melakukan proses prediksi dari hasil deteksi.

\begin{figure}[H]
  \centering
  \includegraphics[scale=0.5]{gambar/cahaya_tinggi.png}
  \caption{Pengujian pada intensitas cahaya tinggi}
  \label{fig:PengujianIntensitasTinggi}
\end{figure}

\begin{figure}[H]
  \centering
  \includegraphics[scale=0.5]{gambar/cahaya_tinggi2.png}
  \caption{Hasil deteksi pengujian pada intensitas cahaya tinggi}
  \label{fig:PengujianIntensitasTinggi2}
\end{figure}

Pengujian dilakukan setelah didapatkan data hasil deteksi yang telah didapat dan juga data sebenarnya berdasarkan hasil perhitungan yang dilakukan terhadap data video. Dengan data video sebagai banyaknya percobaan dilakukan analisa terkait perbandingan hasil yang didapat antara data hasil deteksi dan data perhitungan sebenarnya. Data langkah merupakan data sebagai data sebenarnya atau \emph{actual score} dengan melakukan perhitungan pada data video dan data deteksi langkah sebagai data hasil deteksi langkah yang didapatkan berdasarkan proses deteksi pada sistem. Perbandingan hasil deteksi dengan data sebenarnya dicantumkan pada data eror dan dilakukan persentase berdasarkan data sebenarnya. Hasil persentase eror mendapatkan hasil persentase akurasi yang juga didapatkan berdasarkan persentase eror. Hasil analisa yang didapat dengan melakukan analisa pengujian hasil deteksi didapatkan hasil akurasi rata-rata sebesar 95,54\% dengan hasil eror rata-rata sebesar 4,46\%. Tabel \ref{tb:PengujianIntensitasTinggiAnalisaDeteksi} menunjukkan hasil pengujian dari setiap percobaan yang dilakukan analisa pengujian terhadap hasil deteksi.

\begin{longtable}{|c|c|c|c|c|c|}
  \caption{Hasil Deteksi Pengujian Pada Intensitas Cahaya Tinggi}
  \label{tb:PengujianIntensitasTinggiAnalisaDeteksi}                                   \\
  \hline
  \rowcolor[HTML]{C0C0C0}
  \textbf{Percobaan} & \textbf{Langkah} & \textbf{Deteksi Langkah} & \textbf{Eror} & \textbf{Eror\%} & \textbf{Akurasi\%} \\
  \hline
  1   & 274   & 274 & 0    & 0\%        & 100\%   \\
  \hline
  2   & 260   & 256 & 4    & 1,54\%     & 99,46\%   \\
  \hline
  3   & 276   & 268 & 8    & 2,90\%     & 97,10\%     \\
  \hline
  4   & 321   & 334 & 13   & 4,05\%     & 95,95\%   \\
  \hline
  5   & 309   & 312 & 3    & 0,97\%     & 99,03\%   \\
  \hline
  6   & 317   & 296 & 21   & 6,62\%     & 93,38\%   \\
  \hline
  7   & 276   & 240 & 36   & 13,04\%    & 86,96\%   \\
  \hline
  8   & 259   & 260 & 1    & 0,39\%     & 99,61\%   \\
  \hline
  9   & 292   & 261 & 31   & 10,62\%    & 89,38\%   \\
  \hline
  

  \multicolumn{4}{|c|}{\textbf{Rata-rata}} & 4,46\% & 95,54\% \\
  \hline
\end{longtable}

Pada pengujian prediksi kalori dilakukan berdasarkan hasil deteksi yang didapat. Hasil deteksi berupa banyaknya langkah, jarak dan waktu tempuh dari data video yang digunakan. Data hasil deteksi akan diilakukan untuk proses prediksi kalori dengan menggunakan model regresi linear yang sudah didapatkan dan menghasilkan data pada prediksi kalori. Analisa pengujian hasil prediksi kalori dilakukan dengan melakukan perbandingan data prediksi kalori dari sistem yang digunakan dengan data kalori treadmill sebagai data sebenarnya atau data \emph{actual score}. Perbandingan hasil prediksi kalori dengan kalori treadmill didapatkan data eror yang kemudian dilakukan perhitungan persentase berdasarkan data sebenarnya pada data kalori treadmill. Proses analisa pengujian dilakukan terhadap seluruh data video pada data percobaan. Hasil analisa yang didapat dengan melakukan analisa pengujian prediksi kalori didapatkan hasil akurasi rata-rata sebesar 73,61\% dengan hasil eror rata-rata sebesar 26,39\%. Tabel \ref{tb:PengujianIntensitasTinggiAnalisaPrediksiRegresi} menunjukkan hasil analisa pengujian dari setiap percobaan yang dilakukan analisa prediksi kalori dengan regresi linear.

\begin{longtable}{|c|c|c|c|c|c|c|c|}
  \caption{Hasil Prediksi Pengujian Pada Intensitas Cahaya Tinggi dengan Regresi Linear}
  \label{tb:PengujianIntensitasTinggiAnalisaPrediksiRegresi}                                   \\
  \hline
  \rowcolor[HTML]{C0C0C0}
  & & & & \textbf{Kalori} & \textbf{Prediksi} & & \\
  \rowcolor[HTML]{C0C0C0}
  \multirow{-2}{*}{\textbf{Percobaan}} & \multirow{-2}{*}{\textbf{Langkah}} & \multirow{-2}{*}{\textbf{Jarak}} & \multirow{-2}{*}{\textbf{Waktu}} & \textbf{Treadmill} & \textbf{Kalori} & \multirow{-2}{*}{\textbf{Eror\%}} & \multirow{-2}{*}{\textbf{Akurasi\%}} \\
  
  \hline
  1   & 274   & 175,936    & 2:49    & 10    & 12,267   & 22,67\%      & 77,33\%   \\
  \hline  
  2   & 256   & 177,101    & 2:49    & 10    & 12,349   & 23,49\%      & 76,51\%  \\
  \hline
  3   & 268   & 177,545    & 2:49    & 10    & 12,381   & 23,81\%      & 76,19\%   \\
  \hline
  4   & 334   & 533,158    & 2:01    & 20    & 37,430   & 87,15\%      & 12,85\%  \\
  \hline
  5   & 312   & 298,637    & 2:02    & 20    & 20,920   & 4,60\%       & 95,40\%    \\
  \hline
  6   & 296   & 212,781    & 2:02    & 20    & 14,876   & 25,62\%      & 74,38\%   \\
  \hline
  7   & 240   & 202,035    & 1:39    & 20    & 14,126   & 29,37\%      & 70,63\%   \\
  \hline
  8   & 260   & 280,461    & 1:40    & 20    & 19,647   & 1,76\%       & 98,24\%   \\
  \hline
  9   & 261   & 231,528    & 1:45    & 20    & 16,201   & 18,99\%      & 81,01\%   \\
  \hline

  \multicolumn{6}{|c|}{\textbf{Rata-rata}} & 26,39\% & 73,61\% \\
  \hline
\end{longtable}

Pada pengujian prediksi kalori dilakukan berdasarkan hasil deteksi yang didapat. Hasil deteksi berupa kecepatan, MET dan waktu tempuh dari data video yang digunakan. Data hasil deteksi akan diilakukan untuk proses prediksi kalori dengan menggunakan perhitungan rumus berdasarkan MET yang sudah didapatkan dan menghasilkan data pada prediksi kalori. Analisa pengujian hasil prediksi kalori dilakukan dengan melakukan perbandingan data prediksi kalori dari sistem yang digunakan dengan data kalori treadmill sebagai data sebenarnya atau data \emph{actual score}. Perbandingan hasil prediksi kalori dengan kalori treadmill didapatkan data eror yang kemudian dilakukan perhitungan persentase berdasarkan data sebenarnya pada data kalori treadmill. Proses analisa pengujian dilakukan terhadap seluruh data video pada data percobaan. Hasil analisa yang didapat dengan melakukan analisa pengujian prediksi kalori didapatkan hasil akurasi rata-rata sebesar 68,92\% dengan hasil eror rata-rata sebesar 31,08\%. Tabel \ref{tb:PengujianIntensitasTinggiAnalisaPrediksiPerhitungan} menunjukkan hasil analisa pengujian dari prediksi kalori dengan perhitungan rumus.

\begin{longtable}{|c|c|c|c|c|c|c|c|}
  \caption{Hasil Prediksi Pengujian Pada Intensitas Cahaya Tinggi dengan Perhitungan Rumus}
  \label{tb:PengujianIntensitasTinggiAnalisaPrediksiPerhitungan}                                   \\
  \hline
  \rowcolor[HTML]{C0C0C0}
  & & & & \textbf{Kalori} & \textbf{Prediksi} & & \\
  \rowcolor[HTML]{C0C0C0}
  \multirow{-2}{*}{\textbf{Percobaan}} & \multirow{-2}{*}{\textbf{Kecepatan}} & \multirow{-2}{*}{\textbf{MET}} & \multirow{-2}{*}{\textbf{Waktu}} & \textbf{Treadmill} & \textbf{Kalori} & \multirow{-2}{*}{\textbf{Eror\%}} & \multirow{-2}{*}{\textbf{Akurasi\%}} \\
  \hline
  1   & 3,572   & 2,178    & 2:49    & 10   & 6,979   & 30,21\%     & 69,79\%   \\
  \hline
  2   & 3,862   & 1,029    & 2:49    & 10   & 3,296   & 67,04\%     & 32,96\%   \\
  \hline
  3   & 3,697   & 1,675    & 2:49    & 20   & 5,367   & 46,33\%     & 53,67\%   \\
  \hline
  4   & 15,114  & 13,453   & 2:01    & 20   & 30,862  & 54,31\%     & 45,69\%   \\
  \hline
  5   & 8,923   & 9,283    & 2:02    & 20   & 21,471  & 7,36\%      & 92,64\%   \\
  \hline
  6   & 6,427   & 6,023    & 2:02    & 20   & 13,930  & 30,35\%     & 69,65\%   \\
  \hline
  7   & 7,589   & 7,846    & 1:39    & 20   & 14,726  & 26,37\%     & 73,63\%   \\
  \hline
  8   & 10,234  & 10,270   & 1:40    & 20   & 19,470  & 2,65\%      & 97,35\%   \\
  \hline
  9   & 8,160   & 8,533    & 1:45    & 20   & 16,985  & 15,07\%     & 84,93\%   \\
  \hline

  \multicolumn{6}{|c|}{\textbf{Rata-rata}} & 31,08\% & 68,92\%  \\
  \hline
\end{longtable}

Berdasarkan pengujian yang telah dilakukan terhadap hasil performa deteksi langkah, prediksi kalori dengan regresi dan prediksi kalori dengan perhitungan rumus didapatkan hasil performa dalam nilai akurasi rata-rata dan eror rata-rata dari setiap pengujian. Pada pengujian deteksi langkah didapatkan akurasi rata-rata lebih baik pada intensitas cahaya tinggi dengan nilai akurasi rata-rata sebesar 95,54\% sedangkan pada intensitas cahaya rendah didapat nilai akurasi rata-rata sebesar 90,69\%. Pada pengujian prediksi kalori dengan regresi didapatkan akurasi rata-rata lebih baik pada intensitas cahaya rendah dengan nilai akurasi rata-rata sebesar 73,68\% sedangkan pada intensitas cahaya tinggi didapat nilai akurasi rata-rata sebesar 73,61\%. Pada pengujian prediksi kalori dengan perhitungan rumus didapatkan akurasi rata-rata lebih baik pada intensitas cahaya tinggi cm dengan nilai akurasi rata-rata sebesar 68,92\% sedangkan pada intensitas cahaya rendah didapat nilai akurasi rata-rata sebesar 58,08\%. Gambar \ref{fig:DiagramPosisi} menunjukkan diagram perbandingan performa pengujian dengan nilai akurasi rata-rata berdasarkan intensitas cahaya.

\begin{figure}[H]
  \centering
  \includegraphics[scale=0.7]{gambar/diagram_cahaya.png}
  \caption{Diagram perbandingan performa berdasarkan intensitas cahaya}
  \label{fig:DiagramPosisi}
\end{figure}

\section{Pengujian Sistem secara \emph{Real Time}}
\label{sec:PengujianRealTime}

Pengujian dilakukan dengan melakukan pengujian sistem secara \emph{real Time} dengan melakukan proses sistem secara langsung pada objek yang dideteksi. Proses pengujian dilakukan dengan mengambil data video yang bersumber pada kamera yang akan langsung diproses pada perangkat yang digunakan untuk proses sistem. Data video yang digunakan akan digunakan sebagai data video percobaan dan data pengujian secara bersamaan. Percobaan pada pengujian yang dilakukan sebanyak 6 kali dengan menggunakan variasi kecepatan dan hasil kalori yang didapatkan. Variasi kecepatan yang digunakan adalah 4 km/jam dan 12 km/jam. Kemudian variasi hasil kalori yang didapatkan adalah 5 dan 20 kcal.

Data percobaan yang digunakan dalam pengujian ini dilakukan proses deteksi dengan sistem yang dibuat untuk dapat melakukan proses estimasi dan deteksi langkah maupun waktu. Proses deteksi yang dilakukan pada data percobaan dapat dilihaat pada Gambar \ref{fig:PengujianRealTime2}. Pengujian dilanjutkan dengan melakukan proses prediksi dari hasil deteksi.

\begin{figure}[H]
  \centering
  \includegraphics[scale=0.5]{gambar/realtimedeteksi.png}
  \caption{Hasil deteksi pengujian secara \emph{real time}}
  \label{fig:PengujianRealTime2}
\end{figure}

Pengujian dilakukan setelah didapatkan data hasil deteksi yang telah didapat dan juga data sebenarnya berdasarkan hasil perhitungan yang dilakukan terhadap data video. Dengan data video sebagai banyaknya percobaan dilakukan analisa terkait perbandingan hasil yang didapat antara data hasil deteksi dan data perhitungan sebenarnya. Data langkah merupakan data sebagai data sebenarnya atau \emph{actual score} dengan melakukan perhitungan pada data video dan data deteksi langkah sebagai data hasil deteksi langkah yang didapatkan berdasarkan proses deteksi pada sistem. Perbandingan hasil deteksi dengan data sebenarnya dicantumkan pada data eror dan dilakukan persentase berdasarkan data sebenarnya. Hasil persentase eror mendapatkan hasil persentase akurasi yang juga didapatkan berdasarkan persentase eror. Hasil analisa yang didapat dengan melakukan analisa pengujian hasil deteksi didapatkan hasil akurasi rata-rata sebesar 81,35\% dengan hasil eror rata-rata sebesar 18,65\%. Tabel \ref{tb:PengujianRealTimeAnalisaDeteksi} menunjukkan hasil pengujian dari setiap percobaan yang dilakukan analisa pengujian terhadap hasil deteksi.

\begin{longtable}{|c|c|c|c|c|c|}
  \caption{Hasil Deteksi Pengujian Sistem secara \emph{Real Time}}
  \label{tb:PengujianRealTimeAnalisaDeteksi}                                   \\
  \hline
  \rowcolor[HTML]{C0C0C0}
  \textbf{Percobaan} & \textbf{Langkah} & \textbf{Deteksi Langkah} & \textbf{Eror} & \textbf{Eror\%} & \textbf{Akurasi\%} \\
  \hline
  1   & 105   & 101 & 4    & 3,81\%     & 96,19\%   \\
  \hline
  2   & 125   & 124 & 1    & 0,80\%     & 99,20\%   \\
  \hline
  3   & 117   & 104 & 13   & 11,11\%    & 88,89\%     \\
  \hline
  4   & 255   & 214 & 41   & 16,08\%    & 83,92\%   \\
  \hline
  5   & 269   & 230 & 39   & 14,50\%    & 85,50\%   \\
  \hline
  6   & 220   & 218 & 187  & 65,61\%    & 34,39\%   \\
  \hline

  \multicolumn{4}{|c|}{\textbf{Rata-rata}} & 18,65\% & 81,35\% \\
  \hline
\end{longtable}

Pada pengujian prediksi kalori dilakukan berdasarkan hasil deteksi yang didapat. Hasil deteksi berupa banyaknya langkah, jarak dan waktu tempuh dari data video yang digunakan. Data hasil deteksi akan diilakukan untuk proses prediksi kalori dengan menggunakan model regresi linear yang sudah didapatkan dan menghasilkan data pada prediksi kalori. Analisa pengujian hasil prediksi kalori dilakukan dengan melakukan perbandingan data prediksi kalori dari sistem yang digunakan dengan data kalori treadmill sebagai data sebenarnya atau data \emph{actual score}. Perbandingan hasil prediksi kalori dengan kalori treadmill didapatkan data eror yang kemudian dilakukan perhitungan persentase berdasarkan data sebenarnya pada data kalori treadmill. Proses analisa pengujian dilakukan terhadap seluruh data video pada data percobaan. Hasil analisa yang didapat dengan melakukan analisa pengujian prediksi kalori didapatkan hasil akurasi rata-rata sebesar 64,05\% dengan hasil eror rata-rata sebesar 35,95\%. Tabel \ref{tb:PengujianRealTimeAnalisaPrediksiRegresi} menunjukkan hasil analisa pengujian dari setiap percobaan yang dilakukan analisa prediksi kalori dengan regresi linear.

\begin{longtable}{|c|c|c|c|c|c|c|c|}
  \caption{Hasil Prediksi Pengujian Sistem secara \emph{Real Time} dengan Regresi Linear}
  \label{tb:PengujianRealTimeAnalisaPrediksiRegresi}                                   \\
  \hline
  \rowcolor[HTML]{C0C0C0}
  & & & & \textbf{Kalori} & \textbf{Prediksi} & & \\
  \rowcolor[HTML]{C0C0C0}
  \multirow{-2}{*}{\textbf{Percobaan}} & \multirow{-2}{*}{\textbf{Langkah}} & \multirow{-2}{*}{\textbf{Jarak}} & \multirow{-2}{*}{\textbf{Waktu}} & \textbf{Treadmill} & \textbf{Kalori} & \multirow{-2}{*}{\textbf{Eror\%}} & \multirow{-2}{*}{\textbf{Akurasi\%}} \\
  
  \hline 
  1   & 101   & 59,396     & 1:13    & 5    & 4,092    & 18,16\%      & 81,84\%   \\
  \hline  
  2   & 124   & 82,546     & 1:13    & 5    & 5,722    & 14,44\%      & 85,56\%  \\
  \hline
  3   & 104   & 62,357     & 1:10    & 5    & 4,301    & 13,98\%      & 86,02\%   \\
  \hline
  4   & 214   & 361,373    & 1:49    & 20    & 25,352  & 26,76\%      & 73,24\%  \\
  \hline
  5   & 230   & 468,476    & 1:50    & 20    & 32,891  & 64,45\%      & 35,55\%    \\
  \hline
  6   & 98    & 63,921     & 1:41    & 20    & 4,414   & 77,98\%      & 22,07\%   \\
  \hline

  \multicolumn{6}{|c|}{\textbf{Rata-rata}} & 35,95\% & 64,05\% \\
  \hline
\end{longtable}

Pada pengujian prediksi kalori dilakukan berdasarkan hasil deteksi yang didapat. Hasil deteksi berupa kecepatan, MET dan waktu tempuh dari data video yang digunakan. Data hasil deteksi akan diilakukan untuk proses prediksi kalori dengan menggunakan perhitungan rumus berdasarkan MET yang sudah didapatkan dan menghasilkan data pada prediksi kalori. Analisa pengujian hasil prediksi kalori dilakukan dengan melakukan perbandingan data prediksi kalori dari sistem yang digunakan dengan data kalori treadmill sebagai data sebenarnya atau data \emph{actual score}. Perbandingan hasil prediksi kalori dengan kalori treadmill didapatkan data eror yang kemudian dilakukan perhitungan persentase berdasarkan data sebenarnya pada data kalori treadmill. Proses analisa pengujian dilakukan terhadap seluruh data video pada data percobaan. Hasil analisa yang didapat dengan melakukan analisa pengujian prediksi kalori didapatkan hasil akurasi rata-rata sebesar 46,63\% dengan hasil eror rata-rata sebesar 53,37\%. Tabel \ref{tb:PengujianRealTimeAnalisaPrediksiPerhitungan} menunjukkan hasil analisa pengujian dari prediksi kalori dengan perhitungan rumus.

\begin{longtable}{|c|c|c|c|c|c|c|c|}
  \caption{Hasil Prediksi Pengujian Sistem secara \emph{Real Time} dengan Perhitungan Rumus}
  \label{tb:PengujianRealTimeAnalisaPrediksiPerhitungan}                                   \\
  \hline
  \rowcolor[HTML]{C0C0C0}
  & & & & \textbf{Kalori} & \textbf{Prediksi} & & \\
  \rowcolor[HTML]{C0C0C0}
  \multirow{-2}{*}{\textbf{Percobaan}} & \multirow{-2}{*}{\textbf{Kecepatan}} & \multirow{-2}{*}{\textbf{MET}} & \multirow{-2}{*}{\textbf{Waktu}} & \textbf{Treadmill} & \textbf{Kalori} & \multirow{-2}{*}{\textbf{Eror\%}} & \multirow{-2}{*}{\textbf{Akurasi\%}} \\
  \hline
  1   & 2,795   & 5,702    & 1:13    & 5    & 7,891   & 57,83\%      & 42,17\%   \\
  \hline
  2   & 4,533   & 1,320    & 1:13    & 5    & 1,751   & 64,97\%      & 35,03\%   \\
  \hline
  3   & 3,275   & 3,488    & 1:10    & 5    & 4,575   & 8,49\%       & 91,51\%   \\
  \hline
  4   & 4,925   & 2,510    & 1:49    & 20   & 4,807   & 75,97\%      & 24,03\%   \\
  \hline
  5   & 5,983   & 5,140    & 1:50    & 20   & 9,842   & 50,79\%      & 49,21\%   \\
  \hline
  6   & 0,875   & 3,952    & 1:41    & 20   & 7,568   & 62,16\%      & 37,84\%   \\
  \hline

  \multicolumn{6}{|c|}{\textbf{Rata-rata}} & 53,37\% & 46,63\% \\
  \hline
\end{longtable}
\cleardoublepage

% Bab 5 penutup
\chapter{PENUTUP}
\label{chap:penutup}

% Ubah bagian-bagian berikut dengan isi dari penutup

\section{Kesimpulan}
\label{sec:kesimpulan}

Berdasarkan hasil pengujian yang dilakukan dengan perhitungan analisa performa dan hasil yang dilakukan mengenai sistem prediksi kalori yang terbakar, didapatkan beberapa kesimpulan sebagai berikut:

\begin{enumerate}[nolistsep]

  \item Penelitian sistem prediksi kalori yang terbakar saat berolahraga dengan treadmill berbasis kamera dengan menggunakan \emph{convolutional neural network} dapat bekerja dengan baik dengan melakukan hasil akhir nilai prediksi kalori yang terbakar menggunakan kamera.

  \item Berdasarkan hasil pengujian dataset yang digunakan sebagai klasifikasi langkah dengan melakukan pengambilan data didapatkan hasil akurasi \emph{training} sebesar 0.951 dan \emph{validation} sebesar 0.977, sedangkan hasil \emph{loss} pada \emph{training} sebesar 0.127 dan \emph{validation} sebesar 0.059.

  \item Berdasarkan hasil pengujian \emph{testing} model yang digunakan didapatkan hasil akurasi sebesar 95\% dengan hasil deteksi kelas kanan dengan benar sebanyak 170 sampel (100\%) dan kelas kiri sebanyak 161 sampel (91\%).

  \item Pengujian hasil deteksi dengan model yang dibuat untuk melakukan klasifikasi langkah didapatkan hasil akurasi deteksi berdasarkan model klasifikasi sebesar 99,36\% dengan hasil error sebesar 0,64\%.
  
  \item Pengujian hasil prediksi dengan sistem yang digunakan berdasarkan metode regresi didapat hasil akurasi sebesar 93,61\% dengan error sebesar 6,39\% dan berdasarkan metode perhitungan rumus didapat hasil akurasi sebesar 81,03\% dengan error sebesar 18,97\%.

\end{enumerate}

\section{Saran}
\label{chap:saran}

Untuk pengembangan lebih lanjut pada \lipsum[1][1-3] antara lain:

\begin{enumerate}[nolistsep]

  \item Memperbaiki \lipsum[2][1-3]

  \item \lipsum[2][4-6]

  \item \lipsum[2][7-10]

\end{enumerate}

\cleardoublepage

\chapter*{DAFTAR PUSTAKA}
\addcontentsline{toc}{chapter}{DAFTAR PUSTAKA}

\begingroup
%\renewcommand{\section}[2]{}%
\renewcommand{\chapter}[2]{}% for other classes
\begin{thebibliography}{}
  \bibitem{ano05}
    Okmayura, F., Effendi, N., Ramadhani, W., Jefiza, A. (2019). Analysis and Design of Calories Burning Calculation in Jogging Using Thresholding Based Accelerometer Sensor. Advances in Engineering Research, vol 190. https://doi.org/10.2991/iccelst-st-19.2019.3
  \bibitem{oe04}
    Utami, D., B. dan Ichwan, M. (2017). Sistem Prediksi Kalori Terbakar Pada Pesepeda Menggunakan Feedforward Neural Network. https:// https://lib.itenas.ac.id/kti/?p=5185
  \bibitem{oe04}
    Saponaro, P., Wei, H., Dominick, G., Kambhamettu, C. (2019). Estimating Physical Activity Intensity And Energy Expenditure Using Computer Vision On Videos. https://doi.org/10.1109/ICIP.2019.8803535
  \bibitem{oe04}
    P. Ilmiah, M. Ajidarma, P. S. Informatika, F. Komunikasi, D. A. N. Informatika, and U. M. Surakarta. (2019). Aplikasi perhitungan kebutuhan kalori dan perhitungan kalori dari makanan yang dikonsumsi.
  \bibitem{oe04}
    F. T. Informasi. (2016). Pengembangan Sistem Monitoring Aktivitas Fisik User Bergerak dengan Analisa Langkah ( Step Analysis ) untuk Estimasi Pembakaran Kalori secara Real-Time.
  \bibitem{oe04}
    W. Widiantini et al. 2013. Aktivitas Fisik , Stres , dan Obesitas pada Pegawai Negeri Sipil Physical Activity , Stress and Obesity among Civil Servant, no. 4.
  \bibitem{oe04}
    D. Kurniawan. (2008). Regresi Linier. Statistic, vol. 2, no. 3, pp. 1–6.
  \bibitem{oe04}
    Syilfi, D. S., Ispriyanti, D. (2012). Analsis Regresi. J. Gaussin, vol. 1.
  \bibitem{oe04}
    P. Sulardi, T. Hendro, and F. R. Umbara. (2017). Prediksi Kebutuhan Obat Menggunakan Regresi Linier. Pros. SNATIF, vol. 0, no. 0, pp. 57–62.
  \bibitem{oe04}
    P. Ilmiah, R. D. Nurfita, P. S. Informatika, F. Komunikasi, D. A. N. Informatika, and U. M. Surakarta. (2018). Implementasi Deep Learning Berbasis Tensorflow.
  \bibitem{oe04}
    P. A. Nugroho, I. Fenriana, R. Arijanto, and M. Kom. (2020). Implementasi Deep Learning Menggunakan Convolutional Neural Network ( CNN ) pada Ekspresi Manusia, vol. 1.
  \bibitem{oe04}
    R. Mehindra Prasmatio, B. Rahmat, and I. Yuniar. (2020). Deteksi Dan Pengenalan Ikan Menggunakan Algoritma Convolutional Neural Network.  J. Inform. dan Sist. Inf., vol. 1, no. 2, pp. 510–521.
  \bibitem{oe04}
    H. Fonda. (2020). Klasifikasi Batik Riau Dengan Menggunakan Convolutional Neural Networks (Cnn). J. Ilmu Komput., vol. 9, no. 1, pp. 7–10. doi: 10.33060/jik/2020/vol9.iss1.144.
  \bibitem{oe04}
    W. S. Eka Putra. (2016). Klasifikasi Citra Menggunakan Convolutional Neural Network (CNN) pada Caltech 101. J. Tek. ITS, vol. 5, no. 1, 2016, doi: 10.12962/j23373539.v5i1.15696.
  \end{thebibliography}
\endgroup

%\renewcommand\refname{}
%\vspace{2ex}
%\renewcommand{\bibname}{}
%\begingroup
%\def\chapter*#1{}
%\printbibliography
%\endgroup
\cleardoublepage

% Biografi penulis
\begin{center}
  \Large
  \textbf{BIOGRAFI PENULIS}
\end{center}

\addcontentsline{toc}{chapter}{BIOGRAFI PENULIS}

\vspace{2ex}

\begin{wrapfigure}{L}{0.3\textwidth}
  \centering
  \vspace{-3ex}
  % Ubah file gambar berikut dengan file foto dari mahasiswa
  \includegraphics[width=0.3\textwidth]{gambar/biodimas.jpg}
  \vspace{-4ex}
\end{wrapfigure}

% Ubah kalimat berikut dengan biografi dari mahasiswa
\name{}, atau biasa dipanggil Dimas, lahir di Bondowoso, Jawa Timur pada 5 Desember 2000. Merupakan anak kedua dari dua saudara. Penulis lulus dari SMP Negeri 1 Bondowoso dan melanjutkan ke SMA Negeri 2 Bondowoso. Penulis melanjutkan ke jenjang strata satu di Departemen Teknik Komputer Fakultas Teknologi Elektro dan Informatika Cerdas Institut Teknologi Sepuluh Nopember (ITS). Dalam masa kuliah, penulis tertarik dengan jaringan komputer, pengembangan Robotika dan \emph{Internet of Things} (IoT), \emph{Web Design} dan UI/UX. Penulis pernah aktif menjadi salah satu anggota kru hingga menjadi \emph{quality control} dari ITS TV (2020-2023) dan aktif dalam organisasi mahasiswa yaitu anggota staf Departemen Komunikasi dan Informasi (Kominfo) dan Wakil Departemen \emph{Relation and Comunication} HIMATEKKOM ITS (2021-2023). Pada penelitian akhir ini, penulis memilih mengembangkan penelitian di bidang \emph{Machine Learning} yang berfokus pada Visi Komputer dalam lingkup olahraga pada treadmill. Bagi pembaca yang memiliki kritik, saran, atau pertanyaan mengenai tugas akhir ini dapat menghubungi penulis melalui surel dimas.adityamf@gmail.com.


\cleardoublepage

\end{document}
